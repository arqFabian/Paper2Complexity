%! Author = arqfa
%! Date = 8/8/2023

% Preamble
%\documentclass[11pt]{article}
\documentclass[final,5p,times]{elsarticle}
% Packages
%% The amssymb package provides various useful mathematical symbols
\usepackage{amssymb}
%% The amsthm package provides extended theorem environments
\usepackage{amsthm}
\usepackage{graphicx}
\usepackage{tabularx}
\usepackage{booktabs}
%% The lineno packages adds line numbers. Start line numbering with
%%\begin{linenumbers}, end it with \end{linenumbers}. Or switch it on
%% for the whole article with \linenumbers
\usepackage{lineno}
\modulolinenumbers[10]
\usepackage{enumitem}

\usepackage{multirow}

\usepackage{caption}

\journal{Computer Aid Design}


% Document
\begin{document}

\begin{frontmatter}


\title{Embracing Complexity: Mixed Reality Assessment of User Tolerance for Complex Facades in Future Construction Trends}

%% use optional labels to link authors explicitly to addresses:
%% \author[label1,label2]{}
%% \affiliation[label1]{organization={},
%%             addressline={},
%%             city={},
%%             postcode={},
%%             state={},
%%             country={}}
%%
%% \affiliation[label2]{organization={},
%%             addressline={},
%%             city={},
%%             postcode={},
%%             state={},
%%             country={}}

\author[inst1]{Fabian Jarrin}
\author[inst1]{Yasuko Koga}
\author[inst2]{Diego Thomas}

\affiliation[inst1]{organization={Graduate School of Human-Environment Studies, Department of Architecture and Urban Design, Kyushu University},%Department and Organization
            addressline={744 Motooka, Nishi Ward},
            city={Fukuoka},
            postcode={819-0382},
            state={Kyushu},
            country={Japan}}

\affiliation[inst2]{organization={Graduate School of Information Science and Electrical Engineering, Kyushu University},%Department and Organization
            addressline={744 Motooka, Nishi Ward},
            city={Fukuoka},
            postcode={819-0382},
            state={Kyushu},
            country={Japan}}

\begin{abstract}
%% Text of abstract
%!Asbtract guidelines
        %A concise and factual abstract (100-150 words ) is required. The abstract should state briefly the purpose of the research, the principal results and major conclusions. An abstract is often presented separately from the article, so it must be able to stand alone. For this reason, References should be avoided, but if essential, then cite the author(s) and year(s). Also, non-standard or uncommon abbreviations should be avoided, but if essential they must be defined at their first mention in the abstract itself.



% check
% new version under reviewer guidelines: Abstracts should contain only 6 short sentences: 1) What is the problem being addressed? 2) What is the research question being asked? 3) What is the methodology being used to answer the stated research question? 4) What are the results obtained? 5) What is the meaning and importance of these results? 6) What are the directions for follow-up research?


Architectural practice is evolving through digital fabrication, enabling complex designs that challenge the uniformity of barren walls and fully glazed facades that often dominate contemporary streetscapes.
This study investigates user tolerance and acceptance of complex facades using Virtual Reality (VR) and Computer Vision through the Computational Image Complexity Analysis (CICA) system.
We applied VR simulations and CICA to quantitatively and qualitatively assess reactions to various facade complexities.
Results reveal a preference for moderate complexity, suggesting an ideal balance between simplicity and intricacy.
This highlights the importance of aligning architectural complexity with user preferences to enhance sustainability and satisfaction.
Future research should explore the long-term impact of complex facades on user well-being and environmental sustainability.


%version 2024/05
%This research uses virtual reality (VR) assessment and computer vision to explore complexity in architectural facade design.
%We aim to examine user tolerance and acceptance of complex facades, providing insights for future construction practices.
%A literature review confirms a trend towards increased complexity, preferring richly detailed facades with elements at various scales or materials with fractal qualities, diverging from modernist minimalism.
%We introduce the Computational Image Complexity Analysis (CICA) system to quantify this trend, revealing an upward complexity trajectory since the late 20th century.
%A VR experiment indicates a user preference for moderate complexity, suggesting a balance between intricacy and simplicity.
%Discrepancies between participant perceptions and CICA rankings highlight the subjective nature of complexity perception.
%Qualitative data suggests a shift towards customizable, user-responsive designs.
%Overall, the study underscores a shift towards embracing complexity in facade design, emphasizing the need for a balanced approach that aligns with user preferences and cultural contexts.

% version 2024/04
%This research uses virtual reality (VR) assessment, and computer vision to understand complexity in architectural facade design.
%We aim to examine user tolerance and acceptance of complex facades, providing insights for future construction practices.
%A literature review confirms a contemporary trend towards increased facade complexity, moving away from modernist minimalism.
%We introduce the `Computational Image Complexity Analysis' (CICA) system to quantify this trend, revealing an upward complexity trajectory since the late 20th century.
%A VR experiment indicates a user preference for moderate complexity, suggesting a balance between intricacy and simplicity in future architectural trends.
%Discrepancies between participant perceptions and CICA rankings highlight the subjective nature of complexity perception.
%Qualitative data suggests a shift towards customizable, user-responsive designs.
%Overall, the study underscores a contemporary shift towards embracing complexity in facade design, emphasizing the need for a balanced approach that aligns with user preferences and cultural contexts.



%previous
%This paper examines user perceptions and acceptance of complex facade designs in contemporary architecture, integrating digital fabrication, virtual reality (VR), and computer vision.
%Our research aims to shed light on the evolving trends in construction design.
%A literature review reveals a historical oscillation between simplicity and complexity in architectural styles.
%We introduce the `Computational Image Complexity Analysis' (CICA) method to quantitatively evaluate the complexity of building facades.
%This approach verifies a noticeable trend towards greater complexity since the late 20th century.
%In our VR experiment, participants evaluated various facades, expressing a preference for designs with a CICA complexity score of 3.82 out of 10.
%Subsequent surveys indicated a positive reception towards intricate designs, with an average rating of 4.9 on a 7-point scale.
%These outcomes align with our architectural analysis, indicating a continuing trend towards increased complexity in modern architectural design.




\end{abstract}


%%Graphical abstract

\begin{graphicalabstract}
    \centering
    \includegraphics[width= \textwidth, trim = 0 80 0 80, clip]{Images/GraphicAbstract.pdf}
    \label{fig:graphic_abstract}
\end{graphicalabstract}

%%Research highlights
\begin{highlights}
%highlights
% highlights
% These bullet points should capture the novel results of your research as well as new methods that were used during the study (if any).
% Think of them as the "elevator pitch" of your article. Please include terms that you know your readers will be looking for online. Don't try to capture all ideas, concepts or conclusions as highlights are meant to be short:
% 85 characters or fewer, including spaces.


\item Architectural design faces challenges in quantifying facade complexity effectively.
\item CICA system developed using VR and Computer Vision to assess facade complexity.
\item CICA system aids in historical trend analysis of architectural complexity.
\item Findings show a 9\% deviation between CICA system complexity scores and user perceptions.
\item Study highlights user preference for moderate complexity, balancing design intricacy.


%previous iteration
%\item Investigates VR and CV methods for quantifying facade complexity in architectural design.
%\item CICA system integrates VR and CV to measure complexity and align with user perceptions.
%\item CICA system aids in historical trend analysis of architectural complexity.
%\item Study reveals an average 9\% deviation between system measurements and user perceptions.
%\item Findings show preference for moderate complexity, balancing simplicity and intricacy.


\end{highlights}

\begin{keyword}
%% keywords here, in the form: keyword \sep keyword
Data-driven design\sep Site Layout Planning \sep Virtual reality \sep Optimization\sep

\end{keyword}

\end{frontmatter}
%\linenumbers
%\modulolinenumbers[10]
%
\begin{linenumbers}


\section{Introduction}
\label{sec:1Introduction}
%%State the objectives of the work and provide an adequate background, avoiding a detailed literature survey or a summary of the results.
%%Introduction

%=================================
%%Reference
%%https://www.scribbr.com/research-paper/research-paper-introduction/
%%State the objectives of the work and provide an adequate background, avoiding a detailed literature survey or a summary of the results.

%Step1. Introduce your topic.
     %This is generally accomplished with a strong opening hook.
%Step2. Describe the background.
     %For a paper describing original research, you’ll instead provide an overview of the most relevant research that has already been conducted.
%Step3. Establish your research problem.
     %In an empirical research paper, try to lead into the problem on the basis of your discussion of the literature.
%Step4. Specify your objective(s).
     %The research question is the question you want to answer in an empirical research paper. If your research involved testing hypotheses, these should be stated along with your research question.
%Step 5: Map out your paper.
     %The final part of the introduction is often dedicated to a brief overview of the rest of the paper.

%recommended limit 500 words
%=================================

Recent advancements in Building Information Modeling (BIM) and digital fabrication are transforming architectural practice.
These technologies enable architects to design intricate and complex forms, moving beyond the uniformity of barren walls and fully glazed facades that often dominate contemporary streetscapes.
By leveraging these advancements, architects can introduce complexity and detail into their designs, enhancing both the visual and functional aspects of buildings, and creating more engaging and dynamic environments that potentially redefine the relationship between form and function~\cite{Leach2016}.

\deleted{
However, the pursuit of complexity in architectural design must be balanced with sustainability and user satisfaction.
Designs that are overly complex without consideration of these factors can quickly become outdated and disconnected from their inhabitants, leading to issues of obsolescence and lack of relevance~\cite{Oberfrancova2021}.
Understanding how complexity can enhance both environmental sustainability and user satisfaction is therefore crucial for modern architectural practice.
}

\added{
However, the pursuit of complexity in architectural design must be balanced with sustainability and user satisfaction.
Overly complex designs, when not thoughtfully integrated, can quickly become outdated, contributing to obsolescence and construction waste, a major source of carbon emissions~\cite{Oberfrancova2021}.
By incorporating complexity analysis into the design process and controlling and optimizing facade complexity, architects can create designs that are not only visually engaging but also adaptable and long-lasting, reducing the need for frequent renovations and replacements.
}

\added{
Understanding the complexity of facade designs is crucial because facades are the most visible part of a building, playing a significant role in urban aesthetics and user perception. Designs that strike the right balance between simplicity and complexity can create environments that are not only visually stimulating but also comfortable and functional for occupants~\cite{Browning2014}. Complexity can enhance the user experience, making facades more engaging and improving user satisfaction with the built environment. Moreover, facade design can contribute to energy efficiency and material optimization, especially when combined with advanced technologies like digital fabrication and parametric design.
}

\added{
This study proposes the development of a system to measure and adjust facade complexity, which could be integrated with existing tools for energy efficiency, material optimization, and environmental comfort.
Such an approach could significantly minimize environmental impact while addressing the sustainability challenges in modern construction.
Understanding how complexity can enhance both environmental sustainability and user satisfaction is therefore crucial for modern architectural practice.
}

\deleted{
Previous research has extensively explored the impact of complexity in architectural design, identifying mathematical relationships between complexity and aesthetic value ~\cite{Bies2016, Douchova2016, Redies2015}.
Despite these insights, the architectural field has yet to develop frameworks that leverage these principles for practical design applications, especially considering modern technological advancements aimed at sustainability.
}

\added{
Previous research has extensively explored the impact of complexity in architectural design, identifying mathematical relationships between complexity and aesthetic value ~\cite{Bies2016, Douchova2016, Redies2015}. Despite these insights, the architectural field has yet to develop frameworks that leverage these principles for practical design applications, especially considering modern technological advancements such as digital fabrication and parametric design. These technologies not only enable the creation of complex forms but when paired with `Data-driven Building Design' (DBD) optimization they also support energy efficiency, material reduction, and long-term sustainability.
}

This study aims to bridge the gap between theoretical understanding and practical application by developing a methodology to measure facade complexity.
The objectives are to generate data that can improve DBD by integrating a complexity scoring function that can inform on the optimal rate between simplicity and complexity based on historical analysis and user preferences.
By integrating complexity insights with modern technological applications, we seek to provide actionable, data-driven insights for future architectural practices promoting the advancement aimed at sustainability.

The methodology is structured around 4 primary components:

\begin{enumerate}
    \item Literature review: Significant studies on the foundational theories of complexity, and an exploration of the fluctuation between simplicity and complexity in architectural history.
    \item Complexity Analysis System Development: \deleted{Implements a Virtual Reality (VR) framework, and combines it with a Computational Image Complexity Analysis (CICA) component using computer vision (CV) algorithms to quantitatively assess the complexity of facade designs.}\added{This component integrates a Virtual Reality (VR) framework with the Computational Image Complexity Analysis (CICA) system, specifically developed for this study. The CICA system utilizes computer vision (CV) algorithms to quantitatively assess facade design complexity.}
    \item Experiment Execution: involving VR to facilitate participant interaction with complex facades, augmented by surveys and interviews for qualitative insight.
    \item Data Analysis and Validation: Assessing the data collected during the experiment to evaluate the effectiveness of the Complexity Analysis System and CICA framework in measuring complexity and user preferences.
\end{enumerate}

This comprehensive approach aims to enrich our understanding of facade complexity and its role in the contemporary Architectural, Engineering, and Construction (AEC) industry.



\section{Research Background}
\label{sec:Background}
%% A Theory section should extend, not repeat, the background to the article already dealt with in the Introduction and lay the foundation for further work. In contrast, a Calculation section represents a practical development from a theoretical basis.
%% Research Background

%% A Theory section should extend, not repeat, the background to the article already dealt with in the Introduction and lay the foundation for further work. In contrast, a Calculation section represents a practical development from a theoretical basis.


The act of building our environment has its roots in our attempt to emulate nature. As Baker \cite{Baker2003} defines it, architecture creates a universe crafted by humans for humans. While buildings may originate from nature and try to blend with their surroundings, they remain distinct entities in the continuous environment, serving as private domains dedicated to humanity. However, the challenges of the future, as pointed out by Knippers et al. \cite{Knippers2021}, lie in constructing more buildings while minimizing pollutant emissions, conserving non-renewable resources, and ensuring the creation of high-quality and sustainable living spaces.

Regardless of the scale of the building project, whether it be a single housing unit or a towering skyscraper, Site Layout Planning (SLP) design remains a crucial and acknowledged step in the construction process across the entire Architecture, Engineering, and Construction (AEC) field. 

        %The act of building our environment started with our attempt to emulate nature and as Baker \cite{Baker2003} defined it, a universe created by man for man, and while it may have originated from nature in the semblance of the world it is by itself a separate entity from the otherwise continuous environment, and even though a building can attempt to blend with its surroundings it remains as a private domain dedicated to humanity. Knippers et al. \cite{Knippers2021} believe that the challenge for building in the future will be to build more, while emitting fewer pollutants, consuming less non-renewable resources, and still ensuring the outcome of a high-quality and liveable built environment. 
        
        

\subsection{Site Layout Planning}
\label{subsec:SLP}
As described by Ning \cite{Ning2011}, SLP involves a decision-making process encompassing problem identification, recognizing opportunities, developing solutions, selecting the best alternatives, and executing them. This critical phase sets the foundation for successful building construction and plays a pivotal role in achieving optimized outcomes that balance various design objectives, economic factors, and environmental considerations.
        %The process to build successfully a building includes several stages and regardless of the scale of that endeavor, be it in the form of a single housing unit or the planning of a skyscraper, Site Layout Planning (SLP) design continues to be recognized as a critical step in the construction process all across the AEC field. Ning \cite{Ning2011} defines SLP as a “decision-making process, which involves identifying problems and opportunities, developing solutions, choosing the best alternative and implementing it.” 

Kulabi et al. \cite{Kulabi2020} identified that the problem of Site Layout Planning (SLP) has been solved by two main approaches which are heuristic methods and mathematical optimization. 
Heuristic methods depend on knowledge-based systems and they are adopted to give good but not optimal results \cite{Kulabi2020}. Traditionally the construction industry has established rules and regulations to help establish these guidelines to achieve a successful Site Layout Planning design and in real life, the execution of most projects continues to be guided by heuristics   \cite{Augenbroe2012} and is done based on the experience of planners, and as a consequence, the level of accuracy and reliability vary among different projects \cite{Yi2018}.

Mathematical optimization methods appear to improve the current circumstances with the objective of improving the reliability of the results and a more stable outcome across field striving to achieve optimal results, however, Kulabi et al. \cite{Kulabi2020} explains that due to the complexity in computations derived by large projects, mathematical optimizations are not being applied to them and heuristic methods dominate on these projects. 

\subsection{Data-driven Building Design Optimization}
\label{subsec:PBD}

The continuous research towards mathematical optimization methods proves the allure of the promise for optimal design. The recurrent venture into this method has created a trend in data-driven building design optimization where a solution to a building project is no longer a single static result instead it is seen as a process from which an assort of solutions can be presented with various degrees of optimization, while still given the final choice to the stakeholders.

It is understood that traditionally design solutions were driven by prescriptive terms, rather than the expected performance of the solution, with building codes and regulations being the main contributors to prescriptive specifications \cite{Augenbroe2012} and as Hemsath \cite{Hemsath2012} argues, data-driven design, such as Performance-based Design, requires a deliberate approach with a focus on front-loading information and reducing the time between critical feedback.

Aside from the data which sits at the core of the data-driven building design, an accurate building performance simulation is needed to support the design evolution, not only for reviewing purposes but on the role of a virtual experiment \cite{Augenbroe2012}.

An example that explores the data-driven strategy in the field of Site Layout Planning (SLP) is given by Yi et al. \cite{Yi2018} who propose a mathematical model to solve the problem of allocating the temporary facilities needed for a construction Site Layout considering a multi-objective optimization approach that covers several safety, health and environmental concerns as well as transportation costs. Leveraging on the potential of computation to solve a non-linear optimization model that deals with different criteria, the team verified that compared to traditional heuristic methods the proposed optimization mathematical model could outperform the traditional approach improvement up to 19\% even when they tested it on a real scenario, proving the advantages of data-driven design optimization. 

The benefits of this approach in practice, recognize “that predicting and analyzing building behavior is more efficient and economical than resolving these issues once the building is built” \cite{Hemsath2012}. These benefits extend to Site Layout Planning.

Beyond the analysis of the strategies to solve the process of Site Layout Planning, it is important to recognize the way a building is considered in relation to its site, what Baker \cite{Baker2003} referred to as the site forces (which include orientation, views, access), and how they interact with the organizational forces identified within the building.

To include these variants implied for this exercise a combination between performance-based design and simulation with the traditional heuristic approach with a renewed interaction that takes advantage of Virtual reality as a new frontier, with the purpose of accelerating the adoption of data-driven design by the general public. 

\subsection{Virtual Reality (VR) and its impact on the AEC field}
\label{subsec:VR and AEC}

The Architecture, Engineering, and Construction (AEC) industry are striving to keep pace with the rapid advancements in technology \cite{Moonhubwebsite}. Despite innovations in building materials and existing CAD tools, the industry continues to face hazards and inefficiencies, making it increasingly challenging to attract new workers, leading to a labor shortage. For instance, countries like the UK are experiencing a record high of 37,000 job vacancies in the construction sector \cite{ConstructionVacancies}. In this competitive landscape, AEC firms that invest in modernization and embrace the latest technologies position themselves ahead of the competition due to their adaptability \cite{Alizadehsalehi2020}.

    %The Architecture, Engineering and Construction (AEC) industry is also trying to keep up with the current advancements of technology \cite{Moonhubwebsite}, regardless of the innovation in building materials and existing CAD tools the work in the AEC industry remains hazardous, plagued with inefficiencies which have made so that continuously more difficult to attract new workers creating the current environment of labor shortage which in countries like UK reach a new record high of 37000 vacancies\cite{ConstructionVacancies}. That is why the AEC firms that invest in modernization and continuously embrace the latest technologies  will position themselves ahead of the competition due to their willingness to adapt \cite{Alizadehsalehi2020}.  

To address safety, labor, and efficiency issues, the AEC industry is actively exploring various strategies. A pivotal technology in this pursuit is Virtual Reality (VR), as highlighted by a 2016 poll conducted by ARC Document Solutions \cite{ARCdocument}, where 65.7\% of respondents recognized VR as the most critical technology impacting key aspects of the industry. As Schnabel \cite{Schnabel2009} accurately points out, the emergence of VR and other forms of Mixed Reality will have a profound impact on the way we perceive the world, where reality redefines itself anew and will stand as one realm among others on the spectrum ranging from reality to virtuality. 
        %And as Schnabels \cite{Schnabel2009} accurately points out, the influence of the emergence of VR and other forms of Mixed Reality, will have a profound impact on the way we perceive the world, where reality redefines itself anew and will stand as one realm among others on the spectrum ranging from reality to virtuality.
        
VR offers a range of unique advantages, including enhanced visualization, faster project completion, reduced labor requirements, and minimized material usage. One of the most remarkable features of VR technology is its ability to provide an immersive experience of a building before it is physically constructed—a capability unmatched by traditional schematics or concept renderings \cite{Moonhubwebsite}. These computer-generated realities merge with the perception of reality and are specifically tailored to enhance and communicate key aspects guiding a building's design \cite{Schnabel2009}\cite{wang-2009}. By immersing stakeholders in a virtual environment, VR allows for interaction with designed spaces at a human scale, and facilitates a deeper understanding of building designs, enabling more informed decision-making and efficient project execution \cite{Castronovo2013}. As the AEC industry embraces VR technology, it becomes better equipped to address its challenges, drive innovation, and revolutionize the way buildings are designed and constructed. 
    
An example of this technology is the "VR smart" tool \cite{Wolfartsberger2019}, a VR-based tool created to support engineering design review. The “VR smart” team evaluated the usability of this app by comparing it to the mainstream approach that uses 2D documentations paired with a PC that shows on-screen CAD data. The “VR smart” app was then tested in a real industrial environment and an authentic engineering review. The results show that users of the VR system could identify slightly more errors in a design and the interaction with the system showed a much faster entry into the design review, which aligns with the evidence that the information becomes more accessible to groups with less knowledge of engineering documentation. 

For the specific case of Site Layout Planning (SLP), there have also been research studies on integrating Virtual reality into solving the SLP problem. A recent example of this was given by  Muhammad et al. \cite{Muhammad2020} that developed a VR simulation of a site with the intention of evaluating job site organization for Site Layout Planning, collision detection and assessment of construction site layout arrangement with the final purpose of evaluating the impact that adding VR technology has on solving the SLP problem and, in a similar fashion as the “VR smart” team, compare it to the traditional approach of 2D documentation review by conducting a survey pos-testing the simulation. The results showed that while VR accelerated the rate of comprehension of the participants regarding the Site Layout and increased the identification of collision points, it also showed that traditional 2D methods are easier to understand and less time-consuming when dealing with tasks that require a bigger overview of the whole site such as Site Layout Planning, however, they denoted that this preference could also be due to the inexperience of the participants when using VR headsets.

The authors of the previous examples \cite{Wolfartsberger2019} \cite{Muhammad2020} have confirmed the relevance that VR will have as a new tool to help the way that we interact and approach a project. Therefore, the possibility of harvesting the potential of VR technology in facilitating the comprehension of complex sets of data represented as responsive graphs with immediate feedback could prove to be the path toward embracing data-driven design optimization.

\end{linenumbers}
    \bibliography{main}
    %\bibliographystyle{plain}
    \bibliographystyle{elsarticle-num}

\end{document}