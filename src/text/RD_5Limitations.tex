%!% limitations: What can’t your results tell us?

\textbf{Limitations}

The findings of this research provide valuable insights  into architectural complexity, yet certain limitations warrant cautious interpretation of the results:
%1 Sample Size and Demographic Representation:
The study involved a relatively small sample of 10 participants, primarily composed by university students and faculty members, which may limit the generalizability of the findings.

%2 Virtual Reality Environment:
While VR provides a controlled and immersive environment for assessing user preferences, it may not fully replicate the experience of interacting with real-world facades.
The VR setting might influence participants' perceptions of complexity and comfort.

%3 CICA System Metrics and Historical Analysis Dataset: :
The CICA system's metrics for assessing complexity are based on specific architectural features.
There might be other factors contributing to perceived complexity that are not captured by the system.
The study highlights the subjective nature of complexity perception.
Individual differences in aesthetic preferences, prior experiences, and cultural influences can affect how complexity is perceived and evaluated.
Additionaly, the CICA system for Historical Analysis was based on a limited Dataset of 177 buildings.
While this provides a valuable overview, a larger and more comprehensive dataset could offer a more nuanced understanding of architectural complexity trends over time.

%4 Focus on Facade Design
This research's emphasis on facade design does not encompass the full scope of architectural complexity.

The study concentrated on facade design, which is just one aspect of architectural complexity.


%!% limitations: What can’t your results tell us?

%\textbf{Limitations}
%
%The findings of this research provide valuable insights, however, it is important to interpret the results with caution due to certain limitations.
%
%However, it's essential to recognize that every research endeavor carries its unique set of limitations, and our study is no exception.
%In the spirit of transparency, we candidly acknowledge these limitations, which we will address.
%Nevertheless, the following subsections, which meticulously analyze our results, were arranged in accordance with the core themes and research questions that guided our inquiry.

        %The findings of this research provide valuable insights into the potential of VR immersion in data-driven design for Site Layout Planning, as well as potential implications for other areas of the building design process. However, it is important to interpret the results with caution due to certain limitations. The generalizability of the findings is limited by the small sample size and the fact that the participants were limited to individuals affiliated with the university.

        %Furthermore, the experiment design aimed to simulate the current methodology used in addressing SLP through a "screen-based interaction" stage, which involved CAD plans, perspectives, and expert recommendations typically provided to design teams.

        %However, the introduction of data visualization techniques and charts was exclusive to the "VR stage" (see Figure \ref{fig:VRinterface}). As a result, it is challenging to determine the exact contribution of VR-based interaction versus more efficient data visualization techniques to the observed accuracy improvement. It can be speculated that if the same graphs and interface used in the VR stage were introduced to the screen-based stage, the observed accuracy improvement may have been different.