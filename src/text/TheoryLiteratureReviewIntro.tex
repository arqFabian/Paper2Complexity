%% Research Background

%% A Theory section should extend, not repeat, the background to the article already dealt with in the Introduction and lay the foundation for further work. In contrast, a Calculation section represents a practical development from a theoretical basis.

%A theoretical framework is a foundational review of existing theories that serves as a roadmap for developing the arguments you will use in your own work.

%=============================================================================
%Outline.
%1. Introduction
%2. The importance of facade in a building.
%3. A brief analysis of architecture styles and the shift from Images and simplicity
%3. A reflexion into digital fabrication techniques specially parametric design.
%4. The principle of data driven design.
%5. An analysis into mixed reality and the advantages of introducing this  into the design review.
%=============================================================================

%///////////////////////////////////////////////////////////////
%% refined outline of theory framework

%% Introduction

%!Concise intro

This section explores the multi-faceted nature of architectural complexity, from its historical evolution to contemporary technological advancements.
It aims to provide a comprehensive understanding of how complexity influences both design practices and user experiences.

This section focuses on four key themes:
\begin{itemize}
    \item Evolution and Theoretical Foundations of Architectural Complexity: Explores the historical context and foundational theories of complexity in architecture, including Birkhoff's ratio of order to complexity.
    \item Technological Advancements: Analyzes the impact of BIM, digital fabrication, and VR on architectural design and user perception.
    \item User Perception and Quantitative Analysis of Complexity: Investigates the psychological and physiological impacts of complexity on users, principles of biophilic design, and gaps in aligning data-driven design with user satisfaction.
    This includes quantitative methods and computational studies on complexity.
    \item Gaps in Current Research: Identifies the existing gaps in the research, highlighting the need for comprehensive frameworks that integrate theoretical principles with practical applications, and discusses the challenges and opportunities of rapidly advancing technologies.
\end{itemize}

These insights will be instrumental for validating CICA findings by comparing empirical data with theoretical insights into architectural evolution, confirming observed patterns of simplicity and complexity.




