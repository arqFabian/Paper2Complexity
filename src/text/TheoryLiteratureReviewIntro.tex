%% Research Background

%% A Theory section should extend, not repeat, the background to the article already dealt with in the Introduction and lay the foundation for further work. In contrast, a Calculation section represents a practical development from a theoretical basis.

%A theoretical framework is a foundational review of existing theories that serves as a roadmap for developing the arguments you will use in your own work.

%=============================================================================
%Outline.
%1. Introduction
%2. The importance of facade in a building.
%3. A brief analysis of architecture styles and the shift from Images and simplicity
%3. A reflexion into digital fabrication techniques specially parametric design.
%4. The principle of data driven design.
%5. An analysis into mixed reality and the advantages of introducing this  into the design review.
%=============================================================================

%///////////////////////////////////////////////////////////////
%% refined outline of theory framework

%% Introduction

%!Concise intro

This section delves into the multifaceted nature of architectural complexity, examining its historical evolution and the theoretical foundations that underpin contemporary architectural practices.
It aims to provide a comprehensive understanding of how complexity influences both design practices and user experiences.
The literature review is organized into two key themes:

\begin{itemize}
    \item \textit{Evolution of Architectural Styles:} focusing on the historical transitions between simplicity and complexity in architecture, highlighting significant shifts across different periods.
    \item \textit{Theoretical Foundations of Architectural Complexity:} exploring foundational theories and previous research surrounding complexity in architecture, offering insights into how these theories inform current design practices.
\end{itemize}

\added{By linking historical analysis, and complexity theories with user perception studies, this review provides a framework for understanding how past architectural trends in complexity influence modern user preferences, allowing us to explore whether contemporary design trends align with these historical patterns.}
These insights are crucial for validating the CICA findings by comparing empirical data with theoretical perspectives on architectural evolution, thereby confirming observed patterns of simplicity and complexity.

