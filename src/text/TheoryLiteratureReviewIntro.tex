%% Research Background

%% A Theory section should extend, not repeat, the background to the article already dealt with in the Introduction and lay the foundation for further work. In contrast, a Calculation section represents a practical development from a theoretical basis.

%A theoretical framework is a foundational review of existing theories that serves as a roadmap for developing the arguments you will use in your own work.

%=============================================================================
%Outline.
%1. Introduction
%2. The importance of facade in a building.
%3. A brief analysis of architecture styles and the shift from Images and simplicity
%3. A reflexion into digital fabrication techniques specially parametric design.
%4. The principle of data driven design.
%5. An analysis into mixed reality and the advantages of introducing this  into the design review.
%=============================================================================

%///////////////////////////////////////////////////////////////
%% refined outline of theory framework

%6. **Human-Centric Design Philosophy:** Explore the historical evolution of architectural philosophies that prioritize human experience and well-being. Discuss how different architectural styles and movements have addressed the balance between ornamentation, functionality, and human comfort.

%7. **Cultural Significance of Facades:** Delve deeper into how facades reflect cultural values, societal norms, and historical contexts. Explore how different societies and civilizations have expressed their identity through architectural ornamentation and symbolism.

%8. **Environmental Sustainability:** Investigate how the integration of complex facades and digital fabrication aligns with contemporary sustainability principles. Examine how parametric design and data-driven approaches can enhance energy efficiency, reduce material waste, and contribute to sustainable construction practices.

%9. **Ethics and Social Responsibility:** Consider the ethical implications of embracing ornate designs and digital fabrication in a world grappling with issues of resource scarcity and inequality. Discuss how responsible architectural practices can address these challenges and contribute positively to society.

%10. **Collaborative Design Process:** Explore the collaborative aspects of architectural design when utilizing digital fabrication and mixed reality. Discuss how these technologies foster interdisciplinary collaboration among architects, engineers, urban planners, and other stakeholders.

%11. **Challenges and Limitations:** Address the potential challenges and limitations of implementing complex facades and digital fabrication techniques. Discuss technical, economic, and cultural barriers that may arise and the strategies to overcome them.

%12. **Case Studies and Success Stories:** Showcase real-world examples of projects that have successfully integrated complex facades, digital fabrication, and mixed reality. Analyze the impact of these projects on user experience, urban aesthetics, and architectural innovation.

%13. **User-Centered Design:** Delve into the principles of user-centered design and human factors that influence the acceptance of intricate facades. Consider factors like psychological comfort, emotional response, and sensory experience in relation to complex architectural designs.

%14. **Cognitive Aspects of Mixed Reality:** Explore the cognitive psychology behind mixed reality experiences and how they influence user perceptions of architectural Images. Consider how elements like immersion, presence, and interaction affect users' understanding and appreciation of design.

%15. **Future Trends and Speculation:** Speculate on the potential future trajectories of architectural design, digital fabrication, and mixed reality. Discuss how these trends might shape the built environment, and propose ideas for further research and exploration.

%By incorporating these additional topics into your theory framework, you can provide a comprehensive and well-rounded context for your research, offering deeper insights into the historical, cultural, ethical, and technological dimensions of complex facades, digital fabrication, and mixed reality in architecture.
%///////////////////////////////////////////////////////////////

%% Introduction

%!Concise intro

This section explores the multi-faceted nature of architectural complexity, from its historical evolution to contemporary technological advancements.
It aims to provide a comprehensive understanding of how complexity influences both design practices and user experiences.

This section focuses on four key themes:
\begin{itemize}
    \item Evolution and Theoretical Foundations of Architectural Complexity: Explores the historical context and foundational theories of complexity in architecture, including Birkhoff's ratio of order to complexity.
    \item Technological Advancements: Analyzes the impact of BIM, digital fabrication, and VR on architectural design and user perception.
    \item User Perception and Quantitative Analysis of Complexity: Investigates the psychological and physiological impacts of complexity on users, principles of biophilic design, and gaps in aligning data-driven design with user satisfaction.
    This includes quantitative methods and computational studies on complexity.
    \item Gaps in Current Research: Identifies the existing gaps in the research, highlighting the need for comprehensive frameworks that integrate theoretical principles with practical applications, and discusses the challenges and opportunities of rapidly advancing technologies.
\end{itemize}

These insights will be instrumental for validating CICA findings by comparing empirical data with theoretical insights into architectural evolution, confirming observed patterns of simplicity and complexity.




