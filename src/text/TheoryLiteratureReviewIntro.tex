%% Research Background

%% A Theory section should extend, not repeat, the background to the article already dealt with in the Introduction and lay the foundation for further work. In contrast, a Calculation section represents a practical development from a theoretical basis.

%A theoretical framework is a foundational review of existing theories that serves as a roadmap for developing the arguments you will use in your own work.

%=============================================================================
%Outline.
%1. Introduction
%2. The importance of facade in a building.
%3. A brief analysis of architecture styles and the shift from Images and simplicity
%3. A reflexion into digital fabrication techniques specially parametric design.
%4. The principle of data driven design.
%5. An analysis into mixed reality and the advantages of introducing this  into the design review.
%=============================================================================

%///////////////////////////////////////////////////////////////
%% refined outline of theory framework

%6. **Human-Centric Design Philosophy:** Explore the historical evolution of architectural philosophies that prioritize human experience and well-being. Discuss how different architectural styles and movements have addressed the balance between ornamentation, functionality, and human comfort.

%7. **Cultural Significance of Facades:** Delve deeper into how facades reflect cultural values, societal norms, and historical contexts. Explore how different societies and civilizations have expressed their identity through architectural ornamentation and symbolism.

%8. **Environmental Sustainability:** Investigate how the integration of complex facades and digital fabrication aligns with contemporary sustainability principles. Examine how parametric design and data-driven approaches can enhance energy efficiency, reduce material waste, and contribute to sustainable construction practices.

%9. **Ethics and Social Responsibility:** Consider the ethical implications of embracing ornate designs and digital fabrication in a world grappling with issues of resource scarcity and inequality. Discuss how responsible architectural practices can address these challenges and contribute positively to society.

%10. **Collaborative Design Process:** Explore the collaborative aspects of architectural design when utilizing digital fabrication and mixed reality. Discuss how these technologies foster interdisciplinary collaboration among architects, engineers, urban planners, and other stakeholders.

%11. **Challenges and Limitations:** Address the potential challenges and limitations of implementing complex facades and digital fabrication techniques. Discuss technical, economic, and cultural barriers that may arise and the strategies to overcome them.

%12. **Case Studies and Success Stories:** Showcase real-world examples of projects that have successfully integrated complex facades, digital fabrication, and mixed reality. Analyze the impact of these projects on user experience, urban aesthetics, and architectural innovation.

%13. **User-Centered Design:** Delve into the principles of user-centered design and human factors that influence the acceptance of intricate facades. Consider factors like psychological comfort, emotional response, and sensory experience in relation to complex architectural designs.

%14. **Cognitive Aspects of Mixed Reality:** Explore the cognitive psychology behind mixed reality experiences and how they influence user perceptions of architectural Images. Consider how elements like immersion, presence, and interaction affect users' understanding and appreciation of design.

%15. **Future Trends and Speculation:** Speculate on the potential future trajectories of architectural design, digital fabrication, and mixed reality. Discuss how these trends might shape the built environment, and propose ideas for further research and exploration.

%By incorporating these additional topics into your theory framework, you can provide a comprehensive and well-rounded context for your research, offering deeper insights into the historical, cultural, ethical, and technological dimensions of complex facades, digital fabrication, and mixed reality in architecture.
%///////////////////////////////////////////////////////////////

%% Introduction

%!Concise intro

This section outlines existing architectural theories, identifies trends in architectural complexity, and traces the evolution of design practices.
It lays the groundwork for understanding the historical shifts between complexity and simplicity in architecture.
By examining the role of facades in conveying a building's identity and the relationship between architectural evolution and societal changes, we aim to identify strategies that ensure built environments align harmoniously with community needs.

Our analysis delves into architecture's complex narrative, highlighting its critical role at the convergence of innovation, practicality, and heritage.
We aim to clarify how various architectural styles enrich the cultural fabric of urban spaces, referencing Gage\cite{Gage2015} to emphasize the necessity for architecture to move beyond basic functionality and resonate with current societal trends. 
This is particularly relevant in the context of the Modernist movement's aftermath, which, despite its breakthroughs, has been critiqued for its emphasis on uniformity and functionality at the expense of cultural heritage, leading to a lack of diversity in urban aesthetics and a dilution of cities' unique visual identities\cite{Elsheshtawy1997}.

This section focuses on two key themes:

\begin{itemize}
    \item The Architectural Journey: Oscillations Between Simplicity and Complexity: Examining how historical and contemporary architectural styles have transitioned from minimalistic approaches to more complex and ornamented styles.
    \item The Cultural Significance of Facades and Ornamentation: Exploring how facades reflect and shape the cultural and historical contexts in which they exist.
\end{itemize}

Through this analytical lens, we seek to unravel the complex interplay between architectural form, functionality, and societal context, aiming to guide future architectural practices with insights that foster inclusive, engaging, and culturally rich environments.

Furthermore, insights from this section will be instrumental in validating the results obtained from the CICA system applied to images of buildings across different epochs.
Comparing CICA`s empirical findings with theoretical insights into architectural evolution allows us to confirm the observed cyclical patterns of simplicity and complexity.
This validation will not only bolster the CICA system's reliability but also ensures our exploration of architectural complexity is grounded in historical and technological contexts.


%! previous intro

%Architecture, as a reflection of society, has continually evolved to accommodate the needs of the communities it shelters.
%Buildings, in essence, serve as tools with a purpose—guardians of societies, nurturers of generations, and manifestations of future aspirations.
%
%Critique is inherent to architecture, and it falls upon successive generations to discern flaws within the inherited built environment.
%Amidst these legacies, the legacy of the modernist movement, emerging in the mid-20th century, emerges as particularly relevant.
%Anchored in principles of simplicity and the maxim ``form follows function'', this movement gained global prominence, addressing urbanization challenges triggered by rural migration to cities.
%
%Yet, this era's legacy brings forth a profound debate.
%While remarkable creations emerged, it's important not to label nearly a century of architectural style as universally negative.
%However, society acknowledges the unintended consequences.
%
%The fervor for uniformity and functionalism, epitomized by the modernism style of the 20th century, led to a disconnect from cultural roots.
%As cities embraced this discourse, they lost distinct identities, homogenizing urban landscapes and erasing their unique memories.
%
%As Venturi\cite{Venturi1972} explains learning from  the existing landscape  is  a  way of being revolutionary for  an  architect, not the obvious way, which is to tear down the existing city and start again, but another, more tolerant way;
%that is, to question how we look at things.
%
%This realization exemplifies the complex narrative of architecture—a dynamic interplay between innovation, utility, and cultural heritage.
%It reminds us that while architectural styles may have their merits, the preservation of cultural essence and identity is vital for thriving urban spaces that resonate with inhabitants and stand the test of time.
%
% Because as Gage~\cite{Gage2015} eloquently puts it ``If architecture is to exist in the 21st century, when attention is focused on the fast-paced worlds of technology, fashion, and entertainment, it must not recede into the background as mere functional equipment''.
%
%As we delve into the theory framework that underpins this research, we embark on a journey through historical shifts between architectural complexity and simplicity.
%We explore the significance of facades in shaping the identity of structures.
%We reflect on the integration of digital fabrication techniques, particularly parametric design, and the fundamental principles of data-driven design.
%Our exploration extends to the realm of mixed reality and the promising advantages it introduces to the architectural design review process.
%
%Each component within this theory framework contributes to our understanding of the intricate interplay between architectural evolution and societal dynamics.
%As we navigate through complexities and contemplate the subtleties of design paradigms, we seek to uncover insights that illuminate the path to a more harmonious relationship between built environments and the people who inhabit them.






