%\subsubsection{VR integration and simulation tools}
%\label{subsubsec:VR_integration}
%%Define the sections of the VR interface and how information is displayed
%\input{Text/VR_integration}


The goal of this component is to integrate the virtual environment from the `3D Modeling and Environment Setup' with data from the CICA complexity analysis (Figure~\ref{fig:MethodologyFlowchartComplexity}, element 3).
This module features an immersive `VR simulation' and a `data visualization interface' that allows users to explore and interact with the building's interior and exterior, visualize its context, and manipulate facade variations (Figure~\ref{fig:VRinterfaceComplexity}).

The `VR simulation,' was developed using Unity (v.2022.2.21f1) and accessible through a Head-Mounted Display (HMD), Oculus Quest 2.
This software was chosen for its robust VR support, pre-built templates, and seamless integration with Python and C\#, enhancing simulation interactivity and data handling.

%!Vr interface
The VR data visualization interface provides real-time feedback on facade variations, facilitating data collection on user response to varying levels of facade complexity.
Structured into five key sections—Viewpoint Navigation, Facade Variation Slider, Facade Render Preview, CICA Scores Comparative Analysis Charts, and Utility Functions (labeled 1 to 5 in Figure~\ref{fig:VRinterfaceComplexity})—it enhances usability and interpretability, thereby optimizing the facade selection process.





