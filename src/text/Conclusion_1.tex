%!%\section{Conclusions}
%%\label{sec:Conclusion}
%%%!%\section{Conclusions}
%%\label{sec:Conclusion}
%%%!%\section{Conclusions}
%%\label{sec:Conclusion}
%%%!%\section{Conclusions}
%%\label{sec:Conclusion}
%%\input{Text/Conclusions}

%% The main conclusions of the study may be presented in a short Conclusions section, which may stand alone or form a subsection of a Discussion or Results and Discussion section.
%% Answer the Research question

%revision

This paper explores architectural design at the intersection of digital fabrication, virtual reality assessment, and computer vision to deepen our understanding of intricate facade design.
Our primary goal is to gauge user tolerance and acceptance of complex facades, offering insights into future construction practices.
A literature review confirms a resurgence of complexity in contemporary architecture.
We introduce the `Computational Image Complexity Analysis' (CICA) system, using computer vision to quantitatively analyzing building complexity across epochs, revealing an upward complexity trendline since the late 20th century.

In a virtual reality experiment, we quantify user responses to complex facades, shedding light on their tolerance and acceptance levels.
On average, participants favor complexity, with an average score of 3.82 (out of 10) and a \(40\%\) probability of selecting a score close to this value according to CICA.
Post-survey qualitative scores averaged 4.9 on a 7-point Likert scale, indicating favorable attitudes toward complex facade variations.
These results align with the historical analysis, confirming contemporary architecture's embrace of complexity.

%! conclusions from survey results
 %Q9
 Architects and designers can consider this result as a reflection of the delicate balance required in incorporating patterns and textures into facades.
 While complexity can be visually engaging, it should not overshadow other design elements or create visual clutter.
 A moderate perception of complexity suggests that the design struck a reasonable equilibrium in this regard.

%Q10

Furthermore, a moderate rating implies that the design may have struck a chord with a broad range of participants, appealing to those who appreciate both intricate and simpler ornamentation styles.
 This adaptability in the perception of detail suggests that the design achieved a versatile and inclusive approach to ornamentation.

%Q11

From an architectural standpoint, this finding underscores the importance of materials in creating visually appealing and complex facades.
 The materials used in construction play a crucial role in defining the facade's character and can significantly impact its aesthetic appeal.

%Q12
The participants' feedback on the aesthetic intricacy of the facade compositions aligns with the idea that a balanced and moderate level of complexity in composition can enhance the overall visual appeal of architectural facades.

%Q13
The results indicate that the majority of participants recognized the pivotal role played by the arrangement of architectural elements in enhancing the facade's complexity.

%Q14
While color does contribute to complexity, its role might be less pronounced compared to factors like shape arrangement.Color has a role in creating visually engaging facades, however other design elements may have a more substantial influence on complexity.

%!draft ideas
 Architects may be responding to a growing emphasis on user experience and well-being.
 Complex facades could be designed to engage occupants and create aesthetically pleasing environments.

%!Ideas for interpretations of upwards trajectory of trendline in historical analysis chart graph

%Resurgence of Ornamentation: This trend may reflect a resurgence of interest in architectural ornamentation and intricate facade design. After the stark minimalism of the mid-20th century, architects may be returning to more decorative and visually complex elements in building design.
%
%Advancements in Technology: The last half-century has witnessed significant technological advancements, particularly in computer-aided design and construction techniques. These technological leaps may have empowered architects to explore more complex and daring designs that were previously challenging to execute.
%
%Sustainability and Integration: Contemporary architectural practices emphasize sustainability and the integration of buildings with their surroundings. Complex facade designs may facilitate innovative approaches to sustainability, such as incorporating natural ventilation and shading strategies.
%
%Expression of Identity: In an era of globalization, architects may be using complex facade designs as a means of expressing regional or cultural identities. This could explain the resurgence of ornamentation and complex patterns.
%
%Human-Centric Design: Architects may be responding to a growing emphasis on user experience and well-being. Complex facades could be designed to engage occupants and create aesthetically pleasing environments.
%
%Economic Prosperity: Economic factors can also influence architectural trends. Periods of economic prosperity often lead to more ambitious and complex building designs, as resources are more readily available for experimentation.
%
%Post-Modernism and Pluralism: The rejection of the rigid principles of Modernism in favor of post-modernism and architectural pluralism has contributed to a more diverse range of design possibilities, including complex facades.
%
%Interdisciplinary Collaboration: Architects are increasingly collaborating with other disciplines, such as artists and engineers. This interdisciplinary approach may encourage more complex and innovative facade designs.
%
%Environmental Considerations: The focus on environmental sustainability may drive architects to explore facade designs that optimize energy efficiency, daylighting, and thermal comfort, often requiring intricate solutions.
%
%Global Connectivity: Architects today have access to a wealth of global architectural influences and references through the internet. This exposure to diverse architectural traditions may encourage experimentation with complex facades.

%% The main conclusions of the study may be presented in a short Conclusions section, which may stand alone or form a subsection of a Discussion or Results and Discussion section.
%% Answer the Research question

%revision

This paper explores architectural design at the intersection of digital fabrication, virtual reality assessment, and computer vision to deepen our understanding of intricate facade design.
Our primary goal is to gauge user tolerance and acceptance of complex facades, offering insights into future construction practices.
A literature review confirms a resurgence of complexity in contemporary architecture.
We introduce the `Computational Image Complexity Analysis' (CICA) system, using computer vision to quantitatively analyzing building complexity across epochs, revealing an upward complexity trendline since the late 20th century.

In a virtual reality experiment, we quantify user responses to complex facades, shedding light on their tolerance and acceptance levels.
On average, participants favor complexity, with an average score of 3.82 (out of 10) and a \(40\%\) probability of selecting a score close to this value according to CICA.
Post-survey qualitative scores averaged 4.9 on a 7-point Likert scale, indicating favorable attitudes toward complex facade variations.
These results align with the historical analysis, confirming contemporary architecture's embrace of complexity.

%! conclusions from survey results
 %Q9
 Architects and designers can consider this result as a reflection of the delicate balance required in incorporating patterns and textures into facades.
 While complexity can be visually engaging, it should not overshadow other design elements or create visual clutter.
 A moderate perception of complexity suggests that the design struck a reasonable equilibrium in this regard.

%Q10

Furthermore, a moderate rating implies that the design may have struck a chord with a broad range of participants, appealing to those who appreciate both intricate and simpler ornamentation styles.
 This adaptability in the perception of detail suggests that the design achieved a versatile and inclusive approach to ornamentation.

%Q11

From an architectural standpoint, this finding underscores the importance of materials in creating visually appealing and complex facades.
 The materials used in construction play a crucial role in defining the facade's character and can significantly impact its aesthetic appeal.

%Q12
The participants' feedback on the aesthetic intricacy of the facade compositions aligns with the idea that a balanced and moderate level of complexity in composition can enhance the overall visual appeal of architectural facades.

%Q13
The results indicate that the majority of participants recognized the pivotal role played by the arrangement of architectural elements in enhancing the facade's complexity.

%Q14
While color does contribute to complexity, its role might be less pronounced compared to factors like shape arrangement.Color has a role in creating visually engaging facades, however other design elements may have a more substantial influence on complexity.

%!draft ideas
 Architects may be responding to a growing emphasis on user experience and well-being.
 Complex facades could be designed to engage occupants and create aesthetically pleasing environments.

%!Ideas for interpretations of upwards trajectory of trendline in historical analysis chart graph

%Resurgence of Ornamentation: This trend may reflect a resurgence of interest in architectural ornamentation and intricate facade design. After the stark minimalism of the mid-20th century, architects may be returning to more decorative and visually complex elements in building design.
%
%Advancements in Technology: The last half-century has witnessed significant technological advancements, particularly in computer-aided design and construction techniques. These technological leaps may have empowered architects to explore more complex and daring designs that were previously challenging to execute.
%
%Sustainability and Integration: Contemporary architectural practices emphasize sustainability and the integration of buildings with their surroundings. Complex facade designs may facilitate innovative approaches to sustainability, such as incorporating natural ventilation and shading strategies.
%
%Expression of Identity: In an era of globalization, architects may be using complex facade designs as a means of expressing regional or cultural identities. This could explain the resurgence of ornamentation and complex patterns.
%
%Human-Centric Design: Architects may be responding to a growing emphasis on user experience and well-being. Complex facades could be designed to engage occupants and create aesthetically pleasing environments.
%
%Economic Prosperity: Economic factors can also influence architectural trends. Periods of economic prosperity often lead to more ambitious and complex building designs, as resources are more readily available for experimentation.
%
%Post-Modernism and Pluralism: The rejection of the rigid principles of Modernism in favor of post-modernism and architectural pluralism has contributed to a more diverse range of design possibilities, including complex facades.
%
%Interdisciplinary Collaboration: Architects are increasingly collaborating with other disciplines, such as artists and engineers. This interdisciplinary approach may encourage more complex and innovative facade designs.
%
%Environmental Considerations: The focus on environmental sustainability may drive architects to explore facade designs that optimize energy efficiency, daylighting, and thermal comfort, often requiring intricate solutions.
%
%Global Connectivity: Architects today have access to a wealth of global architectural influences and references through the internet. This exposure to diverse architectural traditions may encourage experimentation with complex facades.

%% The main conclusions of the study may be presented in a short Conclusions section, which may stand alone or form a subsection of a Discussion or Results and Discussion section.
%% Answer the Research question

%revision

This paper explores architectural design at the intersection of digital fabrication, virtual reality assessment, and computer vision to deepen our understanding of intricate facade design.
Our primary goal is to gauge user tolerance and acceptance of complex facades, offering insights into future construction practices.
A literature review confirms a resurgence of complexity in contemporary architecture.
We introduce the `Computational Image Complexity Analysis' (CICA) system, using computer vision to quantitatively analyzing building complexity across epochs, revealing an upward complexity trendline since the late 20th century.

In a virtual reality experiment, we quantify user responses to complex facades, shedding light on their tolerance and acceptance levels.
On average, participants favor complexity, with an average score of 3.82 (out of 10) and a \(40\%\) probability of selecting a score close to this value according to CICA.
Post-survey qualitative scores averaged 4.9 on a 7-point Likert scale, indicating favorable attitudes toward complex facade variations.
These results align with the historical analysis, confirming contemporary architecture's embrace of complexity.

%! conclusions from survey results
 %Q9
 Architects and designers can consider this result as a reflection of the delicate balance required in incorporating patterns and textures into facades.
 While complexity can be visually engaging, it should not overshadow other design elements or create visual clutter.
 A moderate perception of complexity suggests that the design struck a reasonable equilibrium in this regard.

%Q10

Furthermore, a moderate rating implies that the design may have struck a chord with a broad range of participants, appealing to those who appreciate both intricate and simpler ornamentation styles.
 This adaptability in the perception of detail suggests that the design achieved a versatile and inclusive approach to ornamentation.

%Q11

From an architectural standpoint, this finding underscores the importance of materials in creating visually appealing and complex facades.
 The materials used in construction play a crucial role in defining the facade's character and can significantly impact its aesthetic appeal.

%Q12
The participants' feedback on the aesthetic intricacy of the facade compositions aligns with the idea that a balanced and moderate level of complexity in composition can enhance the overall visual appeal of architectural facades.

%Q13
The results indicate that the majority of participants recognized the pivotal role played by the arrangement of architectural elements in enhancing the facade's complexity.

%Q14
While color does contribute to complexity, its role might be less pronounced compared to factors like shape arrangement.Color has a role in creating visually engaging facades, however other design elements may have a more substantial influence on complexity.

%!draft ideas
 Architects may be responding to a growing emphasis on user experience and well-being.
 Complex facades could be designed to engage occupants and create aesthetically pleasing environments.

%!Ideas for interpretations of upwards trajectory of trendline in historical analysis chart graph

%Resurgence of Ornamentation: This trend may reflect a resurgence of interest in architectural ornamentation and intricate facade design. After the stark minimalism of the mid-20th century, architects may be returning to more decorative and visually complex elements in building design.
%
%Advancements in Technology: The last half-century has witnessed significant technological advancements, particularly in computer-aided design and construction techniques. These technological leaps may have empowered architects to explore more complex and daring designs that were previously challenging to execute.
%
%Sustainability and Integration: Contemporary architectural practices emphasize sustainability and the integration of buildings with their surroundings. Complex facade designs may facilitate innovative approaches to sustainability, such as incorporating natural ventilation and shading strategies.
%
%Expression of Identity: In an era of globalization, architects may be using complex facade designs as a means of expressing regional or cultural identities. This could explain the resurgence of ornamentation and complex patterns.
%
%Human-Centric Design: Architects may be responding to a growing emphasis on user experience and well-being. Complex facades could be designed to engage occupants and create aesthetically pleasing environments.
%
%Economic Prosperity: Economic factors can also influence architectural trends. Periods of economic prosperity often lead to more ambitious and complex building designs, as resources are more readily available for experimentation.
%
%Post-Modernism and Pluralism: The rejection of the rigid principles of Modernism in favor of post-modernism and architectural pluralism has contributed to a more diverse range of design possibilities, including complex facades.
%
%Interdisciplinary Collaboration: Architects are increasingly collaborating with other disciplines, such as artists and engineers. This interdisciplinary approach may encourage more complex and innovative facade designs.
%
%Environmental Considerations: The focus on environmental sustainability may drive architects to explore facade designs that optimize energy efficiency, daylighting, and thermal comfort, often requiring intricate solutions.
%
%Global Connectivity: Architects today have access to a wealth of global architectural influences and references through the internet. This exposure to diverse architectural traditions may encourage experimentation with complex facades.

%% The main conclusions of the study may be presented in a short Conclusions section, which may stand alone or form a subsection of a Discussion or Results and Discussion section.
%% Answer the Research question

%revision

This research investigates architectural design at the intersection of digital fabrication, virtual reality (VR) assessment, and computer vision, aiming to deepen our understanding of complexity in facade design.
Our primary goal is to gauge user tolerance and acceptance of complex facades, offering insights into future construction practices.

A literature review confirmed that contemporary architecture is witnessing a trend towards increasing complexity in facade designs, moving away from the minimalist approach of the modernist movement, a trend also evidenced by the quantitative analysis across architectural history, provided by the CICA system, which revealed an upward complexity trendline since the late 20th century (see Figure~\ref{fig:CICAscatterGraphRender}).
The historical analysis using the CICA system underscored the cultural and historical significance of facades, indicating that architectural complexity is not merely a matter of quantitative metrics but also involves cultural resonance and historical context.

Participants in the virtual reality experiment showed a preference for facades with moderate complexity, suggesting that future architectural trends may favor designs that balance intricacy with simplicity.
On average, participants favor a moderate level of complexity, with an average CICA complexity score of 3.82 (out of 10) and a 40\% probability of selecting a score close to this value according to CICA\@.

Discrepancies between participant perceptions and the CICA system's complexity rankings were particularly evident at higher complexity levels.
This highlights the subjective nature of complexity perception and the importance of integrating human feedback into architectural assessments.

The qualitative data suggests a shift towards customizable and user-responsive architectural solutions, with participants favoring form over materials and expressing a preference for facades that consider views and privacy.
This feedback suggests a strategic, view-dependent approach to facade complexity is crucial for user satisfaction.

%!% limitations: What can’t your results tell us?

\textbf{Limitations}

The findings of this research provide valuable insights  into architectural complexity, yet certain limitations warrant cautious interpretation of the results:
%1 Sample Size and Demographic Representation:
This research involved a relatively small sample of 10 participants, primarily composed by university students and faculty members, which may limit the generalizability of the findings, as the results might not fully represent the broader population's preferences and perceptions of complexity.

%2 Virtual Reality Environment:
The choice of VR assesment as the core strategy for this research offered a controlled and immersive environment for evaluating user preferences, however it may not entirely capture the experience of interacting with real-world facades.
The VR setting could potentially affect participants' perceptions of complexity and comfort, leading to results that may differ from real-world reactions.

%3 CICA System Metrics and Historical Analysis Dataset:
In regard to the CICA system, developed for this research, it successfully provided an evaluation of architectural complexity using specific metrics, however,it may not capture all elements influencing perceived complexity.
This underscores the need for further exploration of the subjective nature of complexity perception, which can be shaped by individual aesthetic preferences, prior experiences, and cultural factors, aspects that computer vision algorithms might struggle to fully capture.

The effectiveness of computer vision algorithms depends on the quality and representativeness of the dataset used for training and analysis.
A limited dataset, as in the case of the 177 buildings used in this research, might restrict the comprehensiveness of the complexity assessment and while this provides a useful overview, expanding the dataset could yield a more detailed understanding of trends in architectural complexity over time.

%4 Focus on Facade Design
This research concentrated on facade design, which is just one aspect of architectural complexity.
Computer vision models developed for specific architectural features might not generalize well to other styles or unique design elements, potentially limiting their applicability across diverse architectural contexts.

%!% limitations: What can’t your results tell us?

%\textbf{Limitations}
%
%The findings of this research provide valuable insights, however, it is important to interpret the results with caution due to certain limitations.
%
%However, it's essential to recognize that every research endeavor carries its unique set of limitations, and our study is no exception.
%In the spirit of transparency, we candidly acknowledge these limitations, which we will address.
%Nevertheless, the following subsections, which meticulously analyze our results, were arranged in accordance with the core themes and research questions that guided our inquiry.

        %The findings of this research provide valuable insights into the potential of VR immersion in data-driven design for Site Layout Planning, as well as potential implications for other areas of the building design process. However, it is important to interpret the results with caution due to certain limitations. The generalizability of the findings is limited by the small sample size and the fact that the participants were limited to individuals affiliated with the university.

        %Furthermore, the experiment design aimed to simulate the current methodology used in addressing SLP through a "screen-based interaction" stage, which involved CAD plans, perspectives, and expert recommendations typically provided to design teams.

        %However, the introduction of data visualization techniques and charts was exclusive to the "VR stage" (see Figure \ref{fig:VRinterface}). As a result, it is challenging to determine the exact contribution of VR-based interaction versus more efficient data visualization techniques to the observed accuracy improvement. It can be speculated that if the same graphs and interface used in the VR stage were introduced to the screen-based stage, the observed accuracy improvement may have been different.

%%Highlight potential avenues for further research and development.
%\subsection{Future work}
    %\label{subsec:Future_work}
    %% highlight potential avenues for further research and development.

%% recommendations: Avenues for further studies or analyses

        %% Based on the discussion of your results, you can make recommendations for practical implementation or further research. Sometimes, the recommendations are saved for the conclusion. Suggestions for further research can lead directly from the limitations. Don’t just state that more studies should be done—give concrete ideas for how future work can build on areas that your own research was unable to address.


\textbf{Future Works}

This research's findings and limitations provide opportunities for further exploration in architectural complexity, enhancing our understanding of its impact on user experiences and preferences.

%1Sample Size and Demographic Representation:

Future studies should aim to involve a larger and more diverse group of participants, encompassing various geographic locations, cultural backgrounds, and professions to broaden the generalizability of the findings.
Additionally, conducting long-term studies could shed light on the evolution of preferences for architectural complexity over time and across different contexts.
Such research could offer valuable insights into the sustainability of design trends over time, and user adaptability, as well as provide a more comprehensive understanding from a wider range of viewpoints.

%2 Virtual Reality Environment:

Future research could compare VR-based assessments with evaluations of physical facades to better understand the correlation between virtual experiences and real-world perceptions.
This comparison could help refine VR methodologies for architectural research.
Additionally, leveraging emerging technologies in Extended Reality (ER), such as Mixed Reality (MR) and Augmented Reality (AR), could further bridge the gap between virtual simulations and reality, enhancing the assessment and prediction of user preferences in complexity in architectural design.


%3 CICA System Metrics and Historical Analysis Dataset:

The CICA system presents opportunities for further development and enhancement.
Future works could improve the accuracy of the CICA system by incorporating additional metrics to encompass a wider array of complexity factors, such as color, texture, and contextual integration.
Future iterations should consider both the quantitative aspects of facade complexity and the cultural resonance and historical context to provide a comprehensive evaluation of architectural evolution.
A strategy to achieve this developing methodologies that integrate user feedback more directly into the design process.
This could lead to more personalized and culturally sensitive architectural solutions, providing insights into how cultural influences shape preferences for architectural complexity and informing culturally sensitive design practices.

%4 Focus on Facade Design

Future research should broaden its scope to encompass additional elements, such as interior design and spatial organization, to achieve a more holistic understanding of architectural complexity.
By extending the focus from facade design to include interior spaces and overall building organization, a more comprehensive perspective on architectural complexity can be attained.
Additionally, future studies should examine the relationship between architectural complexity and sustainability, exploring how complex designs can either contribute to or detract from sustainable building practices.

Future research should extend its scope beyond facade design to include interior design and spatial organization for a holistic understanding of architectural complexity.
Additionally, future studies should examine the relationship between architectural complexity and sustainability, exploring how complex designs can either contribute to or detract from sustainable building practices.


In conclusion, this study underscores a shift in contemporary architecture towards embracing complexity in facade design, moving beyond the constraints of the modernist movement.
These insights could inform the development of nuanced and user-centric approaches in architectural design, catering to the evolving demands of modern society.
