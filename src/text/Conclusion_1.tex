%!%\section{Conclusions}
%%\label{sec:Conclusion}
%%%!%\section{Conclusions}
%%\label{sec:Conclusion}
%%%!%\section{Conclusions}
%%\label{sec:Conclusion}
%%%!%\section{Conclusions}
%%\label{sec:Conclusion}
%%\input{Text/Conclusions}

%% The main conclusions of the study may be presented in a short Conclusions section, which may stand alone or form a subsection of a Discussion or Results and Discussion section.
%% Answer the Research question

%revision

This research investigates architectural design at the intersection of digital fabrication, virtual reality (VR) assessment, and computer vision, aiming to deepen our understanding of complexity in facade design.
Our primary goal is to gauge user tolerance and acceptance of complex facades, offering insights into future construction practices.

A literature review confirmed that contemporary architecture is witnessing a trend towards increasing complexity in facade designs, moving away from the minimalist approach of the modernist movement, a trend also evidenced by the quantitative analysis across architectural history, provided by the `Computational Image Complexity Analysis' (CICA) system, which revealed an upward complexity trendline since the late 20th century (see Figure~\ref{fig:CICAscatterGraphRender}).
The historical analysis using the CICA system underscored the cultural and historical significance of facades, indicating that architectural complexity is not merely a matter of quantitative metrics but also involves cultural resonance and historical context.

Participants in the virtual reality experiment showed a preference for facades with moderate complexity, suggesting that future architectural trends may favor designs that balance intricacy with simplicity.
On average, participants favor a moderate level of complexity, with an average CICA complexity score of 3.82 (out of 10) and a 40\% probability of selecting a score close to this value according to CICA\@.

Discrepancies between participant perceptions and the CICA system's complexity rankings were particularly evident at higher complexity levels.
This highlights the subjective nature of complexity perception and the importance of integrating human feedback into architectural assessments.

The qualitative data suggests a shift towards customizable and user-responsive architectural solutions, with participants favoring form over materials and expressing a preference for facades that consider views and privacy.
This feedback suggests a strategic, view-dependent approach to facade complexity is crucial for user satisfaction.

In conclusion, this study underscores a shift in contemporary architecture towards embracing complexity in facade design, moving beyond the minimalist constraints of the modernist movement.
These insights could inform the development of nuanced and user-centric approaches in architectural design, catering to the evolving demands of modern society.


%% The main conclusions of the study may be presented in a short Conclusions section, which may stand alone or form a subsection of a Discussion or Results and Discussion section.
%% Answer the Research question

%revision

This research investigates architectural design at the intersection of digital fabrication, virtual reality (VR) assessment, and computer vision, aiming to deepen our understanding of complexity in facade design.
Our primary goal is to gauge user tolerance and acceptance of complex facades, offering insights into future construction practices.

A literature review confirmed that contemporary architecture is witnessing a trend towards increasing complexity in facade designs, moving away from the minimalist approach of the modernist movement, a trend also evidenced by the quantitative analysis across architectural history, provided by the `Computational Image Complexity Analysis' (CICA) system, which revealed an upward complexity trendline since the late 20th century (see Figure~\ref{fig:CICAscatterGraphRender}).
The historical analysis using the CICA system underscored the cultural and historical significance of facades, indicating that architectural complexity is not merely a matter of quantitative metrics but also involves cultural resonance and historical context.

Participants in the virtual reality experiment showed a preference for facades with moderate complexity, suggesting that future architectural trends may favor designs that balance intricacy with simplicity.
On average, participants favor a moderate level of complexity, with an average CICA complexity score of 3.82 (out of 10) and a 40\% probability of selecting a score close to this value according to CICA\@.

Discrepancies between participant perceptions and the CICA system's complexity rankings were particularly evident at higher complexity levels.
This highlights the subjective nature of complexity perception and the importance of integrating human feedback into architectural assessments.

The qualitative data suggests a shift towards customizable and user-responsive architectural solutions, with participants favoring form over materials and expressing a preference for facades that consider views and privacy.
This feedback suggests a strategic, view-dependent approach to facade complexity is crucial for user satisfaction.

In conclusion, this study underscores a shift in contemporary architecture towards embracing complexity in facade design, moving beyond the minimalist constraints of the modernist movement.
These insights could inform the development of nuanced and user-centric approaches in architectural design, catering to the evolving demands of modern society.


%% The main conclusions of the study may be presented in a short Conclusions section, which may stand alone or form a subsection of a Discussion or Results and Discussion section.
%% Answer the Research question

%revision

This research investigates architectural design at the intersection of digital fabrication, virtual reality (VR) assessment, and computer vision, aiming to deepen our understanding of complexity in facade design.
Our primary goal is to gauge user tolerance and acceptance of complex facades, offering insights into future construction practices.

A literature review confirmed that contemporary architecture is witnessing a trend towards increasing complexity in facade designs, moving away from the minimalist approach of the modernist movement, a trend also evidenced by the quantitative analysis across architectural history, provided by the `Computational Image Complexity Analysis' (CICA) system, which revealed an upward complexity trendline since the late 20th century (see Figure~\ref{fig:CICAscatterGraphRender}).
The historical analysis using the CICA system underscored the cultural and historical significance of facades, indicating that architectural complexity is not merely a matter of quantitative metrics but also involves cultural resonance and historical context.

Participants in the virtual reality experiment showed a preference for facades with moderate complexity, suggesting that future architectural trends may favor designs that balance intricacy with simplicity.
On average, participants favor a moderate level of complexity, with an average CICA complexity score of 3.82 (out of 10) and a 40\% probability of selecting a score close to this value according to CICA\@.

Discrepancies between participant perceptions and the CICA system's complexity rankings were particularly evident at higher complexity levels.
This highlights the subjective nature of complexity perception and the importance of integrating human feedback into architectural assessments.

The qualitative data suggests a shift towards customizable and user-responsive architectural solutions, with participants favoring form over materials and expressing a preference for facades that consider views and privacy.
This feedback suggests a strategic, view-dependent approach to facade complexity is crucial for user satisfaction.

In conclusion, this study underscores a shift in contemporary architecture towards embracing complexity in facade design, moving beyond the minimalist constraints of the modernist movement.
These insights could inform the development of nuanced and user-centric approaches in architectural design, catering to the evolving demands of modern society.


%% The main conclusions of the study may be presented in a short Conclusions section, which may stand alone or form a subsection of a Discussion or Results and Discussion section.
%% Answer the Research question

%Conclusion

This study investigates architectural design at the intersection of digital fabrication, VR assessment, and CV algorithms, aiming to deepen our understanding of complexity in facade design.
Our primary goal is to gauge user tolerance and acceptance of complex facades, offering insights into future construction practices.

A literature review confirmed a trend towards increasing complexity in contemporary architecture, moving away from the uniformity of barren walls and fully glazed facades approach of the modernist movement.
The CICA system quantitatively analyzed this trend, revealing an upward complexity trendline since the late 20th century (see Figure~\ref{fig:CICAscatterGraphRender}).
The historical analysis using the CICA system underscored the cultural and historical significance of facades, indicating that architectural complexity is not merely a matter of quantitative metrics but also involves cultural resonance and historical context.

Participants in the experiment showed a preference for facades with moderate complexity, suggesting that future architectural trends may favor designs that balance intricacy with simplicity.
On average, participants preferred a moderate level of complexity, with a mean CICA complexity score of 4.05 (out of 10) and a 40\% probability of selecting a score close to this value.

Discrepancies between participant perceptions and the CICA system's complexity rankings were more evident at higher complexity levels, highlighting the subjective nature of complexity perception and the need to integrate human feedback into architectural assessments.
Qualitative data suggest a shift towards customizable and user-responsive architectural solutions, with participants favoring form over materials and preferring facades that consider views and privacy.

\textbf{Limitations}

While this research provides valuable insights into architectural complexity, certain limitations warrant cautious interpretation of the results:

%1 Sample Size and Demographic Representation:
The study involved a relatively small sample of 26 participants, primarily university students and faculty members, which may limit the generalizability of the findings.

%2 Virtual Reality Environment:
The use of VR offered a controlled and immersive environment but may not entirely capture the experience of interacting with real-world facades.
VR settings could affect perceptions of complexity and comfort, leading to different results compared to real-world interactions.

%3 CICA System Metrics and Historical Analysis Dataset:
The CICA system, while effective in evaluating architectural complexity using specific metrics, may not capture all elements influencing perceived complexity.
This underscores the need for further exploration of the subjective nature of complexity perception, which can be shaped by individual aesthetic preferences, prior experiences, and cultural factors, aspects that CV algorithms might struggle to fully capture.
The limited dataset of 177 historical buildings may restrict the comprehensiveness of the complexity assessment.
Expanding the dataset could yield a more detailed understanding of trends in architectural complexity over time.

%4 Focus on Facade Design
This study concentrated on facade design, which is just one aspect of architectural complexity.
CV models developed for specific architectural features might not generalize well to other styles or unique design elements, potentially limiting their applicability across diverse architectural contexts.

%%Highlight potential avenues for further research and development.

\textbf{Future Works}

The findings and limitations of this study provide opportunities for further exploration in architectural complexity:

%1Sample Size and Demographic Representation:
Future studies should involve a larger and more diverse group of participants to broaden the generalizability of the findings.
Conducting long-term studies could also shed light on the evolution of preferences for architectural complexity over time.

%2 Virtual Reality Environment:
Future research could compare VR-based assessments with evaluations of physical facades to better understand the correlation between virtual experiences and real-world perceptions.
Leveraging emerging technologies in Extended Reality (ER), such as Mixed Reality (MR) and Augmented Reality (AR), could further bridge the gap between virtual simulations and reality enhancing the assessment and prediction of user preferences in complexity in architectural design.

%3 CICA System Metrics and Historical Analysis Dataset:
Future works could improve the accuracy of the CICA system by incorporating additional metrics, such as color, texture, and contextual integration.
Developing methodologies that integrate user feedback more directly into the design process could lead to more personalized and culturally sensitive architectural solutions.
Future iterations should consider both the quantitative aspects of facade complexity and the cultural resonance and historical context to provide a comprehensive evaluation of architectural evolution.

%4 Focus on Facade Desig
Future research should encompass additional elements, such as interior design and spatial organization, to achieve a more holistic understanding of architectural complexity.
Examining the relationship between architectural complexity and sustainability could also provide insights into how complex designs can contribute to or detract from sustainable building practices.

%%Closing statement
In conclusion, this study underscores a shift in contemporary architecture towards embracing complexity in facade design, moving beyond the constraints of the modernist movement.
These insights could inform the development of nuanced and user-centric approaches in architectural design, catering to the evolving demands of modern society.

