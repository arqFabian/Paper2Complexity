%!%\section{Conclusions}
%%\label{sec:Conclusion}
%%%!%\section{Conclusions}
%%\label{sec:Conclusion}
%%%!%\section{Conclusions}
%%\label{sec:Conclusion}
%%%!%\section{Conclusions}
%%\label{sec:Conclusion}
%%\input{Text/Conclusions}

%% The main conclusions of the study may be presented in a short Conclusions section, which may stand alone or form a subsection of a Discussion or Results and Discussion section.
%% Answer the Research question

%revision

This research investigates architectural design at the intersection of digital fabrication, virtual reality (VR) assessment, and computer vision, aiming to deepen our understanding of complexity in facade design.
Our primary goal is to gauge user tolerance and acceptance of complex facades, offering insights into future construction practices.

A literature review confirmed that contemporary architecture is witnessing a trend towards increasing complexity in facade designs, moving away from the minimalist approach of the modernist movement, a trend also evidenced by the quantitative analysis across architectural history, provided by the `Computational Image Complexity Analysis' (CICA) system, which revealed an upward complexity trendline since the late 20th century (see Figure~\ref{fig:CICAscatterGraphRender}).
The historical analysis using the CICA system underscored the cultural and historical significance of facades, indicating that architectural complexity is not merely a matter of quantitative metrics but also involves cultural resonance and historical context.

Participants in the virtual reality experiment showed a preference for facades with moderate complexity, suggesting that future architectural trends may favor designs that balance intricacy with simplicity.
On average, participants favor a moderate level of complexity, with an average CICA complexity score of 3.82 (out of 10) and a 40\% probability of selecting a score close to this value according to CICA\@.

Discrepancies between participant perceptions and the CICA system's complexity rankings were particularly evident at higher complexity levels.
This highlights the subjective nature of complexity perception and the importance of integrating human feedback into architectural assessments.

The qualitative data suggests a shift towards customizable and user-responsive architectural solutions, with participants favoring form over materials and expressing a preference for facades that consider views and privacy.
This feedback suggests a strategic, view-dependent approach to facade complexity is crucial for user satisfaction.

In conclusion, this study underscores a shift in contemporary architecture towards embracing complexity in facade design, moving beyond the minimalist constraints of the modernist movement.
These insights could inform the development of nuanced and user-centric approaches in architectural design, catering to the evolving demands of modern society.


%% The main conclusions of the study may be presented in a short Conclusions section, which may stand alone or form a subsection of a Discussion or Results and Discussion section.
%% Answer the Research question

%revision

This research investigates architectural design at the intersection of digital fabrication, virtual reality (VR) assessment, and computer vision, aiming to deepen our understanding of complexity in facade design.
Our primary goal is to gauge user tolerance and acceptance of complex facades, offering insights into future construction practices.

A literature review confirmed that contemporary architecture is witnessing a trend towards increasing complexity in facade designs, moving away from the minimalist approach of the modernist movement, a trend also evidenced by the quantitative analysis across architectural history, provided by the `Computational Image Complexity Analysis' (CICA) system, which revealed an upward complexity trendline since the late 20th century (see Figure~\ref{fig:CICAscatterGraphRender}).
The historical analysis using the CICA system underscored the cultural and historical significance of facades, indicating that architectural complexity is not merely a matter of quantitative metrics but also involves cultural resonance and historical context.

Participants in the virtual reality experiment showed a preference for facades with moderate complexity, suggesting that future architectural trends may favor designs that balance intricacy with simplicity.
On average, participants favor a moderate level of complexity, with an average CICA complexity score of 3.82 (out of 10) and a 40\% probability of selecting a score close to this value according to CICA\@.

Discrepancies between participant perceptions and the CICA system's complexity rankings were particularly evident at higher complexity levels.
This highlights the subjective nature of complexity perception and the importance of integrating human feedback into architectural assessments.

The qualitative data suggests a shift towards customizable and user-responsive architectural solutions, with participants favoring form over materials and expressing a preference for facades that consider views and privacy.
This feedback suggests a strategic, view-dependent approach to facade complexity is crucial for user satisfaction.

In conclusion, this study underscores a shift in contemporary architecture towards embracing complexity in facade design, moving beyond the minimalist constraints of the modernist movement.
These insights could inform the development of nuanced and user-centric approaches in architectural design, catering to the evolving demands of modern society.


%% The main conclusions of the study may be presented in a short Conclusions section, which may stand alone or form a subsection of a Discussion or Results and Discussion section.
%% Answer the Research question

%revision

This research investigates architectural design at the intersection of digital fabrication, virtual reality (VR) assessment, and computer vision, aiming to deepen our understanding of complexity in facade design.
Our primary goal is to gauge user tolerance and acceptance of complex facades, offering insights into future construction practices.

A literature review confirmed that contemporary architecture is witnessing a trend towards increasing complexity in facade designs, moving away from the minimalist approach of the modernist movement, a trend also evidenced by the quantitative analysis across architectural history, provided by the `Computational Image Complexity Analysis' (CICA) system, which revealed an upward complexity trendline since the late 20th century (see Figure~\ref{fig:CICAscatterGraphRender}).
The historical analysis using the CICA system underscored the cultural and historical significance of facades, indicating that architectural complexity is not merely a matter of quantitative metrics but also involves cultural resonance and historical context.

Participants in the virtual reality experiment showed a preference for facades with moderate complexity, suggesting that future architectural trends may favor designs that balance intricacy with simplicity.
On average, participants favor a moderate level of complexity, with an average CICA complexity score of 3.82 (out of 10) and a 40\% probability of selecting a score close to this value according to CICA\@.

Discrepancies between participant perceptions and the CICA system's complexity rankings were particularly evident at higher complexity levels.
This highlights the subjective nature of complexity perception and the importance of integrating human feedback into architectural assessments.

The qualitative data suggests a shift towards customizable and user-responsive architectural solutions, with participants favoring form over materials and expressing a preference for facades that consider views and privacy.
This feedback suggests a strategic, view-dependent approach to facade complexity is crucial for user satisfaction.

In conclusion, this study underscores a shift in contemporary architecture towards embracing complexity in facade design, moving beyond the minimalist constraints of the modernist movement.
These insights could inform the development of nuanced and user-centric approaches in architectural design, catering to the evolving demands of modern society.


%% The main conclusions of the study may be presented in a short Conclusions section, which may stand alone or form a subsection of a Discussion or Results and Discussion section.
%% Answer the Research question

%Conclusion

This study investigates architectural design at the intersection of digital fabrication, VR assessment, and CV algorithms, aiming to deepen our understanding of complexity in facade design.
Our primary goal is to verify the practical application of a VR and CV based `Complexity Analysis' system for facade design, offering insights into user acceptance of complex facades.

A literature review theorized a current trend towards increasing complexity in contemporary architecture, moving away from the uniformity of barren walls and fully glazed facades approach of the modernist movement.
The CICA system quantitatively analyzed this same timeframe, proving this theory and revealing the existence of an upward complexity trendline since the late 20th century (see Figure~\ref{fig:CICAscatterGraphRender}).
Furthermore, the historical analysis using the CICA system underscored the cultural and historical significance of facades, indicating that architectural complexity is not merely a matter of quantitative metrics but also involves cultural resonance and historical context.

This study contributes to architectural design theory by bridging qualitative perceptions of complexity with a quantifiable,data-driven approach.
The CICA system offers a novel method for assessing facade complexity using CV algorithms, providing adaptable, quantitative insights across different contexts.
By validating historical trends with empirical data and demonstrating a clear rise in complexity in recent decades, the study advances the theoretical understanding of architectural complexity.
Moreover, the integration of complexity metrics with VR technologies enhances user-centered design, allowing for more interactive assessments of how complexity influences user satisfaction and aesthetics.

Participants in the experiment showed a preference for facades with moderate complexity, suggesting that future architectural trends may favor designs that balance intricacy with simplicity.
On average, participants preferred a moderate level of complexity, with a mean CICA complexity score of 4.05 (out of 10) and a 40\% probability of selecting a score close to this value.

Discrepancies between participant perceptions and the CICA system's complexity rankings, with an average standard deviation of 9\%,  were more evident at higher complexity levels, highlighting the subjective nature of complexity perception and the need to integrate human feedback into architectural assessments.
Qualitative data suggest a shift towards customizable and user-responsive architectural solutions, with participants favoring form over materials and preferring facades that consider views and privacy.

\textbf{Limitations}

While this research provides valuable insights into architectural complexity, certain limitations warrant cautious interpretation of the results:

%1 Sample Size and Demographic Representation:

The study involved a relatively small sample of 26 participants, primarily composed of university students, with 69.2\% aged between 18 and 24.
Most participants lacked extensive professional experience in architecture, particularly in facade design.
This limited demographic representation may affect the generalizability of the findings, as preferences for complexity could vary significantly with broader participant pools, including professionals with diverse levels of expertise and experience.

%2 Virtual Reality Environment:
The use of VR offered a controlled and immersive environment but may not entirely capture the experience of interacting with real-world facades.
VR settings could affect perceptions of complexity and comfort, leading to different results compared to real-world interactions.

%3 CICA System Metrics
The CICA system, while effective in evaluating facade complexity using metrics like edge detection and contour count, may not capture all elements influencing perceived complexity.
These metrics focus on two-dimensional visual data and might not fully address the subjective nature of complexity perception, which is shaped by individual aesthetic preferences, prior experiences, and cultural factors—factors that CV algorithms struggle to encapsulate.
Furthermore, the system lacks tools to analyze three-dimensional articulation and hierarchical design elements, which are crucial to volumetric complexity and the overall form of buildings.
This omission limits the system's ability to fully represent how buildings are perceived in terms of spatial interaction and three-dimensionality.

The survey questions focused mainly on surface-level details, such as patterns and textures, without addressing the building's overall form and three-dimensional geometry.
Incorporating questions and metrics related to volumetric complexity would provide a more complete understanding of how architectural complexity is perceived in real-world settings.

%Image selection criteria and Historical Analysis Dataset:
The evaluation is based on single images of each facade, which limits the ability to fully grasp the overall complexity of a building, as different perspectives might reveal additional elements that contribute to its perceived intricacy.
This choice was made to standardize the analysis, but multiple images from different angles could provide a more comprehensive understanding of architectural complexity.
The limited dataset of 177 historical buildings may restrict the comprehensiveness of the complexity assessment.
Expanding the dataset could yield a more detailed understanding of trends in architectural complexity over time.

%4 Focus on Facade Design
This study concentrated on facade design, which is just one aspect of architectural complexity.
While the insights into patterns, textures, and materials are valuable, the CV models developed for specific facade features might not generalize well to other architectural elements or styles.

%!%Highlight potential avenues for further research and development.

\textbf{Future Works}

The findings and limitations of this study provide opportunities for further exploration in architectural complexity:

%1Sample Size and Demographic Representation:
Future studies should involve a larger and more diverse group of participants to broaden the generalizability of the findings.
Conducting long-term studies could also shed light on the evolution of preferences for architectural complexity over time.

%2 Virtual Reality Environment:
Future research could compare VR-based assessments with evaluations of physical facades to better understand the correlation between virtual experiences and real-world perceptions.
Leveraging emerging technologies in Extended Reality (ER), such as Mixed Reality (MR) and Augmented Reality (AR), could further bridge the gap between virtual simulations and reality enhancing the assessment and prediction of user preferences in complexity in architectural design.

%3 CICA System Metrics and Historical Analysis Dataset:
Future works could improve the accuracy of the CICA system by incorporating additional metrics such as color, texture, and contextual integration, providing a more nuanced understanding of facade complexity.
In particular, incorporating volumetric complexity, such as the three-dimensional articulation, massing, and hierarchical design elements, would significantly enhance the system’s ability to capture how buildings are perceived holistically in real-life contexts.
This could be achieved by developing tools that assess not just the surface-level details but also the overall building form and its spatial interaction with the environment.
Developing methodologies that integrate user feedback more directly into the design process could lead to more personalized and culturally sensitive architectural solutions.
Future iterations should consider both the quantitative aspects of facade complexity and the cultural resonance and historical context to provide a comprehensive evaluation of architectural evolution.

%4 Focus on Facade Design
Furthermore, future research should encompass additional elements, such as interior design and spatial organization, to achieve a more holistic understanding of architectural complexity.
Investigating the relationship between architectural complexity and sustainability could also provide insights into how complex designs impact sustainable building practices, helping architects balance intricate aesthetics with environmental considerations.

%%Closing statement
In conclusion, this study successfully addressed the challenge of quantifying complexity in architectural facade design through the integration of VR and CV technologies.
The practical application of the CICA system extends beyond theoretical analysis, offering architects a tool for informed decision-making in real-world design contexts.
By enabling the quantification of facade complexity, the CICA system provides insights into how visual interest, user satisfaction, and material efficiency can be balanced to create sustainable and adaptable designs.
Architects can apply these findings across various design scenarios, including the development of public spaces, urban renewal projects, and even historic building renovations.

Additionally, by integrating the CICA system with existing energy efficiency tools, the system could be instrumental in designing facades that optimize natural light, thermal performance, and environmental comfort while maintaining aesthetic appeal.
This capability makes the system relevant for a wide range of building projects, helping architects make more DBD decisions that align with both functional and cultural contexts.
This can prove useful for guiding the optimization of building design towards a more user-centric approach in architectural design, catering to the evolving demands of modern society.

