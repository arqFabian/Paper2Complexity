    %\subsubsection{VR integration and data visualization interface}
    %\label{subsubsec:VR_integration}
    %%Define the sections of the VR interface and how information is displayed
    %\input{Text/VR_integration}