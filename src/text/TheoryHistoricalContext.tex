%Historical Context
%\subsection{Evolution of Architectural Styles: Oscillations Between Simplicity and Complexity}
%\label{subsec:TimelineArchitectureStyles}

Architecture stands as a unique art form, transforming the ordinary into the extraordinary while fulfilling functional purposes~\cite{Jiang2023}.
Its evolution is characterized by the integration of technologies and information flows, which shape the complexity and functionality of urban environments, reflecting contemporary societal values and technological advancements~\cite{Leach2015}
From early architectural styles like Romanesque, characterized by robust and simplistic forms, to the Gothic period with its intricate, skyward designs~\cite{Kennedy2013}, architecture has oscillated between simplicity and complexity (see Figure~\ref{fig:Oldtimeline} a,b).

The Renaissance heralded a revival of Greek and Roman ideals with a focus on symmetry, followed by the Baroque's lavish ornamentation in the 16th century~\cite{Marder1990} (see Figure~\ref{fig:Oldtimeline} c,d).
The Neoclassical style, dominant in the 18th and 19th centuries, emphasized symmetry and classical principles while integrating new technologies like reinforced concrete~\cite{Adebusuyi2022} (see Figure~\ref{fig:Oldtimeline} e).
Art Nouveau and Art Deco, emerging in the late 19th and early 20th centuries, celebrated nature and technological advancements, marking a departure from Neoclassical restraint~\cite{Salas2018} (see Figure~\ref{fig:Middletimeline} a, b).

The 20th century saw the rise of Modern Architecture, which advocated for `form follows function' and minimal ornamentation, a significant departure from previous ornate designs~\cite{Leach2016}.
Figures like Adolf Loos and Le Corbusier championed minimalism and functionality, influencing a generation of architects to prioritize structural honesty and simplicity~\cite{Saglam2014}.
However, this movement often led to uniform urban landscapes that lacked the cultural richness and diversity of their predecessors (see Figure~\ref{fig:Middletimeline} c).

In response to Modernism's perceived limitations, the late 1960s saw the emergence of Postmodernism, spearheaded by thinkers like Robert Venturi.
Postmodernism critiqued the minimalist aesthetic and reintroduced complexity, ornament, and form into architectural design, advocating for buildings that engage more deeply with their cultural and historical contexts~\cite{Venturi1972} (see Figure~\ref{fig:Middletimeline} d).

The late 20th and early 21st centuries have seen a resurgence in creativity and expression, with architects utilizing digital technologies to explore new realms of complexity and ornamentation~\cite{Burlando2019} (see Figure~\ref{fig:contemporarytimeline}).
The fusion of digital and physical design processes signals a shift towards the democratization of complex, parametric designs, indicative of a contemporary period that values ornamentation, functionality, and human comfort~\cite{Kim2019}.
This evolving trajectory of architecture suggests a future where design is deeply intertwined with societal values and technological possibilities.

Facades and ornamentation, in this context, become critical in conveying these narratives, bridging the gap between the aesthetic and the symbolic, and establishing the interface between buildings and their environments, influencing comfort and energy efficiency while reflecting the building's identity~\cite{Kamal2020}.
The evolution of facade design and ornamentation mirrors societal transformations, technological progress, and shifts in artistic sensibilities, each impacting how communities relate to their built environment.

In conclusion, the historical context of architectural complexity reveals a rich tapestry of styles and philosophies, from ancient grandiosity to modern minimalism and contemporary innovation.
These shifts reflect the ongoing dialogue between simplicity and complexity, tradition and innovation, and functionality and aesthetics, shaping the built environment in ways that are both imaginative and responsive to societal needs.
