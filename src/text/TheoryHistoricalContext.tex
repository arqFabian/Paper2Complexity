%Historical Context
%\subsection{Evolution of Architectural Styles: Oscillations Between Simplicity and Complexity}
%\label{subsec:TimelineArchitectureStyles}

Architecture stands as a unique art form, transforming the ordinary into the extraordinary while fulfilling functional purposes~\cite{Jiang2023}.
Its evolution is characterized by the integration of technologies and information flows, which shape the complexity and functionality of urban environments, reflecting contemporary societal values and technological advancements~\cite{Leach2015}~\added{(see~\ref{sec:timeline-of-architectural-styles-across-history})}.
\deleted{From early architectural styles like Romanesque, characterized by robust and simplistic forms, to the Gothic period with its intricate, skyward designs~\cite{Kennedy2013}, architecture has oscillated between simplicity and complexity (see Figure~\ref{fig:Earlytimeline} a,b)}
\added{
From early architectural styles like Romanesque, characterized by robust and simplistic forms, to the Gothic period with its intricate, skyward designs, shifts in architectural complexity have often mirrored technological and cultural advancements. Gothic architecture, for example, was made possible by structural innovations like the pointed arch and flying buttress, allowing for taller, more ornate buildings that resonated with the spiritual aspirations of the era~\cite{Kennedy2013}.
}

\deleted{The Renaissance heralded a revival of Greek and Roman ideals with a focus on symmetry, followed by the Baroque's lavish ornamentation in the 16th century~\cite{Marder1990} (see Figure~\ref{fig:Earlytimeline} c,d)}.
\added{
The Renaissance heralded a revival of Greek and Roman ideals, driven by humanism and the desire to return to perceived cultural greatness, with a focus on symmetry and proportion. The Baroque period of the 16th century introduced lavish ornamentation and dynamic designs, reflecting the opulence and power of the ruling classes~\cite{Marder1990}. This move toward complexity was largely influenced by societal shifts towards theatricality and grandeur in both religious and political architecture.
}
\deleted{The Neoclassical style, dominant in the 18th and 19th centuries, emphasized symmetry and classical principles while integrating new technologies like reinforced concrete~\cite{Adebusuyi2022}(see Figure~\ref{fig:Earlytimeline} e)}.
\added{The Neoclassical style, dominant in the 18th and 19th centuries, emphasized symmetry and classical principles, while integrating new technologies like reinforced concrete, which allowed for larger and more functional designs without sacrificing aesthetic form~\cite{Adebusuyi2022}. This balance between the old and the new was a response to Enlightenment ideals of order and rationality.}


\deleted{Art Nouveau and Art Deco, emerging in the late 19th and early 20th centuries, celebrated nature and technological advancements, marking a departure from Neoclassical restraint~\cite{Salas2018}~(see Figure~\ref{fig:Transitionaltimeline} a, b}).
\added{
At the turn of the 20th century, Art Nouveau and Art Deco embraced nature and new materials, with Art Nouveau focusing on organic forms and Art Deco celebrating the technological advancements of the machine age~\cite{Salas2018}. These movements marked a departure from Neoclassical restraint, as societies began to celebrate the possibilities of industrialization and mass production.
}

\deleted{The 20th century saw the rise of Modern Architecture, which advocated for `form follows function' and minimal ornamentation, a significant departure from previous ornate designs~\cite{Leach2016}.
Figures like Adolf Loos and Le Corbusier championed minimalism and functionality, influencing a generation of architects to prioritize structural honesty and simplicity~\cite{Saglam2014}.
However, this movement often led to uniform urban landscapes that lacked the cultural richness and diversity of their predecessors (see Figure~\ref{fig:Transitionaltimeline} c)}.
\added{
The 20th century saw the rise of Modern Architecture, which advocated for `form follows function' and minimal ornamentation~\cite{Leach2016}, reflecting the need for functional, cost-effective buildings during the industrial age. Figures like Adolf Loos and Le Corbusier championed minimalism, rejecting excessive ornamentation in favor of efficiency, influencing a generation of architects to prioritize structural honesty and simplicity~\cite{Saglam2014}. However, this movement often led to uniform urban landscapes that lacked the cultural richness and diversity of their predecessors.
}
In response to Modernism's perceived limitations, the late 1960s saw the emergence of Postmodernism, spearheaded by thinkers like Robert Venturi.
\deleted{
Postmodernism critiqued the minimalist aesthetic and reintroduced complexity, ornament, and form into architectural design, advocating for buildings that engage more deeply with their cultural and historical contexts~\cite{Venturi1972}(see Figure~\ref{fig:Transitionaltimeline} d)}.
\added{
Postmodernism critiqued the stark uniformity of Modernism and reintroduced complexity, ornament, and historical references, advocating for buildings that engage more deeply with their cultural and historical contexts~\cite{Venturi1972}. This shift was driven by a desire to create architecture that was not only functional but also meaningful and contextually rich.
}

The late 20th and early 21st centuries have seen a resurgence in creativity and expression, with architects utilizing digital technologies to explore new realms of complexity and ornamentation~\cite{Burlando2019}\deleted{(see Figure~\ref{fig:contemporarytimeline})}.
The fusion of digital and physical design processes signals a shift towards the democratization of complex, parametric designs, indicative of a contemporary period that values ornamentation, functionality, and human comfort~\cite{Kim2019}.
This evolving trajectory of architecture suggests a future where design is deeply intertwined with societal values and technological possibilities.

\deleted{Facades and ornamentation, in this context, become critical in conveying these narratives, bridging the gap between the aesthetic and the symbolic, and establishing the interface between buildings and their environments, influencing comfort and energy efficiency while reflecting the building's identity~\cite{Kamal2020}.
The evolution of facade design and ornamentation mirrors societal transformations, technological progress, and shifts in artistic sensibilities, each impacting how communities relate to their built environment.}

\added{
Facades and ornamentation, in this context, become critical in conveying these narratives, bridging the gap between the aesthetic and the symbolic, and establishing the interface between buildings and their environments, influencing both aesthetic perception and functionality. As buildings became more energy-conscious, facade design also shifted to balance aesthetic appeal with environmental performance, such as optimizing natural light and improving energy efficiency while reflecting the building's identity~\cite{Kamal2020}. The evolution of facade design and ornamentation not only reflect societal transformations, technological progress, and shifts in artistic sensibilities, but also highlight the changing cultural values toward sustainability and user comfort, each impacting how communities relate to their built environment.
}

In conclusion, the historical context of architectural complexity reveals a rich tapestry of styles and philosophies, from ancient grandiosity to modern minimalism and contemporary innovation.
These shifts reflect the ongoing dialogue between simplicity and complexity, tradition and innovation, and functionality and aesthetics, shaping the built environment in ways that are both imaginative and responsive to societal needs.

