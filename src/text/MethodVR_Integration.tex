    %\subsubsection{VR integration and data visualization interface}
    %\label{subsubsec:VR_integration}
    %%Define the sections of the VR interface and how information is displayed
    %\input{Text/VR_integration}


The purpose of this component is to bridge the gap between the virtual environment generated by the `3D modeling and Environment Setup' component and the data derived from the CICA complexity analysis of facade variations (Figure~\ref{fig:MethodologyFlowchart}, element 3).

This module includes an immersive `VR simulation' and a `data visualization interface' that transports users into the VR simulation of the experiment's location for facade complexity analysis.
Within this dynamic virtual environment, participants can explore and interact with the building from both inside and outside, visualize its context, and manipulate the facade variations through the user interface (Figure~\ref{fig:VRinterface}).
The `VR simulation' was developed using Unity (v.2022.2.21f1), accessible through a Head-Mounted Display (HMD) like the Oculus Quest 2.
Unity was selected due to its robust VR support, which includes pre-built templates and seamless integration with Python and C\#.
This choice enhances our simulation's interactivity and data handling capabilities, allowing for the effective merging of complex 3D geometries in facade design with a dynamic data visualization interface and environment simulations.

%!Vr interface
The VR data visualization interface, integrated with the simulation, provides real-time feedback on facade variations, facilitating informed decision-making regarding facade complexity.
The interface is structured into five key sections: Viewpoint Navigation, Facade Variation Slider, Facade Render Preview, CICA Scores Comparative Analysis Charts, and Utility Functions (labeled 1 to 5 in Figure~\ref{fig:VRinterface}). Each section is designed to enhance usability and interpretability, serving as an interactive tool for effectively measuring user responses to varying levels of facade complexity.
This structure ensures that users can navigate, visualize, compare, and select facade variations with ease, thereby optimizing the design process.


