    %\subsubsection{VR integration and data visualization interface}
    %\label{subsubsec:VR_integration}
    %%Define the sections of the VR interface and how information is displayed
    %\input{Text/VR_integration}


The purpose of this component is to bridge the gap between the virtual environment generated by the `3D modeling' component and the data derived from the CICA complexity analysis of facade variations.
This integration culminates in an immersive virtual reality application developed using Unity (v.2022.2.21f1), accessible through a Head-Mounted Display (HMD) like the Oculus Quest 2.

This module includes an immersive simulation and a data visualization interface that transports users into the VR simulation of the experiment's location for facade complexity analysis.
Within this dynamic virtual environment, participants can explore and interact with the building from both inside and outside, visualize its context, and manipulate the facade variations through the user interface (Figure\ref{fig:VRInteriorExterior}).

Seamlessly integrated with the simulation, the interface provides real-time feedback on the impact of different facade variations on the building, facilitating more effective and informed decision-making when selecting a specific level of complexity.

%!Vr interface

The VR data visualization interface is structured into five key sections (labeled 1 to 5 in Figure\ref{fig:VRinterface}), each designed to enhance the usability and interpretability of the transition between facade variations:

\begin{enumerate}
\item \textit{Viewpoint Navigation:} Three buttons provide predetermined camera views, allowing users to examine facade variations from both the interior and exterior of the building (labeled 1 in Figure \ref{fig:VRinterface}).

\item \textit{Facade Variation Slider:} Positioned centrally, this slider enables users to transition through ten facade variations, including a 'level 0' representing the initial or current state of the actual building (no superimposed facade) (labeled 2 in Figure \ref{fig:VRinterface}).
Users can select their preferred facade variation, which updates the building envelope in the virtual environment.
The facade change is simultaneously reflected on the virtual building and in a scaled model inside the building, providing a comprehensive view of the facade's impact (Figure \ref{fig:VRInteriorExterior}).

\item \textit{Facade Image Preview:} This section displays a photo-realistic render of the building from the exterior, offering constant feedback on the facade's impact, especially useful when exploring the simulated interior (labeled 3 in Figure \ref{fig:VRinterface}).

\item \textit{Comparative Analysis Charts:} Featuring scatter graphs, this module contrasts the CICA complexity scores of all ten facade variations across each pattern, aiding users in making informed decisions during facade selection (labeled 4 in Figure \ref{fig:VRinterface}).

\item \textit{Utility Functions:} Located at the interface's base, two buttons enable users to reset the application for a new analysis cycle and save the selected facade variation, as well as navigate to the next pattern until the decision-making process for all three patterns is complete (labeled 5 in Figure \ref{fig:VRinterface}).
\end{enumerate}

Overall, the VR integration interface serves as an interactive tool for effectively measuring the user response to varying levels of complex facades in an intuitive and dynamic manner, it empowers designers to explore in a real world scale the implications that a facade has on the perception of a built space, making the decision-making process of selecting a specific level of complexity more efficient and informed.






