%!\section{Discussion}
%\label{sec:Discussion}
%%% This should explore the significance of the results of the work, not repeat them. A combined Results and Discussion section is often appropriate. Avoid extensive citations and discussion of published literature.
%%%!\section{Discussion}
%\label{sec:Discussion}
%%% This should explore the significance of the results of the work, not repeat them. A combined Results and Discussion section is often appropriate. Avoid extensive citations and discussion of published literature.
%%%!\section{Discussion}
%\label{sec:Discussion}
%%% This should explore the significance of the results of the work, not repeat them. A combined Results and Discussion section is often appropriate. Avoid extensive citations and discussion of published literature.
%%%!\section{Discussion}
%\label{sec:Discussion}
%%% This should explore the significance of the results of the work, not repeat them. A combined Results and Discussion section is often appropriate. Avoid extensive citations and discussion of published literature.
%%\input{Text/Discussion}


%!% Summary: A brief recap of your key results

Having laid out the results of our comprehensive study, encompassing the intricate analysis of facade complexity throughout history and the user responses to complex facades in contemporary design, we now transition to the discussion phase.

In this section, we interpret our findings, consider their implications, and reflect on their significance in the broader context of architectural trends and design preferences.
This discussion aims to shed light on the evolving relationship between users and the complexity of building facades, offering valuable insights for future construction practices and architectural design.

The multifaceted nature of our data, derived from both historical analysis and user interaction, positions us to answer the core research question of this study: ``What degree of complexity within facade design would users tolerate and accept for a building, and what insights do their preferences provide for future architectural trends?''

%!% Interpretations: What do your results mean?

    %! Historical Analysis graph
    %Discussion on results of history analysis and how theory aligns with the computational image analysis.

Our initial hypothesis, grounded in a rigorous examination of architectural styles from a theoretical standpoint, posited that the evolution of architecture comprises a dynamic interplay between simplicity and complexity (see Section\ref{subsec:TimelineArchitectureStyles}).
We anticipated that in the post-modernist era, contemporary architecture would experience a resurgence of complexity and ornamentation, exemplified by the emergence of five prominent architectural styles: Deconstructivism, Neofuturism, High-tech modernism, Parametricism, and Pragmatic utopianism(see Figure\ref{fig:contemporarytimeline}).

These styles, we contended, were shaped by advancements in technology, the proliferation of computer-aided design tools, and an overarching commitment to sustainability, collectively steering architectural design toward a trajectory characterized by heightened complexity.

Establishing the fundamental principle of architectural complexity is undeniably a complex undertaking, one that is deeply intertwined with numerous socio-economic factors shaping urban development.
Nonetheless, the development of the `Computational Image Complexity Analysis' (CICA) system was underpinned by a visionary premise: that the accumulation of a vast and diverse dataset, spanning architectural creations from centuries past to the contemporary era, could unveil discernible patterns.

These patterns, we hypothesized, would validate our initial proposition concerning the evolutionary essence of architecture - an enduring dialogue between simplicity and complexity, indicating an inexorable shift toward increased complexity in contemporary architecture.

In its quest to do so, the CICA system transcends disparities in building scale, geographic locations, and socio-economic drivers shaping urban development.
Instead, it seeks to reveal an overarching pattern that substantiates the resurgence of architectural complexity within the present epoch.

In accordance with our expectations, our analysis of building complexity throughout history, employing the CICA system, and depicted in the `Historical Complexity Analysis' Chart (Figure\ref{fig:HistoricalComplexityGraph}), meticulously organized by architectural style and year, was complemented by the inclusion of a polynomial trendline.
We opted for this specific trendline due to its versatility in accommodating the intricate data patterns that naturally emerge when assessing historical building complexity scores.
The outcome of this choice was the emergence of a curve characterized by an intriguing cyclic pattern.

This distinctive pattern, akin to an undulating curve traversing centuries, uncovers a continual oscillation between architectural complexity and simplicity.
It resembles the paradigmatic shifts in architectural design discussed within our theoretical analysis (see section\ref{subsec:FacadeandOrnament}), a phenomenon that has endured throughout architectural history.

The dynamic oscillation of the polynomial curve, illustrated in the `Historical Complexity Analysis' Chart in Figure\ref{fig:HistoricalComplexityGraph}, can be interpreted as peaks of ornamental richness giving way to troughs of minimalist restraint, relative to the standards of each era since each historical period has witnessed its unique interpretation of architectural complexity.

Furthermore, our analysis of the last 50 years of data within the trendline curve conspicuously illustrates an upward trajectory in architectural complexity.
This trend signifies a departure from the stark minimalism that characterized the decline of the Modernist movement between the 1950s and 1960s, thereby affirming our initial hypothesis.

These findings not only provide empirical validation of our hypothesis but also emphasize the cyclical nature of architectural complexity, reflecting the continuous dialogues within architectural practice across the ages.

    %! Experiment results discussion
    %Discussion on how the experiment results show a preference or positive take on complexity.



        %Consistent with the hypothesis, a discernible pattern emerges in the accuracy graphs (see Figure \ref{fig:Accuracyscattergraph}) when transitioning from a "screen-based" approach to a "VR-based" approach. The majority of users demonstrate increased proximity to the best-recommended solution provided by the system.

        %These results suggest a deeper understanding of the site and a heightened trust in the system. These findings align with the survey responses, indicating that users perceive the system's recommendations as highly valuable and express a strong likelihood of implementing them (see Figures \ref{fig:UsabilityChartSurvey} and \ref{fig:PerceptionSatisfactionSurveyResults}).

        %Moreover, similar studies, such as AstanehAsl et al. (2022) \cite{AstanehAsl2022}, which compared screen-based approaches with VR in resolving building system conflicts, also reported a greater comprehension of technical issues in immersive VR compared to screen-based approaches.

        %However, it is important to note that the wide standard deviation (\(SD = 44.1\%\)) observed in the analysis of VR system improvements can be attributed to the individual instances. These deviations encompass both substantial improvements of over  \(95\%\) in at least 12 sessions and negative results in 5 sessions, with one test subject experiencing a decline of up to \(-48\%\).

        %These individual variations highlight the potential and current limitations of the selected VR system, emphasizing the need for further refinement of the data visualization interface to enhance the understanding of system recommendations.

        %Effective communication plays a pivotal role in bridging the gap between large amounts of data and user perception, ensuring that instances of decline in identifying the best location for SLP are minimized or eliminated altogether.

        %As suggested by Marsden et al. \cite{Marsden2008}, to create a user-centered design for VR, an integrated approach is needed, which balances the influences of MR technology with SE (Software Engineering) and UE (User Experience) considerations.

        %Such an approach considers the technology-driven procedures of Mixed Reality development, like is the case of VR while maintaining the systematic, controllable, and manageable processes advocated by SE and integrating appropriate methods and procedures from UE to develop usable solutions that are applicable in practice.

        %Upon closer examination of the sites individually, a clear pattern emerges: the more heterogeneous the topography, the easier it is to identify suitable locations (see Figure \ref{fig:ImprovementPerSite}). This observation is supported by the lowest rate of improvement in Site 1, which had a relatively homogeneous terrain compared to Site 2, where the most pronounced level variations occurred (see Table \ref{tab:SiteParametersAndPreview}).

        %It appears that when dealing with topography, complexity is preferred, as it accentuates the differences between different locations, thereby expediting the decision-making process in filtering out undesired options. This principle, which is primarily based on visual cues from the simulation, can be translated into data visualization techniques, where even highlighting differences in homogeneous topography could result in an overall improvement in decision-making

%! Historical Analysis graph

%Followed by a tuning of the level of complexity how we interpret the survey results to propose a graphical conclusion of an apparent complex facade inside the boundaries of tolerance given by the experiment participants.



%!% Implications: Why do your results matter?
%% Your overall aim is to show the reader exactly what your research has contributed, and why they should care.


        %And while previous research has focused on small details \cite{AlSaggaf2021}, these results demonstrate that an overview of the whole scale of the project can also be impacted by the integration of VR techniques in a positive matter and perhaps show a larger impact that may tilt the preferential adoption of VR when solving this issue.

%!% limitations: What can’t your results tell us?

        %The findings of this research provide valuable insights into the potential of VR immersion in data-driven design for Site Layout Planning, as well as potential implications for other areas of the building design process. However, it is important to interpret the results with caution due to certain limitations. The generalizability of the findings is limited by the small sample size and the fact that the participants were limited to individuals affiliated with the university.

        %Furthermore, the experiment design aimed to simulate the current methodology used in addressing SLP through a "screen-based interaction" stage, which involved CAD plans, perspectives, and expert recommendations typically provided to design teams.

        %However, the introduction of data visualization techniques and charts was exclusive to the "VR stage" (see Figure \ref{fig:VRinterface}). As a result, it is challenging to determine the exact contribution of VR-based interaction versus more efficient data visualization techniques to the observed accuracy improvement. It can be speculated that if the same graphs and interface used in the VR stage were introduced to the screen-based stage, the observed accuracy improvement may have been different.

%!% recommendations: Avenues for further studies or analyses
%% Based on the discussion of your results, you can make recommendations for practical implementation or further research. Sometimes, the recommendations are saved for the conclusion. Suggestions for further research can lead directly from the limitations. Don’t just state that more studies should be done—give concrete ideas for how future work can build on areas that your own research was unable to address.

        %The limitations identified in this research present valuable opportunities for further investigation. Future studies can delve into the specific impact of VR immersion on data visualization techniques, aiming to understand how VR influences the adoption of data-driven design methods. To achieve this, an experiment can be designed to provide the same interface for both screen-based and VR interactions, facilitating a comprehensive analysis of the effects of VR on data visualization.

        %Moreover, it is essential for this experiment to explore innovative ways of presenting data, pushing the boundaries of traditional 2D constraints. The true potential of VR can be fully realized when data visualization techniques transcend the limitations of physical reality. By projecting data as spatial representations that generate visual feedback and alter the perception of reality, VR has the capacity to eliminate the constraints of traditional design approaches.

        %By embracing these possibilities, future research can unlock the true potential of VR in data-driven design. This can lead to advancements in how data is visualized, enhancing decision-making processes and paving the way for more immersive and interactive experiences.




%!% Summary: A brief recap of your key results

Having laid out the results of our comprehensive study, encompassing the intricate analysis of facade complexity throughout history and the user responses to complex facades in contemporary design, we now transition to the discussion phase.

In this section, we interpret our findings, consider their implications, and reflect on their significance in the broader context of architectural trends and design preferences.
This discussion aims to shed light on the evolving relationship between users and the complexity of building facades, offering valuable insights for future construction practices and architectural design.

The multifaceted nature of our data, derived from both historical analysis and user interaction, positions us to answer the core research question of this study: ``What degree of complexity within facade design would users tolerate and accept for a building, and what insights do their preferences provide for future architectural trends?''

%!% Interpretations: What do your results mean?

    %! Historical Analysis graph
    %Discussion on results of history analysis and how theory aligns with the computational image analysis.

Our initial hypothesis, grounded in a rigorous examination of architectural styles from a theoretical standpoint, posited that the evolution of architecture comprises a dynamic interplay between simplicity and complexity (see Section\ref{subsec:TimelineArchitectureStyles}).
We anticipated that in the post-modernist era, contemporary architecture would experience a resurgence of complexity and ornamentation, exemplified by the emergence of five prominent architectural styles: Deconstructivism, Neofuturism, High-tech modernism, Parametricism, and Pragmatic utopianism(see Figure\ref{fig:contemporarytimeline}).

These styles, we contended, were shaped by advancements in technology, the proliferation of computer-aided design tools, and an overarching commitment to sustainability, collectively steering architectural design toward a trajectory characterized by heightened complexity.

Establishing the fundamental principle of architectural complexity is undeniably a complex undertaking, one that is deeply intertwined with numerous socio-economic factors shaping urban development.
Nonetheless, the development of the `Computational Image Complexity Analysis' (CICA) system was underpinned by a visionary premise: that the accumulation of a vast and diverse dataset, spanning architectural creations from centuries past to the contemporary era, could unveil discernible patterns.

These patterns, we hypothesized, would validate our initial proposition concerning the evolutionary essence of architecture - an enduring dialogue between simplicity and complexity, indicating an inexorable shift toward increased complexity in contemporary architecture.

In its quest to do so, the CICA system transcends disparities in building scale, geographic locations, and socio-economic drivers shaping urban development.
Instead, it seeks to reveal an overarching pattern that substantiates the resurgence of architectural complexity within the present epoch.

In accordance with our expectations, our analysis of building complexity throughout history, employing the CICA system, and depicted in the `Historical Complexity Analysis' Chart (Figure\ref{fig:HistoricalComplexityGraph}), meticulously organized by architectural style and year, was complemented by the inclusion of a polynomial trendline.
We opted for this specific trendline due to its versatility in accommodating the intricate data patterns that naturally emerge when assessing historical building complexity scores.
The outcome of this choice was the emergence of a curve characterized by an intriguing cyclic pattern.

This distinctive pattern, akin to an undulating curve traversing centuries, uncovers a continual oscillation between architectural complexity and simplicity.
It resembles the paradigmatic shifts in architectural design discussed within our theoretical analysis (see section\ref{subsec:FacadeandOrnament}), a phenomenon that has endured throughout architectural history.

The dynamic oscillation of the polynomial curve, illustrated in the `Historical Complexity Analysis' Chart in Figure\ref{fig:HistoricalComplexityGraph}, can be interpreted as peaks of ornamental richness giving way to troughs of minimalist restraint, relative to the standards of each era since each historical period has witnessed its unique interpretation of architectural complexity.

Furthermore, our analysis of the last 50 years of data within the trendline curve conspicuously illustrates an upward trajectory in architectural complexity.
This trend signifies a departure from the stark minimalism that characterized the decline of the Modernist movement between the 1950s and 1960s, thereby affirming our initial hypothesis.

These findings not only provide empirical validation of our hypothesis but also emphasize the cyclical nature of architectural complexity, reflecting the continuous dialogues within architectural practice across the ages.

    %! Experiment results discussion
    %Discussion on how the experiment results show a preference or positive take on complexity.



        %Consistent with the hypothesis, a discernible pattern emerges in the accuracy graphs (see Figure \ref{fig:Accuracyscattergraph}) when transitioning from a "screen-based" approach to a "VR-based" approach. The majority of users demonstrate increased proximity to the best-recommended solution provided by the system.

        %These results suggest a deeper understanding of the site and a heightened trust in the system. These findings align with the survey responses, indicating that users perceive the system's recommendations as highly valuable and express a strong likelihood of implementing them (see Figures \ref{fig:UsabilityChartSurvey} and \ref{fig:PerceptionSatisfactionSurveyResults}).

        %Moreover, similar studies, such as AstanehAsl et al. (2022) \cite{AstanehAsl2022}, which compared screen-based approaches with VR in resolving building system conflicts, also reported a greater comprehension of technical issues in immersive VR compared to screen-based approaches.

        %However, it is important to note that the wide standard deviation (\(SD = 44.1\%\)) observed in the analysis of VR system improvements can be attributed to the individual instances. These deviations encompass both substantial improvements of over  \(95\%\) in at least 12 sessions and negative results in 5 sessions, with one test subject experiencing a decline of up to \(-48\%\).

        %These individual variations highlight the potential and current limitations of the selected VR system, emphasizing the need for further refinement of the data visualization interface to enhance the understanding of system recommendations.

        %Effective communication plays a pivotal role in bridging the gap between large amounts of data and user perception, ensuring that instances of decline in identifying the best location for SLP are minimized or eliminated altogether.

        %As suggested by Marsden et al. \cite{Marsden2008}, to create a user-centered design for VR, an integrated approach is needed, which balances the influences of MR technology with SE (Software Engineering) and UE (User Experience) considerations.

        %Such an approach considers the technology-driven procedures of Mixed Reality development, like is the case of VR while maintaining the systematic, controllable, and manageable processes advocated by SE and integrating appropriate methods and procedures from UE to develop usable solutions that are applicable in practice.

        %Upon closer examination of the sites individually, a clear pattern emerges: the more heterogeneous the topography, the easier it is to identify suitable locations (see Figure \ref{fig:ImprovementPerSite}). This observation is supported by the lowest rate of improvement in Site 1, which had a relatively homogeneous terrain compared to Site 2, where the most pronounced level variations occurred (see Table \ref{tab:SiteParametersAndPreview}).

        %It appears that when dealing with topography, complexity is preferred, as it accentuates the differences between different locations, thereby expediting the decision-making process in filtering out undesired options. This principle, which is primarily based on visual cues from the simulation, can be translated into data visualization techniques, where even highlighting differences in homogeneous topography could result in an overall improvement in decision-making

%! Historical Analysis graph

%Followed by a tuning of the level of complexity how we interpret the survey results to propose a graphical conclusion of an apparent complex facade inside the boundaries of tolerance given by the experiment participants.



%!% Implications: Why do your results matter?
%% Your overall aim is to show the reader exactly what your research has contributed, and why they should care.


        %And while previous research has focused on small details \cite{AlSaggaf2021}, these results demonstrate that an overview of the whole scale of the project can also be impacted by the integration of VR techniques in a positive matter and perhaps show a larger impact that may tilt the preferential adoption of VR when solving this issue.

%!% limitations: What can’t your results tell us?

        %The findings of this research provide valuable insights into the potential of VR immersion in data-driven design for Site Layout Planning, as well as potential implications for other areas of the building design process. However, it is important to interpret the results with caution due to certain limitations. The generalizability of the findings is limited by the small sample size and the fact that the participants were limited to individuals affiliated with the university.

        %Furthermore, the experiment design aimed to simulate the current methodology used in addressing SLP through a "screen-based interaction" stage, which involved CAD plans, perspectives, and expert recommendations typically provided to design teams.

        %However, the introduction of data visualization techniques and charts was exclusive to the "VR stage" (see Figure \ref{fig:VRinterface}). As a result, it is challenging to determine the exact contribution of VR-based interaction versus more efficient data visualization techniques to the observed accuracy improvement. It can be speculated that if the same graphs and interface used in the VR stage were introduced to the screen-based stage, the observed accuracy improvement may have been different.

%!% recommendations: Avenues for further studies or analyses
%% Based on the discussion of your results, you can make recommendations for practical implementation or further research. Sometimes, the recommendations are saved for the conclusion. Suggestions for further research can lead directly from the limitations. Don’t just state that more studies should be done—give concrete ideas for how future work can build on areas that your own research was unable to address.

        %The limitations identified in this research present valuable opportunities for further investigation. Future studies can delve into the specific impact of VR immersion on data visualization techniques, aiming to understand how VR influences the adoption of data-driven design methods. To achieve this, an experiment can be designed to provide the same interface for both screen-based and VR interactions, facilitating a comprehensive analysis of the effects of VR on data visualization.

        %Moreover, it is essential for this experiment to explore innovative ways of presenting data, pushing the boundaries of traditional 2D constraints. The true potential of VR can be fully realized when data visualization techniques transcend the limitations of physical reality. By projecting data as spatial representations that generate visual feedback and alter the perception of reality, VR has the capacity to eliminate the constraints of traditional design approaches.

        %By embracing these possibilities, future research can unlock the true potential of VR in data-driven design. This can lead to advancements in how data is visualized, enhancing decision-making processes and paving the way for more immersive and interactive experiences.




%!% Summary: A brief recap of your key results

Having laid out the results of our comprehensive study, encompassing the intricate analysis of facade complexity throughout history and the user responses to complex facades in contemporary design, we now transition to the discussion phase.

In this section, we interpret our findings, consider their implications, and reflect on their significance in the broader context of architectural trends and design preferences.
This discussion aims to shed light on the evolving relationship between users and the complexity of building facades, offering valuable insights for future construction practices and architectural design.

The multifaceted nature of our data, derived from both historical analysis and user interaction, positions us to answer the core research question of this study: ``What degree of complexity within facade design would users tolerate and accept for a building, and what insights do their preferences provide for future architectural trends?''

%!% Interpretations: What do your results mean?

    %! Historical Analysis graph
    %Discussion on results of history analysis and how theory aligns with the computational image analysis.

Our initial hypothesis, grounded in a rigorous examination of architectural styles from a theoretical standpoint, posited that the evolution of architecture comprises a dynamic interplay between simplicity and complexity (see Section\ref{subsec:TimelineArchitectureStyles}).
We anticipated that in the post-modernist era, contemporary architecture would experience a resurgence of complexity and ornamentation, exemplified by the emergence of five prominent architectural styles: Deconstructivism, Neofuturism, High-tech modernism, Parametricism, and Pragmatic utopianism(see Figure\ref{fig:contemporarytimeline}).

These styles, we contended, were shaped by advancements in technology, the proliferation of computer-aided design tools, and an overarching commitment to sustainability, collectively steering architectural design toward a trajectory characterized by heightened complexity.

Establishing the fundamental principle of architectural complexity is undeniably a complex undertaking, one that is deeply intertwined with numerous socio-economic factors shaping urban development.
Nonetheless, the development of the `Computational Image Complexity Analysis' (CICA) system was underpinned by a visionary premise: that the accumulation of a vast and diverse dataset, spanning architectural creations from centuries past to the contemporary era, could unveil discernible patterns.

These patterns, we hypothesized, would validate our initial proposition concerning the evolutionary essence of architecture - an enduring dialogue between simplicity and complexity, indicating an inexorable shift toward increased complexity in contemporary architecture.

In its quest to do so, the CICA system transcends disparities in building scale, geographic locations, and socio-economic drivers shaping urban development.
Instead, it seeks to reveal an overarching pattern that substantiates the resurgence of architectural complexity within the present epoch.

In accordance with our expectations, our analysis of building complexity throughout history, employing the CICA system, and depicted in the `Historical Complexity Analysis' Chart (Figure\ref{fig:HistoricalComplexityGraph}), meticulously organized by architectural style and year, was complemented by the inclusion of a polynomial trendline.
We opted for this specific trendline due to its versatility in accommodating the intricate data patterns that naturally emerge when assessing historical building complexity scores.
The outcome of this choice was the emergence of a curve characterized by an intriguing cyclic pattern.

This distinctive pattern, akin to an undulating curve traversing centuries, uncovers a continual oscillation between architectural complexity and simplicity.
It resembles the paradigmatic shifts in architectural design discussed within our theoretical analysis (see section\ref{subsec:FacadeandOrnament}), a phenomenon that has endured throughout architectural history.

The dynamic oscillation of the polynomial curve, illustrated in the `Historical Complexity Analysis' Chart in Figure\ref{fig:HistoricalComplexityGraph}, can be interpreted as peaks of ornamental richness giving way to troughs of minimalist restraint, relative to the standards of each era since each historical period has witnessed its unique interpretation of architectural complexity.

Furthermore, our analysis of the last 50 years of data within the trendline curve conspicuously illustrates an upward trajectory in architectural complexity.
This trend signifies a departure from the stark minimalism that characterized the decline of the Modernist movement between the 1950s and 1960s, thereby affirming our initial hypothesis.

These findings not only provide empirical validation of our hypothesis but also emphasize the cyclical nature of architectural complexity, reflecting the continuous dialogues within architectural practice across the ages.

    %! Experiment results discussion
    %Discussion on how the experiment results show a preference or positive take on complexity.



        %Consistent with the hypothesis, a discernible pattern emerges in the accuracy graphs (see Figure \ref{fig:Accuracyscattergraph}) when transitioning from a "screen-based" approach to a "VR-based" approach. The majority of users demonstrate increased proximity to the best-recommended solution provided by the system.

        %These results suggest a deeper understanding of the site and a heightened trust in the system. These findings align with the survey responses, indicating that users perceive the system's recommendations as highly valuable and express a strong likelihood of implementing them (see Figures \ref{fig:UsabilityChartSurvey} and \ref{fig:PerceptionSatisfactionSurveyResults}).

        %Moreover, similar studies, such as AstanehAsl et al. (2022) \cite{AstanehAsl2022}, which compared screen-based approaches with VR in resolving building system conflicts, also reported a greater comprehension of technical issues in immersive VR compared to screen-based approaches.

        %However, it is important to note that the wide standard deviation (\(SD = 44.1\%\)) observed in the analysis of VR system improvements can be attributed to the individual instances. These deviations encompass both substantial improvements of over  \(95\%\) in at least 12 sessions and negative results in 5 sessions, with one test subject experiencing a decline of up to \(-48\%\).

        %These individual variations highlight the potential and current limitations of the selected VR system, emphasizing the need for further refinement of the data visualization interface to enhance the understanding of system recommendations.

        %Effective communication plays a pivotal role in bridging the gap between large amounts of data and user perception, ensuring that instances of decline in identifying the best location for SLP are minimized or eliminated altogether.

        %As suggested by Marsden et al. \cite{Marsden2008}, to create a user-centered design for VR, an integrated approach is needed, which balances the influences of MR technology with SE (Software Engineering) and UE (User Experience) considerations.

        %Such an approach considers the technology-driven procedures of Mixed Reality development, like is the case of VR while maintaining the systematic, controllable, and manageable processes advocated by SE and integrating appropriate methods and procedures from UE to develop usable solutions that are applicable in practice.

        %Upon closer examination of the sites individually, a clear pattern emerges: the more heterogeneous the topography, the easier it is to identify suitable locations (see Figure \ref{fig:ImprovementPerSite}). This observation is supported by the lowest rate of improvement in Site 1, which had a relatively homogeneous terrain compared to Site 2, where the most pronounced level variations occurred (see Table \ref{tab:SiteParametersAndPreview}).

        %It appears that when dealing with topography, complexity is preferred, as it accentuates the differences between different locations, thereby expediting the decision-making process in filtering out undesired options. This principle, which is primarily based on visual cues from the simulation, can be translated into data visualization techniques, where even highlighting differences in homogeneous topography could result in an overall improvement in decision-making

%! Historical Analysis graph

%Followed by a tuning of the level of complexity how we interpret the survey results to propose a graphical conclusion of an apparent complex facade inside the boundaries of tolerance given by the experiment participants.



%!% Implications: Why do your results matter?
%% Your overall aim is to show the reader exactly what your research has contributed, and why they should care.


        %And while previous research has focused on small details \cite{AlSaggaf2021}, these results demonstrate that an overview of the whole scale of the project can also be impacted by the integration of VR techniques in a positive matter and perhaps show a larger impact that may tilt the preferential adoption of VR when solving this issue.

%!% limitations: What can’t your results tell us?

        %The findings of this research provide valuable insights into the potential of VR immersion in data-driven design for Site Layout Planning, as well as potential implications for other areas of the building design process. However, it is important to interpret the results with caution due to certain limitations. The generalizability of the findings is limited by the small sample size and the fact that the participants were limited to individuals affiliated with the university.

        %Furthermore, the experiment design aimed to simulate the current methodology used in addressing SLP through a "screen-based interaction" stage, which involved CAD plans, perspectives, and expert recommendations typically provided to design teams.

        %However, the introduction of data visualization techniques and charts was exclusive to the "VR stage" (see Figure \ref{fig:VRinterface}). As a result, it is challenging to determine the exact contribution of VR-based interaction versus more efficient data visualization techniques to the observed accuracy improvement. It can be speculated that if the same graphs and interface used in the VR stage were introduced to the screen-based stage, the observed accuracy improvement may have been different.

%!% recommendations: Avenues for further studies or analyses
%% Based on the discussion of your results, you can make recommendations for practical implementation or further research. Sometimes, the recommendations are saved for the conclusion. Suggestions for further research can lead directly from the limitations. Don’t just state that more studies should be done—give concrete ideas for how future work can build on areas that your own research was unable to address.

        %The limitations identified in this research present valuable opportunities for further investigation. Future studies can delve into the specific impact of VR immersion on data visualization techniques, aiming to understand how VR influences the adoption of data-driven design methods. To achieve this, an experiment can be designed to provide the same interface for both screen-based and VR interactions, facilitating a comprehensive analysis of the effects of VR on data visualization.

        %Moreover, it is essential for this experiment to explore innovative ways of presenting data, pushing the boundaries of traditional 2D constraints. The true potential of VR can be fully realized when data visualization techniques transcend the limitations of physical reality. By projecting data as spatial representations that generate visual feedback and alter the perception of reality, VR has the capacity to eliminate the constraints of traditional design approaches.

        %By embracing these possibilities, future research can unlock the true potential of VR in data-driven design. This can lead to advancements in how data is visualized, enhancing decision-making processes and paving the way for more immersive and interactive experiences.




%!% Summary: A brief recap of your key results

Having laid out the results of our comprehensive study, encompassing the intricate analysis of facade complexity throughout history and the user responses to complex facades in contemporary design, we now transition to the discussion phase.
In this section, we interpret our findings, consider their implications, and reflect on their significance in the broader context of architectural trends and design preferences.
This discussion aims to shed light on the evolving relationship between users and the complexity of building facades, offering valuable insights for future construction practices and architectural design.

The multifaceted nature of our data, derived from both historical analysis and user interaction, positions us to answer the core research question of this study: ``What degree of complexity within facade design would users tolerate and accept for a building, and what insights do their preferences provide for future architectural trends?''

%!% Interpretations: What do your results mean?

    %! Historical Analysis graph
    %Discussion on results of history analysis and how theory aligns with the computational image analysis.

Our initial hypothesis, grounded in a rigorous examination of architectural styles from a theoretical standpoint, posited that the evolution of architecture comprises a dynamic interplay between simplicity and complexity (see Section\ref{subsec:TimelineArchitectureStyles}).
We anticipated that in the post-modernist era, contemporary architecture would experience a resurgence of complexity and ornamentation, exemplified by the emergence of five prominent architectural styles: Deconstructivism, Neofuturism, High-tech modernism, Parametricism, and Pragmatic utopianism(see Figure\ref{fig:contemporarytimeline}).
These styles, we contended, were shaped by advancements in technology, the proliferation of computer-aided design tools, and an overarching commitment to sustainability, collectively steering architectural design toward a trajectory characterized by heightened complexity.

Establishing the fundamental principle of architectural complexity is undeniably a complex undertaking, one that is deeply intertwined with numerous socio-economic factors shaping urban development.
Nonetheless, the development of the `Computational Image Complexity Analysis' (CICA) system was underpinned by a visionary premise: that the accumulation of a vast and diverse dataset, spanning architectural creations from centuries past to the contemporary era, could unveil discernible patterns.
These patterns, we hypothesized, would validate our initial proposition concerning the evolutionary essence of architecture - an enduring dialogue between simplicity and complexity, indicating an inexorable shift toward increased complexity in contemporary architecture.
In its quest to do so, the CICA system transcends disparities in building scale, geographic locations, and socio-economic drivers shaping urban development.
Instead, it seeks to reveal an overarching pattern that substantiates the resurgence of architectural complexity within the present epoch.

In accordance with our expectations, our analysis of building complexity throughout history, employing the CICA system, and depicted in the `Historical Complexity Analysis' Chart (Figure\ref{fig:HistoricalComplexityGraph}), meticulously organized by architectural style and year, was complemented by the inclusion of a polynomial trendline.
We opted for this specific trendline due to its versatility in accommodating the intricate data patterns that naturally emerge when assessing historical building complexity scores.
The outcome of this choice was the emergence of a curve characterized by an intriguing cyclic pattern.
This distinctive pattern, akin to an undulating curve traversing centuries, uncovers a continual oscillation between architectural complexity and simplicity.
It resembles the paradigmatic shifts in architectural design discussed within our theoretical analysis (see section\ref{subsec:FacadeandOrnament}), a phenomenon that has endured throughout architectural history.

The dynamic oscillation of the polynomial curve, illustrated in the `Historical Complexity Analysis' Chart in Figure\ref{fig:HistoricalComplexityGraph}, can be interpreted as peaks of ornamental richness giving way to troughs of minimalist restraint, relative to the standards of each era since each historical period has witnessed its unique interpretation of architectural complexity.
Furthermore, our analysis of the last 50 years of data within the trendline curve conspicuously illustrates an upward trajectory in architectural complexity.
This trend signifies a departure from the stark minimalism that characterized the decline of the Modernist movement between the 1950s and 1960s, thereby affirming our initial hypothesis.
These findings not only provide empirical validation of our hypothesis but also emphasize the cyclical nature of architectural complexity, reflecting the continuous dialogues within architectural practice across the ages.

%! Experiment results discussion
    %Discussion on how the experiment results show a preference or positive take on complexity.

        %Consistent with the hypothesis, a discernible pattern emerges in the accuracy graphs (see Figure \ref{fig:Accuracyscattergraph}) when transitioning from a "screen-based" approach to a "VR-based" approach. The majority of users demonstrate increased proximity to the best-recommended solution provided by the system.

        %These results suggest a deeper understanding of the site and a heightened trust in the system. These findings align with the survey responses, indicating that users perceive the system's recommendations as highly valuable and express a strong likelihood of implementing them (see Figures \ref{fig:UsabilityChartSurvey} and \ref{fig:PerceptionSatisfactionSurveyResults}).

        %Moreover, similar studies, such as AstanehAsl et al. (2022) \cite{AstanehAsl2022}, which compared screen-based approaches with VR in resolving building system conflicts, also reported a greater comprehension of technical issues in immersive VR compared to screen-based approaches.

        %However, it is important to note that the wide standard deviation (\(SD = 44.1\%\)) observed in the analysis of VR system improvements can be attributed to the individual instances. These deviations encompass both substantial improvements of over  \(95\%\) in at least 12 sessions and negative results in 5 sessions, with one test subject experiencing a decline of up to \(-48\%\).

        %These individual variations highlight the potential and current limitations of the selected VR system, emphasizing the need for further refinement of the data visualization interface to enhance the understanding of system recommendations.

        %Effective communication plays a pivotal role in bridging the gap between large amounts of data and user perception, ensuring that instances of decline in identifying the best location for SLP are minimized or eliminated altogether.

        %As suggested by Marsden et al. \cite{Marsden2008}, to create a user-centered design for VR, an integrated approach is needed, which balances the influences of MR technology with SE (Software Engineering) and UE (User Experience) considerations.

        %Such an approach considers the technology-driven procedures of Mixed Reality development, like is the case of VR while maintaining the systematic, controllable, and manageable processes advocated by SE and integrating appropriate methods and procedures from UE to develop usable solutions that are applicable in practice.

        %Upon closer examination of the sites individually, a clear pattern emerges: the more heterogeneous the topography, the easier it is to identify suitable locations (see Figure \ref{fig:ImprovementPerSite}). This observation is supported by the lowest rate of improvement in Site 1, which had a relatively homogeneous terrain compared to Site 2, where the most pronounced level variations occurred (see Table \ref{tab:SiteParametersAndPreview}).

        %It appears that when dealing with topography, complexity is preferred, as it accentuates the differences between different locations, thereby expediting the decision-making process in filtering out undesired options. This principle, which is primarily based on visual cues from the simulation, can be translated into data visualization techniques, where even highlighting differences in homogeneous topography could result in an overall improvement in decision-making

%Followed by a tuning of the level of complexity how we interpret the survey results to propose a graphical conclusion of an apparent complex facade inside the boundaries of tolerance given by the experiment participants.


%! Survey discussion

%Q6 To what extent do you find the overall complexity of this facade design appealing?
The survey question 6, ``To what extent do you find the overall complexity of this facade design appealing?'' yielded an average rating of 4.3 on a 7-point Likert scale (Question 6 in Figure \ref{fig:SurveyQuestions6-10}).
This result is indicative of a generally positive perception of complex facade designs among the participants in our study.
A rating of 4.3 signifies a moderate-to-high level of appeal.
Participants, on average, expressed a significant degree of appreciation for complex facade designs.
This finding aligns with the broader architectural discourse, suggesting a growing acceptance of architectural complexity in contemporary design and highlights the potential for the integration of such designs in future architectural practices.

%Q7.How do you rate the intricacy of the patterns and textures used in this facade design?

The survey response indicating an average rating of 4.6 for the intricacy of patterns and textures in the facade design (Question 7 in Figure \ref{fig:SurveyQuestions6-10}) signifies a noteworthy degree of appreciation among participants in our study.
This result suggests that the majority of respondents found the intricacy of patterns and textures to be visually engaging and appealing.
It's worth noting that this positive response to intricate patterns and textures provides architects and designers with valuable feedback.
It encourages the exploration and incorporation of such elements into future architectural projects, knowing that they are likely to resonate with users and contribute positively to the overall appeal of the design.
This appreciation aligns with contemporary architectural preferences and suggests that intricate patterns and textures are an asset in creating visually appealing and engaging architectural facades.

%Q8. To what extent do you think the arrangement of architectural elements on this facade adds to its visual interest?

The survey results, with an average rating of 5.5, for the question, ``To what extent do you think the arrangement of architectural elements on this facade adds to its visual interest?'' (Question 8 in Figure \ref{fig:SurveyQuestions6-10}) reflects a relatively high level of endorsement for the visual interest created by the arrangement of architectural elements.
This finding underscores the importance of thoughtful and creative composition in architectural design.
It suggests that participants recognized and appreciated the skillful placement of elements on the facade.
The relatively high average score indicates a degree of consensus among participants regarding the impact of architectural element arrangement on visual interest.
This underscores the importance of skillful composition in architectural aesthetics and provides architects and designers with valuable insights into the preferences of potential users and observers.

%Q9. How complex do you perceive the facade's use of patterns and textures?

The survey results, regarding how participants perceived the patterns and textures used for the facade variations during the vr experiment, (Question 9 in Figure \ref{fig:SurveyQuestions6-10}), received an average rating of 3.9 on the Likert scale which falls slightly below the midpoint, signifying a neutral to moderately positive perception of complexity.
This response suggests that while participants recognized some level of intricacy in the patterns and textures employed, it did not strike them as highly complex.
It reflects a balanced viewpoint, suggesting that the facade design achieved a level of complexity that was neither too minimal nor overly intricate for the participants' tastes.
The perception of complexity in architectural design can be subjective and influenced by individual preferences and experiences.
Therefore, a moderate rating may indicate that the facade design applied for the facade variations successfully balanced complexity with accessibility, appealing to a broad range of participants.
This suggests that the design successfully balanced intricacy with accessibility, making it engaging without being overwhelming.
Architects and designers can interpret this result as a successful attempt to strike a harmonious balance in its use of patterns and textures.

%Q10. How detailed do you find the ornamentation on this facade design?

The responses to the question, ``How detailed do you find the ornamentation on this facade design?'' (Question 10 in Figure \ref{fig:SurveyQuestions6-10}) , with an average rating of 5.0 on a 7-point Likert scale, indicate that participants perceived the ornamentation on the experimental facade variations as moderately detailed.
This collective response suggests that the majority of participants found the level of detail in the ornamentation to be satisfactory and visually appealing.
In this context, it suggests that participants neither found the ornamentation overly simple nor excessively intricate.
From an architectural and design perspective, this feedback implies that the design successfully integrated ornamentation to enhance the visual appeal of the facade.
The balanced level of detail made the facade visually engaging without overwhelming the occupants.

%Q11. How much do the combination of materials contribute to the overall complexity of the facade?

Moving on to question 11, which inquired about the significance of the combination of materials in gauging the overall complexity of the facade variations (Figure \ref{fig:SurveyQuestions11-15}), participants provided an average rating of 4.9 on a 7-point Likert scale.
This response indicates that participants considered the combination of materials as a significant contributor to the overall complexity of the facade.
They emphasized that the choice and arrangement of materials effectively enhanced the facade's complexity.
This finding underscores the importance of materials in creating visually appealing and complex facades, highlighting how materials play a crucial role in defining the facade's character and can significantly impact its aesthetic appeal.
Architects and designers can take this feedback as an affirmation of the role of materials in achieving the desired level of complexity and visual interest in facade design.

%Q12. To what degree does the composition of the facade strike you as aesthetically intricate?

Having discussed the influence of material combinations on facade complexity in question 11, we now delve into participants' perceptions of the aesthetic intricacy brought about by the composition of these facades, as explored in question 12 of the survey (Figure \ref{fig:SurveyQuestions11-15}). This question examines whether the arrangement of architectural elements contributes to an aesthetically intricate facade.
The collected responses yielded an average rating of 4.9. This rating indicates that, on the whole, participants perceived the compositions of the facades as possessing a moderate degree of aesthetic intricacy.
It suggests that most participants observed a well-balanced blend of complexity and sophistication in the compositions, which likely positively influenced their overall impressions of the facade designs.
From a design standpoint, this feedback implies that the compositions of the facades successfully achieved a harmonious fusion of visual elements, resulting in an aesthetically pleasing and moderately intricate appearance.
This underscores the notion that a well-calibrated and moderately complex composition can augment the overall visual appeal of architectural facades.

%Q13. How much do you believe that the arrangement of shapes and forms on the facade contributes to its complexity?

Transitioning from the exploration of participants' perceptions of aesthetic intricacy in question 12, we now turn our attention to question 13 (Figure \ref{fig:SurveyQuestions11-15}). This question investigates the role of the arrangement of shapes and forms in contributing to the overall complexity of the facades.
Building upon our previous discussion, question 13 yielded an average rating of 6.3 on a 7-point Likert scale.
This substantial rating suggests that participants highly regarded the arrangement of shapes and forms as a significant contributor to the overall complexity of the facades.
This affirmation underscores the pivotal role of deliberate design decisions in crafting complex and visually captivating facades.
It solidifies the notion that the strategic placement of visual elements holds substantial sway over how a facade is perceived, highlighting its essential role in architectural design endeavors aimed at achieving complexity.

%Q14. How significantly does the use of color enhance the facade's visual complexity?

Building on the insights gathered from participants' assessments of the arrangement of shapes and forms in question 13, we now proceed to question 14 (Figure \ref{fig:SurveyQuestions11-15}). This question focuses on understanding the role of color in augmenting the visual complexity of the facades.
Exploring this further, question 14 produced an average rating of 5.1 on a 7-point Likert scale.
This rating implies that participants generally perceived the use of color as moderately significant in enhancing the visual complexity of the facades.
This recognition emphasizes the impact of color choices on architectural aesthetics.
It suggests that, while color does contribute to complexity, its role might be less pronounced compared to factors like shape arrangement.
This insight is valuable for architects and designers, indicating that while color is a consideration in creating visually engaging facades, other design elements may have a more substantial influence on complexity.
Overall, these findings reaffirm the multifaceted nature of architectural complexity, influenced by a combination of design factors that collectively shape how a facade is perceived by its viewers.

%Q15. How much depth and layering do you observe in the design of this facade?

As we transition from the assessment of color's role in enhancing facade complexity in question 14, we delve into question 15 (Figure \ref{fig:SurveyQuestions11-15}). This question probes into participants' perceptions of depth and layering in the design of the facades, examining the extent to which these factors contribute to the overall complexity.
Question 15 yielded an average rating of 4.3 on a 7-point Likert scale.
This rating implies that participants generally perceived a moderate degree of depth and layering in the design of the facades.
This observation indicates that the majority of participants recognized some level of depth and layering within the facade designs.
While not rated as highly significant as other factors like the arrangement of shapes and forms, it still contributes to the overall perception of complexity.
For architects and designers, this feedback underscores the importance of incorporating depth and layering elements judiciously into facade design to achieve a balanced level of complexity.
It also highlights the nuanced nature of architectural complexity, where various design elements, including color and depth, come together to shape the final aesthetic impression.

%!% Implications: Why do your results matter?
%% Your overall aim is to show the reader exactly what your research has contributed, and why they should care.

While individual aesthetic preferences can vary, the collective response suggests a shared understanding of the significance of this aspect of design.

        %And while previous research has focused on small details \cite{AlSaggaf2021}, these results demonstrate that an overview of the whole scale of the project can also be impacted by the integration of VR techniques in a positive matter and perhaps show a larger impact that may tilt the preferential adoption of VR when solving this issue.

%!% limitations: What can’t your results tell us?

        %The findings of this research provide valuable insights into the potential of VR immersion in data-driven design for Site Layout Planning, as well as potential implications for other areas of the building design process. However, it is important to interpret the results with caution due to certain limitations. The generalizability of the findings is limited by the small sample size and the fact that the participants were limited to individuals affiliated with the university.

        %Furthermore, the experiment design aimed to simulate the current methodology used in addressing SLP through a "screen-based interaction" stage, which involved CAD plans, perspectives, and expert recommendations typically provided to design teams.

        %However, the introduction of data visualization techniques and charts was exclusive to the "VR stage" (see Figure \ref{fig:VRinterface}). As a result, it is challenging to determine the exact contribution of VR-based interaction versus more efficient data visualization techniques to the observed accuracy improvement. It can be speculated that if the same graphs and interface used in the VR stage were introduced to the screen-based stage, the observed accuracy improvement may have been different.

%!% recommendations: Avenues for further studies or analyses
%% Based on the discussion of your results, you can make recommendations for practical implementation or further research. Sometimes, the recommendations are saved for the conclusion. Suggestions for further research can lead directly from the limitations. Don’t just state that more studies should be done—give concrete ideas for how future work can build on areas that your own research was unable to address.

        %The limitations identified in this research present valuable opportunities for further investigation. Future studies can delve into the specific impact of VR immersion on data visualization techniques, aiming to understand how VR influences the adoption of data-driven design methods. To achieve this, an experiment can be designed to provide the same interface for both screen-based and VR interactions, facilitating a comprehensive analysis of the effects of VR on data visualization.

        %Moreover, it is essential for this experiment to explore innovative ways of presenting data, pushing the boundaries of traditional 2D constraints. The true potential of VR can be fully realized when data visualization techniques transcend the limitations of physical reality. By projecting data as spatial representations that generate visual feedback and alter the perception of reality, VR has the capacity to eliminate the constraints of traditional design approaches.

        %By embracing these possibilities, future research can unlock the true potential of VR in data-driven design. This can lead to advancements in how data is visualized, enhancing decision-making processes and paving the way for more immersive and interactive experiences.

