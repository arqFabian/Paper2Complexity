%!\section{Discussion}
%\label{sec:Discussion}
%%% This should explore the significance of the results of the work, not repeat them. A combined Results and Discussion section is often appropriate. Avoid extensive citations and discussion of published literature.
%%%!\section{Discussion}
%\label{sec:Discussion}
%%% This should explore the significance of the results of the work, not repeat them. A combined Results and Discussion section is often appropriate. Avoid extensive citations and discussion of published literature.
%%%!\section{Discussion}
%\label{sec:Discussion}
%%% This should explore the significance of the results of the work, not repeat them. A combined Results and Discussion section is often appropriate. Avoid extensive citations and discussion of published literature.
%%%!\section{Discussion}
%\label{sec:Discussion}
%%% This should explore the significance of the results of the work, not repeat them. A combined Results and Discussion section is often appropriate. Avoid extensive citations and discussion of published literature.
%%\input{Text/Discussion}

Discussion on results of history analysis and how theory aligns with the computational image analysis.
Followed by a tuning of the level of complexity how we interpret the survey results to propose a graphical conclusion of an apparent complex facade inside the boundaries of tolerance given by the experiment participants.

%! Historical Analysis graph

Through Image Complexity Analysis, we assign a quantitative complexity value to these architectural images.
This complexity value is derived from a comprehensive examination of spatial intricacies, frequency patterns, and color information present within, based on the research strategy by Ishra et al\cite{Ishrat2020}.

By employing this quantitative approach, we aim to provide an objective measure of the shift from simplicity towards complexity in architecture, further strengthening the thesis of an impending era characterized by architectural intricacy and richness in facades and ornamentation.

This graphical representation unmistakably illustrates a growing inclination towards architecturally complex designs, further corroborating the conclusion drawn from our extensive literature review.
%=========
%
% Complexity of an image is governed by spatial, frequency and color information present in the image.\cite{Ishrat2020}
%
%Scanpath based image complexity analysis determines human visual behavior that could lead to development of interactive and intelligent systems.\cite{Ishrat2020}
%
%The objective of current research work is to establish the complexity of the given set of images while target objects are searched and to present analysis ofgaze search pattern.
%To achieve these objectives a remote gaze estimation and analysis model has been proposed for scanpath identification and analysis.\cite{Ishrat2020}


%! Experiment results discussions


Discussion on results of history analysis and how theory aligns with the computational image analysis.
Followed by a tuning of the level of complexity how we interpret the survey results to propose a graphical conclusion of an apparent complex facade inside the boundaries of tolerance given by the experiment participants.

%! Historical Analysis graph

Through Image Complexity Analysis, we assign a quantitative complexity value to these architectural images.
This complexity value is derived from a comprehensive examination of spatial intricacies, frequency patterns, and color information present within, based on the research strategy by Ishra et al\cite{Ishrat2020}.

By employing this quantitative approach, we aim to provide an objective measure of the shift from simplicity towards complexity in architecture, further strengthening the thesis of an impending era characterized by architectural intricacy and richness in facades and ornamentation.

This graphical representation unmistakably illustrates a growing inclination towards architecturally complex designs, further corroborating the conclusion drawn from our extensive literature review.
%=========
%
% Complexity of an image is governed by spatial, frequency and color information present in the image.\cite{Ishrat2020}
%
%Scanpath based image complexity analysis determines human visual behavior that could lead to development of interactive and intelligent systems.\cite{Ishrat2020}
%
%The objective of current research work is to establish the complexity of the given set of images while target objects are searched and to present analysis ofgaze search pattern.
%To achieve these objectives a remote gaze estimation and analysis model has been proposed for scanpath identification and analysis.\cite{Ishrat2020}


%! Experiment results discussions


Discussion on results of history analysis and how theory aligns with the computational image analysis.
Followed by a tuning of the level of complexity how we interpret the survey results to propose a graphical conclusion of an apparent complex facade inside the boundaries of tolerance given by the experiment participants.

%! Historical Analysis graph

Through Image Complexity Analysis, we assign a quantitative complexity value to these architectural images.
This complexity value is derived from a comprehensive examination of spatial intricacies, frequency patterns, and color information present within, based on the research strategy by Ishra et al\cite{Ishrat2020}.

By employing this quantitative approach, we aim to provide an objective measure of the shift from simplicity towards complexity in architecture, further strengthening the thesis of an impending era characterized by architectural intricacy and richness in facades and ornamentation.

This graphical representation unmistakably illustrates a growing inclination towards architecturally complex designs, further corroborating the conclusion drawn from our extensive literature review.
%=========
%
% Complexity of an image is governed by spatial, frequency and color information present in the image.\cite{Ishrat2020}
%
%Scanpath based image complexity analysis determines human visual behavior that could lead to development of interactive and intelligent systems.\cite{Ishrat2020}
%
%The objective of current research work is to establish the complexity of the given set of images while target objects are searched and to present analysis ofgaze search pattern.
%To achieve these objectives a remote gaze estimation and analysis model has been proposed for scanpath identification and analysis.\cite{Ishrat2020}


%! Experiment results discussions


Discussion on results of history analysis and how theory aligns with the computational image analysis.
Followed by a tuning of the level of complexity how we interpret the survey results to propose a graphical conclusion of an apparent complex facade inside the boundaries of tolerance given by the experiment participants.

%! Historical Analysis graph

Through Image Complexity Analysis, we assign a quantitative complexity value to these architectural images.
This complexity value is derived from a comprehensive examination of spatial intricacies, frequency patterns, and color information present within, based on the research strategy by Ishra et al\cite{Ishrat2020}.

By employing this quantitative approach, we aim to provide an objective measure of the shift from simplicity towards complexity in architecture, further strengthening the thesis of an impending era characterized by architectural intricacy and richness in facades and ornamentation.

This graphical representation unmistakably illustrates a growing inclination towards architecturally complex designs, further corroborating the conclusion drawn from our extensive literature review.
%=========
%
% Complexity of an image is governed by spatial, frequency and color information present in the image.\cite{Ishrat2020}
%
%Scanpath based image complexity analysis determines human visual behavior that could lead to development of interactive and intelligent systems.\cite{Ishrat2020}
%
%The objective of current research work is to establish the complexity of the given set of images while target objects are searched and to present analysis ofgaze search pattern.
%To achieve these objectives a remote gaze estimation and analysis model has been proposed for scanpath identification and analysis.\cite{Ishrat2020}


%! Experiment results discussions
