

    %In Complexity and Contradiction in Architecture, Robert Venturi tries to give counter-arguments to the modernist approach. He advocates embracing ‘contradiction and complexity’ to create valid, vital works.\cite{Lutolli2020}


    %It is noteworthy that he doesn’t oppose aesthetic simplicity. What he rejects is the ‘oversimplification’ of architecture, indicated when he inverted the famous Mies van der Rohe statement ‘Less is more’ into ‘Less is a bore’.\cite{Lutolli2020}

    %Modern architects, who shunned symbolism of form  as  an expression or reinforcement of content: meaning was  to  be  communicated,  not  through  allusion  to  previously  known forms, but through the inherent, physiognomic characteristics of form.
    %The creation of architectural form was to be a logical process, free  from images  of past experiences, determined solely by program and structure, with an  occasional  assist,  as  Alan Colquhoun has  suggested, from  intuition.
    But some recent critics  have  questioned  the possible level of content to  be derived  from  abstract forms.
    Others have  demonstrated that the functionalists,  despite  their protestations, derived  a  formal vocabulary of their own, mainly from current art movements and the industrial vernacular;
    and  latter-day  followers  such  as  the  Archigram  group  have turned,  while  similarly  protesting,  to Pop  Art and  the space  industry.\cite{Venturi1972}

   % Because the spatial relationships are made by symbols more than by forms, architecture in this landscape becomes symbol in space rather than form in space.
    Architecture defines very little: The big sign and the little building is the rule of Route 66.
    %The sign is more important than the architecture.
    %This is reflected in the proprietor's budget the sign at the front is a vulgar extravaganza, the building at the back, a modest necessity.
    The architecture is what is cheap.\cite{Venturi1972}

    %The commercial persuasion of road­side eclecticism provokes bold impact in the vast and complex setting of a  new landscape of big spaces, high speeds, and complex programs.\cite{Venturi1972}

    %parking lot is a current phase in the evolution of vast space since Versailles (Fig. 12). The space that divides high-speed highway and low, sparse buildings produces no enclosure and Iittle direction. \cite{Venturi1972}

    The little low buildings, gray-brown like the desert, separate and recede from the street that is now the highway, their false fronts disengaged and turned perpendicular to the highway as big, high signs.
    If you take the signs away, there is no place.
    The desert town is intensified communication along the highway.\cite{Venturi1972}

    Ugly and ordinary as symbol and style
    Heroic and Original, or ugly and ordinary

    %Robert Venturi, an iconoclastic architect often considered the father of postmodernism who rejected sterile, glass-cube structures in favor of an inclusive, eclectic style that embraced community values and a touch of vulgarity\cite{Schudel2018}

    He turned Mies van der Rohe’s famous dictum about simplicity in design — “Less is more” — upside down, cheekily declaring, “Less is a bore.”\cite{Schudel2018}

    Mr. Venturi’s declaration of architectural values, “Complexity and Contradiction,” was a manifesto that took aim at the prevailing modernist notion that architecture should aspire to a cold, glassy perfection with cold, glassy buildings.
    He argued instead that architecture should reflect changing times and social needs.
    The world of design, he said, had too long been in the grip of the dogma that architects were godlike creators who could impose their vision on the landscape.\cite{Schudel2018}

     Las Vegas wasn’t just a neon-lit den of vulgarity, they concluded.
     It was a prime example of a city built to accommodate the automobile.
     As a result, signs were often more important than the buildings they loomed over.\cite{Schudel2018}

    “In the landscape of the automobile, the architecture becomes insignificant, a pimple on the landscape of parking lots,” Mr. Venturi told the Times in 1971.\cite{Schudel2018}

    He never lost his disdain for what he considered the soulless architecture of his modernist predecessors, who often designed glass-box buildings with walls of windows, “but you would never have a wall with a window in it.”\cite{Schudel2018}

    Instead, Mr. Venturi wrote in “Complexity and Contradiction in Architecture,” he drew inspiration from the “everyday landscape, vulgar and disdained,” which was “valid and vital for our architecture as a whole.”\cite{Schudel2018}

    some architects wanted to move away from minimalist glass and steel and return to the ornamentation of the past. Postmodernists such as Michael Graves, James Stirling, Robert Venturi, and Denise Scott Brown responded to the work of their predecessors with bold buildings that showcased color and references to classical design. [...] Discover five of the most influential buildings of the postmodern movement and see how their eclectic and innovative designs pushed the boundaries of architecture in the 20th century\cite{Stamp2016}.

    Venturi wants modern architects to realize only one thing— perfection in the architectural world can and should include imperfection, in all its forms\cite{Lutolli2020}.


    %
    According to Krier the mature city achieves balance with nature and with the people that it serves in its scale, size and integration of residential, commercial and civic functions.
    Krier argues that the reconstruction of a city is a moral imperative, a global project that it is at once cultural social economic and ecological. Time of video 1:06:29\cite{Economakis2023}