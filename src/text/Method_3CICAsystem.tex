% this section describes the computational image complexity analysis
%!Concise version

The literature review in Section~\ref{sec:LiteratureReview} revealed a cyclical nature in architectural evolution, alternating between complex and simple styles.
Our initial goal for the CICA system is to empirically validate these trends by developing a quantifiable scoring system capable of evaluating the complexity of historical and 3D-modeled building facades (Figure~\ref{fig:MethodologyFlowchartComplexity}, element 2).

Implemented as a Python script, the CICA system leverages Python's compatibility with Blender and its robust CV libraries, facilitating the integration of 3D models with complexity analysis scripts.

Inspired by Venturi et al.'s perspective on complexity~\cite{Venturi1977}, the CICA system measures complexity by the mental processing time required for a building's elements.
It uses two primary metrics: edge density and contour count, selected for their relevance to human vision's edge and object contour detection~\cite{Yang2022}.

\textit{Edge Density:} Utilizing the Canny Edge Detection algorithm~\cite{EdgeOpenCV2023}, this metric focuses on edge presence and density, defining architectural boundaries (Figure~\ref{fig:CICA_and_VR_flowchart}, element [(2)]).
\deleted{(Table 1, column 3).}

\textit{Contour Count:} Using contour approximation techniques~\cite{ContourOpenCV2023}, this metric assesses shape intricacies outlined by edges (Figure~\ref{fig:CICA_and_VR_flowchart}, element [(3)]).
\deleted{(Table 1, column 4).}

Both metrics are essential for shaping perceived complexity and are computationally efficient for large datasets~\cite{Yang2022}.

To quantify facade complexity from both metrics perspective, we employed a Multi-Objective Optimization (MOO) algorithm using the Analytic Hierarchy Process (AHP), a robust Multi-Criteria Decision-Making (MCDM) technique for detailed analysis and prioritization based on expert input and quantitative data~\cite{Taherdoost2023}.
The MOO algorithm is represented in the `Complexity Score' function \(f_1(x)\), defined in Equation~\ref{eq:F1_ComplexityScoreFunction1}, which normalizes the metrics and combines them into a `Unified Complexity Score':

\begin{equation}
    f_1(x) = \mathrm{round}\left(\sum_{i=1}^{n} w_i \cdot a_i, 2\right) = \text{complexity\_score}
    \label{eq:F1_ComplexityScoreFunction1}
\end{equation}

where \(n\) is the number of performance indicators, \(w_i\) is the weight of the \(i\)-th element, and \(a_i\) is the normalized score for the \(i\)-th metric (e.g., `Edge Density' and `Contour Count').
This weighted sum provides the overall complexity score or `CICA score' a quantifiable measure of facade complexity, crucial for the CICA system.

The CICA system has two main applications:
\begin{itemize}
    \item \textit{Historical Analysis:} Evaluates over 180 buildings from various architectural eras, creating a scatter graph of complexity scores organized by year and architecture style, showing complexity trends over time (Figure~\ref{fig:CICA_and_VR_flowchart}[b]). Results are presented in Section~\ref{subsec:ResultsComplexityImageAnalysishistory}.\deleted{(Figure 4, element 2[b]; Table 3[b]).}

    \item \textit{3D-Modeled Facades Analysis:} Analyzes 10 facade variations across 3 patterns of 3D-modeled facades with varying complexity for the VR experiment (Figure~\ref{fig:CICA_and_VR_flowchart}[a]). The CICA scores are used for comparison with user perceptions~(Figure~\ref{fig:CICAscatterGraphRender}).\deleted{(Figure 4, element 2[a]; Table 3[a]).}
\end{itemize}

Through these applications, the CICA system aims to validate architectural complexity trends empirically and prepare for experiments assessing user perceptions of facade complexity.
