%\subsection{Studies on aesthetic preferences related to compexity}
%\label{subsec:TimelineArchitectureStyles}

%--------
%Evolution and Theoretical Foundations of Architectural Complexity
%Introduction to Complexity in Architectural Design
%Historical Context: Discuss the evolution of architectural complexity from classical architecture to modernist minimalism and contemporary design.
%Theoretical Foundations: Explore foundational theories and concepts, such as Birkhoff's ratio of order to complexity \cite{Birkhoff1933}.Evolution and Theoretical Foundations of Architectural Complexity
%--------

%Evolution and Theoretical Foundations of Architectural Complexity


Historical Context


Architecture stands as a unique art form, transforming the ordinary into the extraordinary while fulfilling functional purposes~\cite{Hnin2022}.
Its evolution is marked by a dynamic interplay between simplicity and complexity, reflecting societal values and technological advancements~\cite{Economakis2023}.
From early architectural styles like Romanesque, characterized by robust and simplistic forms, to the Gothic period with its intricate, skyward designs~\cite{Stacbond2020}, architecture has oscillated between simplicity and complexity (see Figure~\ref{fig:Oldtimeline} a,b).

The Renaissance heralded a revival of Greek and Roman ideals with a focus on symmetry, followed by the Baroque's lavish ornamentation in the 16th century~\cite{Economakis2023} (see Figure~\ref{fig:Oldtimeline} c,d).
The Neoclassical style, dominant in the 18th and 19th centuries, emphasized symmetry and classical principles while integrating new technologies like reinforced concrete~\cite{Economakis2023} (see Figure~\ref{fig:Oldtimeline} e).
Art Nouveau and Art Deco, emerging in the late 19th and early 20th centuries, celebrated nature and technological advancements, marking a departure from Neoclassical restraint~\cite{Salas2018, Arora2023} (see Figure~\ref{fig:Middletimeline} a, b).

The 20th century saw the rise of Modern Architecture, which advocated for `form follows function' and minimal ornamentation, a significant departure from previous ornate designs~\cite{Gage2015}.
Figures like Adolf Loos and Le Corbusier championed minimalism and functionality, influencing a generation of architects to prioritize structural honesty and simplicity~\cite{Saglam2014}.
However, this movement often led to uniform urban landscapes that lacked the cultural richness and diversity of their predecessors (see Figure~\ref{fig:Middletimeline} c).

In response to Modernism's perceived limitations, the late 1960s saw the emergence of Postmodernism, spearheaded by thinkers like Robert Venturi.
Postmodernism critiqued the minimalist aesthetic and reintroduced complexity, ornament, and form into architectural design, advocating for buildings that engage more deeply with their cultural and historical contexts~\cite{Venturi1972} (see Figure~\ref{fig:Middletimeline} d).

The late 20th and early 21st centuries have seen a resurgence in creativity and expression, with architects utilizing digital technologies to explore new realms of complexity and ornamentation~\cite{Burlando2019} (see Figure~\ref{fig:contemporarytimeline}).
The fusion of digital and physical design processes signals a shift towards the democratization of complex, parametric designs, indicative of a contemporary period that values ornamentation, functionality, and human comfort~\cite{Schwab2016}.
This evolving trajectory of architecture suggests a future where design is deeply intertwined with societal values and technological possibilities.

Facades and ornamentation, in this context, become critical in conveying these narratives, bridging the gap between the aesthetic and the symbolic, and establishing the interface between buildings and their environments, influencing comfort and energy efficiency while reflecting the building's identity~\cite{Kamal2020}.
The evolution of facade design and ornamentation mirrors societal transformations, technological progress, and shifts in artistic sensibilities, each impacting how communities relate to their built environment.

In conclusion, the historical context of architectural complexity reveals a rich tapestry of styles and philosophies, from ancient grandiosity to modern minimalism and contemporary innovation.
These shifts reflect the ongoing dialogue between simplicity and complexity, tradition and innovation, and functionality and aesthetics, shaping the built environment in ways that are both imaginative and responsive to societal needs.

Theoretical Foundations

Previous research has extensively explored the impact of complexity in architectural design, showing that elements such as chaotic patterns and fractal geometry significantly influence user perceptions and aesthetic preferences~\cite{Bies2016}.
Birkhoff's mathematical formalization of the ratio of order to complexity has been instrumental in understanding aesthetic value~\cite{Birkhoff1933}.
Contemporary studies build on these foundations, suggesting that balanced complexity can create environments that are both stimulating and comfortable~\cite{Redies2015}.
Despite these insights, the architectural field has yet to fully develop frameworks that leverage these principles for practical design applications, particularly in the context of modern technological advancements.


Contemporary architecture, however, has seen a resurgence of interest in complexity, driven by advancements in BIM and digital fabrication.
These technologies have allowed architects to explore forms and structures that were previously impractical or impossible to realize.
The renewed focus on complexity reflects a broader understanding that architectural design can and should engage with both functional requirements and aesthetic aspirations, offering environments that are visually stimulating and experientially rich~\cite{Gage2015}.

One of the foundational theoretical contributions to understanding architectural complexity is George David Birkhoff's theory of aesthetic measure, introduced in 1933. Birkhoff proposed that aesthetic value could be quantified through a ratio of order to complexity. His formula, M = O/C (where M is the aesthetic measure, O is order, and C is complexity), suggests that the most aesthetically pleasing designs achieve a balance between these two elements~\cite{Birkhoff1933}. This theory provided a framework for evaluating the aesthetic appeal of architectural forms based on their structural and decorative elements.

Birkhoff's ideas have influenced subsequent research on architectural complexity, encouraging a systematic approach to studying how different design elements contribute to the overall aesthetic experience. This balance between order and complexity is central to contemporary efforts to integrate complexity into architectural design in a way that enhances both user satisfaction and sustainability.

Another significant theoretical framework is Christopher Alexander's concept of `pattern language,' introduced in the 1970s.
Alexander's work emphasized the importance of recurring design patterns that resonate with human users on a fundamental level.
His theory suggests that certain patterns, when combined effectively, can create environments that feel harmonious and alive.
This approach aligns with the principles of biophilic design, which seeks to connect architecture with natural elements to enhance human well-being.

Contemporary research continues to build on these theoretical foundations, exploring how advanced technologies can be used to create complex designs that are both aesthetically pleasing and functionally effective. The integration of digital tools such as BIM and computational design methods has enabled architects to push the boundaries of complexity, creating structures that are not only visually striking but also optimized for performance and sustainability.

In summary, the evolution of architectural complexity is a testament to the ongoing interplay between cultural, technological, and theoretical influences. From ancient grandiosity to modern minimalism and contemporary innovation, architects have continually sought to balance order and complexity to create meaningful and engaging built environments. Theoretical frameworks such as Birkhoff's aesthetic measure and Alexander's pattern language provide valuable insights into how complexity can be harnessed to enhance architectural design, offering a foundation for future explorations in this dynamic field.


%However, the indiscriminate pursuit of popular trends using new technologies can lead to designs that quickly become outdated.
%Fashion-like gimmicks, often adopted without a deeper understanding of their long-term impact, do not necessarily resonate with users over time.
%Consequently, many buildings constructed in the last fifty years are now facing demolition, lacking enduring qualities that foster a genuine connection with their inhabitants \cite{Aesthetic2022}.
%This highlights a critical need for architectural practices that not only embrace innovation but also ensure relevance and sustainability through designs that engage and inspire \cite{Brielmann2022}.
%
%Despite significant research aimed at optimizing resource use and boosting environmental sustainability, the disconnect between data-driven design and user satisfaction remains pronounced.
%While computational advancements hold potential to transform urban landscapes, neglecting the psychological connection between people and their built environment can doom even sustainably designed buildings to obsolescence, incurring substantial societal and environmental costs \cite{Aesthetic2022}.
%Neuroscientists and other researchers have been working to define the aspects that stimulate our satisfaction with the built environment that transcend subjective aesthetic judgments, suggesting that certain structural elements resonate universally on a neurophysiological level, offering emotional nourishment and promoting urban well-being \cite{Brielmann2022}.
%These themes are often found in Biophilic design patterns that articulate the relationships between nature, human biology, and the design of the built environment.
%
%Historical and contemporary architectures have frequently incorporated natural patterns and complexities into their designs, underscoring a fundamental human affinity for biophilic elements that enhance psychological and physical well-being in urban environments.
%These design strategies, grounded in biophilic principles, align closely with the ingrained human need for interaction with nature-like settings, supporting a vibrant existence within urban landscapes \cite{Browning2014}.
%The recurring use of fractal geometries, seen in the repetitive patterns of elements like clouds, trees, and fern leaves, exemplifies nature's inherent complexity that has historically influenced architectural aesthetics.
%These natural complexities, when incorporated into building designs, place observers in a 'comfort zone' of sensory engagement, balancing stimulation and comfort through visually rich yet orderly environments \cite{Browning2014}.
%
%This well-established relationship between natural patterns and human well-being was mathematically formalized by Birkhoff (1933), who introduced a ratio of order to complexity as a measure of aesthetic value \cite{Birkhoff1933}.
%Despite some uncertainties, there is a widely held consensus within the field that exploring the scientific dimensions of aesthetics and appealing stimuli can offer profound insights into human brain functionality and behavior, significantly enriching our comprehension of these processes \cite{Redies2015}.
%This belief supports the quantitative frameworks established by Birkhoff in 1933, which assess and integrate complexity into design, emphasizing a balanced and engaging complexity as essential for effective architectural expression \cite{Birkhoff1933}.
%Despite these insights, the architectural field has not fully developed a framework that allows us to leverage these principles in complexity analysis.
%Although many tools exist for environmental optimization of buildings, there is still a gap in establishing a balance between complexity and order.
%This research seeks to bridge this gap by incorporating a methodology for measuring the complexity outputted during the design phase of building facade design, generating quantifiable data on the order of complexity of iterations of facade design that can be used to improve data-driven optimization models for building design.