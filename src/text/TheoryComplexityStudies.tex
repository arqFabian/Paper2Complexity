%Historical Context
\subsection{Evolution of Architectural Styles: Oscillations Between Simplicity and Complexity}
\label{subsec:TimelineArchitectureStyles}

Architecture stands as a unique art form, transforming the ordinary into the extraordinary while fulfilling functional purposes~\cite{Jiang2023}.
Its evolution is marked by a dynamic interplay between simplicity and complexity, reflecting societal values and technological advancements~\cite{Economakis2023}.
From early architectural styles like Romanesque, characterized by robust and simplistic forms, to the Gothic period with its intricate, skyward designs~\cite{Stacbond2020}, architecture has oscillated between simplicity and complexity (see Figure~\ref{fig:Oldtimeline} a,b).

The Renaissance heralded a revival of Greek and Roman ideals with a focus on symmetry, followed by the Baroque's lavish ornamentation in the 16th century~\cite{Economakis2023} (see Figure~\ref{fig:Oldtimeline} c,d).
The Neoclassical style, dominant in the 18th and 19th centuries, emphasized symmetry and classical principles while integrating new technologies like reinforced concrete~\cite{Economakis2023} (see Figure~\ref{fig:Oldtimeline} e).
Art Nouveau and Art Deco, emerging in the late 19th and early 20th centuries, celebrated nature and technological advancements, marking a departure from Neoclassical restraint~\cite{Salas2018, Arora2023} (see Figure~\ref{fig:Middletimeline} a, b).

The 20th century saw the rise of Modern Architecture, which advocated for `form follows function' and minimal ornamentation, a significant departure from previous ornate designs~\cite{Leach2016}.
Figures like Adolf Loos and Le Corbusier championed minimalism and functionality, influencing a generation of architects to prioritize structural honesty and simplicity~\cite{Saglam2014}.
However, this movement often led to uniform urban landscapes that lacked the cultural richness and diversity of their predecessors (see Figure~\ref{fig:Middletimeline} c).

In response to Modernism's perceived limitations, the late 1960s saw the emergence of Postmodernism, spearheaded by thinkers like Robert Venturi.
Postmodernism critiqued the minimalist aesthetic and reintroduced complexity, ornament, and form into architectural design, advocating for buildings that engage more deeply with their cultural and historical contexts~\cite{Venturi1972} (see Figure~\ref{fig:Middletimeline} d).

The late 20th and early 21st centuries have seen a resurgence in creativity and expression, with architects utilizing digital technologies to explore new realms of complexity and ornamentation~\cite{Burlando2019} (see Figure~\ref{fig:contemporarytimeline}).
The fusion of digital and physical design processes signals a shift towards the democratization of complex, parametric designs, indicative of a contemporary period that values ornamentation, functionality, and human comfort~\cite{Schwab2016}.
This evolving trajectory of architecture suggests a future where design is deeply intertwined with societal values and technological possibilities.

Facades and ornamentation, in this context, become critical in conveying these narratives, bridging the gap between the aesthetic and the symbolic, and establishing the interface between buildings and their environments, influencing comfort and energy efficiency while reflecting the building's identity~\cite{Kamal2020}.
The evolution of facade design and ornamentation mirrors societal transformations, technological progress, and shifts in artistic sensibilities, each impacting how communities relate to their built environment.

In conclusion, the historical context of architectural complexity reveals a rich tapestry of styles and philosophies, from ancient grandiosity to modern minimalism and contemporary innovation.
These shifts reflect the ongoing dialogue between simplicity and complexity, tradition and innovation, and functionality and aesthetics, shaping the built environment in ways that are both imaginative and responsive to societal needs.


\subsection{Theoretical Foundations of Architectural Complexity}
\label{subsec:ComplexityStudies}

Previous research has extensively explored the impact of complexity in architectural design, showing that elements such as chaotic patterns and fractal geometry significantly influence user perceptions and aesthetic preferences~\cite{Bies2016}.
Contemporary studies suggest that balanced complexity can create environments that are both stimulating and comfortable~\cite{Redies2015}.
However, the architectural field has yet to fully develop frameworks that leverage these principles for practical design applications, particularly with modern technological advancements.

A foundational theoretical contribution to understanding architectural complexity is George David Birkhoff's theory of aesthetic measure, introduced in 1933.
Birkhoff proposed that aesthetic value could be quantified through a ratio of order to complexity, expressed as M = O/C (where M is the aesthetic measure, O is order, and C is complexity)\cite{Douchova2016}.
This theory provided a framework for evaluating the aesthetic appeal of architectural forms based on their structural and decorative elements.
Despite challenges to the validity of Birkhoff's method for penalizing complexity, the concepts of order and complexity, along with objective quantification methods, remain significant in aesthetic evaluation functions\cite{Javaheri2016}.
This balance is central to contemporary efforts to integrate complexity into architectural design, enhancing both user satisfaction and sustainability.

Another significant theoretical framework is Alexander et al.~concept of `pattern language,' introduced in the 1970s~\cite{Alexander1977}.
Alexander et al.~emphasized the importance of recurring design patterns that resonate fundamentally with human users.
is theory suggests that certain patterns, when combined effectively, create environments that feel harmonious and alive, aligning with biophilic design principles that seek to connect architecture with natural elements to enhance human well-being~\cite{Downton2017}.

Browning et al.~(2014) contribute to this discussion by emphasizing the importance of balancing complexity and order in architectural design.
Their research highlights that visually engaging and information-rich spaces should strike a balance between being overly simplistic and overwhelmingly complex.
They draw from studies on fractal geometries found in nature, art, and architecture, suggesting that certain fractal dimensions (D=1.3-1.8) are preferred for their aesthetic and stress-reducing qualities~\cite{Browning2014}.
Fractal designs with three iterations are particularly effective in conveying a sense of order and intrigue.
These principles can be applied to various architectural elements to promote visually stimulating environments that support psychological and cognitive well-being.
However, they caution against the extremes of non-fractal or overly complex designs, which can induce stress or discomfort.
The goal is to integrate fractal geometries and hierarchies into design to create environments that are both engaging and restorative~\cite{Browning2014}.

A more recent method by Lee et al.~(2023) uses fractal dimension analysis to measure the visual complexity of architectural facades, which is crucial for assessing aesthetic character and predicting attractiveness.
They utilized the differential box counting method, which is better suited for handling greyscale images, to calculate fractal dimensions based on grey-level variations.
These fractal dimension values are then used to predict human visual preferences, providing a reliable measure of visual complexity in architectural design~\cite{Lee2023}.
Lee et al.~concluded that computational measures of visual complexity (fractal dimensions) and attractive strength (visual attention simulation) can effectively quantify the visual attractiveness of architectural facades.
Their findings indicate that these measures can distinguish different architectural styles, despite some limitations.
Importantly, they found that visual complexity (D) and attractive strength (S) are not mathematically correlated, suggesting that engagement and appeal may be independent qualities.
The proposed model for predicting visual attractiveness, A = D × S, will require further validation.
This study highlights the significant influence of visual attractiveness on perceptions of architecture and the need to evaluate these attributes during the design process~\cite{Lee2023}.

Contemporary research continues to build on these theoretical foundations, exploring how advanced technologies can be used to create complex designs that are both aesthetically pleasing and functionally effective.
The integration of digital tools such as BIM and computational design methods has enabled architects to push the boundaries of complexity, creating structures that are not only visually striking but also optimized for performance and sustainability, offering visually stimulating and experientially rich environments~\cite{Leach2016}~(Figure~\ref{fig:contemporarytimeline}).

In summary, the evolution of architectural complexity reflects an ongoing interplay between cultural, technological, and theoretical influences.
From ancient grandiosity to modern minimalism and contemporary innovation, architects have continually sought to balance order and complexity to create meaningful and engaging built environments.
Theoretical frameworks such as Birkhoff's aesthetic measure, Alexander's pattern language, and Lee et al.'s fractal dimension analysis provide valuable insights into harnessing complexity to enhance architectural design, offering a foundation for future explorations in this dynamic field.

Despite significant advancements in understanding architectural complexity and its impact on user perceptions, there remains a notable gap in the integration of theoretical insights with practical applications in the context of modern technological advancements.
Current methodologies often lack the ability to provide real-time, interactive evaluations of facade complexity, limiting their applicability in dynamic design environments.
My research aims to bridge this gap by developing a comprehensive system that combines immersive VR experiences with CV algorithms embedded in the CICA system.
This approach allows for the quantification of facade complexity in a way that is both interactive and responsive to user feedback, providing a practical tool for architects to optimize design complexity while considering aesthetic and functional aspects.
By incorporating advanced digital tools and empirical data, this research seeks to enhance the understanding of how complexity can be effectively managed and utilized in contemporary architectural practices.

