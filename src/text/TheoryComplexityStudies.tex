%\subsection{Studies on aesthetic preferences related to compexity}
%\label{subsec:TimelineArchitectureStyles}

However, the indiscriminate pursuit of popular trends using new technologies can lead to designs that quickly become outdated.
Fashion-like gimmicks, often adopted without a deeper understanding of their long-term impact, do not necessarily resonate with users over time.
Consequently, many buildings constructed in the last fifty years are now facing demolition, lacking enduring qualities that foster a genuine connection with their inhabitants \cite{Aesthetic2022}.
This highlights a critical need for architectural practices that not only embrace innovation but also ensure relevance and sustainability through designs that engage and inspire \cite{Brielmann2022}.

Despite significant research aimed at optimizing resource use and boosting environmental sustainability, the disconnect between data-driven design and user satisfaction remains pronounced.
While computational advancements hold potential to transform urban landscapes, neglecting the psychological connection between people and their built environment can doom even sustainably designed buildings to obsolescence, incurring substantial societal and environmental costs \cite{Aesthetic2022}.
Neuroscientists and other researchers have been working to define the aspects that stimulate our satisfaction with the built environment that transcend subjective aesthetic judgments, suggesting that certain structural elements resonate universally on a neurophysiological level, offering emotional nourishment and promoting urban well-being \cite{Brielmann2022}.
These themes are often found in Biophilic design patterns that articulate the relationships between nature, human biology, and the design of the built environment.

Historical and contemporary architectures have frequently incorporated natural patterns and complexities into their designs, underscoring a fundamental human affinity for biophilic elements that enhance psychological and physical well-being in urban environments.
These design strategies, grounded in biophilic principles, align closely with the ingrained human need for interaction with nature-like settings, supporting a vibrant existence within urban landscapes \cite{Browning2014}.
The recurring use of fractal geometries, seen in the repetitive patterns of elements like clouds, trees, and fern leaves, exemplifies nature's inherent complexity that has historically influenced architectural aesthetics.
These natural complexities, when incorporated into building designs, place observers in a 'comfort zone' of sensory engagement, balancing stimulation and comfort through visually rich yet orderly environments \cite{Browning2014}.

This well-established relationship between natural patterns and human well-being was mathematically formalized by Birkhoff (1933), who introduced a ratio of order to complexity as a measure of aesthetic value \cite{Birkhoff1933}.
Despite some uncertainties, there is a widely held consensus within the field that exploring the scientific dimensions of aesthetics and appealing stimuli can offer profound insights into human brain functionality and behavior, significantly enriching our comprehension of these processes \cite{Redies2015}.
This belief supports the quantitative frameworks established by Birkhoff in 1933, which assess and integrate complexity into design, emphasizing a balanced and engaging complexity as essential for effective architectural expression \cite{Birkhoff1933}.
Despite these insights, the architectural field has not fully developed a framework that allows us to leverage these principles in complexity analysis.
Although many tools exist for environmental optimization of buildings, there is still a gap in establishing a balance between complexity and order.
This research seeks to bridge this gap by incorporating a methodology for measuring the complexity outputted during the design phase of building facade design, generating quantifiable data on the order of complexity of iterations of facade design that can be used to improve data-driven optimization models for building design.