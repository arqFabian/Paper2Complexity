%!% limitations: What can’t your results tell us?

\textbf{Limitations}

The findings of this research provide valuable insights  into architectural complexity, yet certain limitations warrant cautious interpretation of the results:
%1 Sample Size and Demographic Representation:
This research involved a relatively small sample of 10 participants, primarily composed by university students and faculty members, which may limit the generalizability of the findings, as the results might not fully represent the broader population's preferences and perceptions of complexity.

%2 Virtual Reality Environment:
The choice of VR assesment as the core strategy for this research offered a controlled and immersive environment for evaluating user preferences, however it may not entirely capture the experience of interacting with real-world facades.
The VR setting could potentially affect participants' perceptions of complexity and comfort, leading to results that may differ from real-world reactions.

%3 CICA System Metrics and Historical Analysis Dataset:
In regard to the CICA system, developed for this research, it successfully provided an evaluation of architectural complexity using specific metrics, however,it may not capture all elements influencing perceived complexity.
This underscores the need for further exploration of the subjective nature of complexity perception, which can be shaped by individual aesthetic preferences, prior experiences, and cultural factors, aspects that computer vision algorithms might struggle to fully capture.

The effectiveness of computer vision algorithms depends on the quality and representativeness of the dataset used for training and analysis.
A limited dataset, as in the case of the 177 buildings used in this research, might restrict the comprehensiveness of the complexity assessment and while this provides a useful overview, expanding the dataset could yield a more detailed understanding of trends in architectural complexity over time.

%4 Focus on Facade Design
This research concentrated on facade design, which is just one aspect of architectural complexity.
Computer vision models developed for specific architectural features might not generalize well to other styles or unique design elements, potentially limiting their applicability across diverse architectural contexts.

%!% limitations: What can’t your results tell us?

%\textbf{Limitations}
%
%The findings of this research provide valuable insights, however, it is important to interpret the results with caution due to certain limitations.
%
%However, it's essential to recognize that every research endeavor carries its unique set of limitations, and our study is no exception.
%In the spirit of transparency, we candidly acknowledge these limitations, which we will address.
%Nevertheless, the following subsections, which meticulously analyze our results, were arranged in accordance with the core themes and research questions that guided our inquiry.

        %The findings of this research provide valuable insights into the potential of VR immersion in data-driven design for Site Layout Planning, as well as potential implications for other areas of the building design process. However, it is important to interpret the results with caution due to certain limitations. The generalizability of the findings is limited by the small sample size and the fact that the participants were limited to individuals affiliated with the university.

        %Furthermore, the experiment design aimed to simulate the current methodology used in addressing SLP through a "screen-based interaction" stage, which involved CAD plans, perspectives, and expert recommendations typically provided to design teams.

        %However, the introduction of data visualization techniques and charts was exclusive to the "VR stage" (see Figure \ref{fig:VRinterface}). As a result, it is challenging to determine the exact contribution of VR-based interaction versus more efficient data visualization techniques to the observed accuracy improvement. It can be speculated that if the same graphs and interface used in the VR stage were introduced to the screen-based stage, the observed accuracy improvement may have been different.