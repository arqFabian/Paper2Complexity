%\subsection{Future work}
    %\label{subsec:Future_work}
    %% highlight potential avenues for further research and development.

%% recommendations: Avenues for further studies or analyses

        %% Based on the discussion of your results, you can make recommendations for practical implementation or further research. Sometimes, the recommendations are saved for the conclusion. Suggestions for further research can lead directly from the limitations. Don’t just state that more studies should be done—give concrete ideas for how future work can build on areas that your own research was unable to address.

The insights into the cultural and historical significance of facades underscore the importance of context in architectural complexity analysis.
In improving the CICA system's assessments, we must consider for future iterations not only the quantitative complexity of a facade but also its cultural resonance and historical context, ensuring a holistic evaluation of architectural evolution.

%1Sample Size and Demographic Representation:
Conduct the study with a larger and more diverse group of participants to enhance the generalizability of the findings. This could include individuals from different geographic locations, cultural backgrounds, and professions.
 Implement long-term studies to observe how preferences for architectural complexity evolve over time and in different contexts. This could provide insights into the sustainability of design trends and user adaptability.

%2 Virtual Reality Environment:
Compare VR-based assessments with evaluations of physical facades to understand how virtual experiences correlate with real-world perceptions. This could help refine VR methodologies for architectural research.

Utilize emerging technologies, such as augmented reality (AR) and artificial intelligence (AI), to enhance the assessment and prediction of user preferences in architectural design.

%1 Sample Size and Demographic Representation and Study Duration::
 A larger and more diverse participant pool would enhance the generalizability of the findings.
Future studies should aim to include a more varied demographic to capture a wider range of viewpoints.
Additionally, Long-term studies could shed light on how these preferences evolve over time and in different contexts.

%3 CICA System Metrics and Historical Analysis Dataset:
 Develop and incorporate additional metrics into the CICA system to capture a broader range of complexity factors.
This could include aspects such as color, texture, and contextual integration.

%4 Focus on Facade Design
Future research should consider other elements, such as interior design and spatial organization, to gain a more holistic understanding of architectural complexity.
Extend the focus beyond facade design to include interior spaces and overall building organization. This could provide a more comprehensive understanding of architectural complexity.


%!Conclusion paragraph
By embracing these possibilities, future research can unlock the true potential of complexity in ......



%!% recommendations: Avenues for further studies or analyses
%% Based on the discussion of your results, you can make recommendations for practical implementation or further research. Sometimes, the recommendations are saved for the conclusion. Suggestions for further research can lead directly from the limitations. Don’t just state that more studies should be done—give concrete ideas for how future work can build on areas that your own research was unable to address.

        %The limitations identified in this research present valuable opportunities for further investigation. Future studies can delve into the specific impact of VR immersion on data visualization techniques, aiming to understand how VR influences the adoption of data-driven design methods. To achieve this, an experiment can be designed to provide the same interface for both screen-based and VR interactions, facilitating a comprehensive analysis of the effects of VR on data visualization.

        %Moreover, it is essential for this experiment to explore innovative ways of presenting data, pushing the boundaries of traditional 2D constraints. The true potential of VR can be fully realized when data visualization techniques transcend the limitations of physical reality. By projecting data as spatial representations that generate visual feedback and alter the perception of reality, VR has the capacity to eliminate the constraints of traditional design approaches.

        %By embracing these possibilities, future research can unlock the true potential of VR in data-driven design. This can lead to advancements in how data is visualized, enhancing decision-making processes and paving the way for more immersive and interactive experiences.