%\subsection{Future work}
    %\label{subsec:Future_work}
    %% highlight potential avenues for further research and development.

%% recommendations: Avenues for further studies or analyses

        %% Based on the discussion of your results, you can make recommendations for practical implementation or further research. Sometimes, the recommendations are saved for the conclusion. Suggestions for further research can lead directly from the limitations. Don’t just state that more studies should be done—give concrete ideas for how future work can build on areas that your own research was unable to address.

This research's findings and limitations provide opportunities for further exploration in architectural complexity, enhancing our understanding of its impact on user experiences and preferences.

%1Sample Size and Demographic Representation:

Future studies should aim to involve a larger and more diverse group of participants, encompassing various geographic locations, cultural backgrounds, and professions to broaden the generalizability of the findings.
Additionally, conducting long-term studies could shed light on the evolution of preferences for architectural complexity over time and across different contexts.
Such research could offer valuable insights into the sustainability of design trends over time, and user adaptability, as well as provide a more comprehensive understanding from a wider range of viewpoints.

%2 Virtual Reality Environment:

Future research could compare VR-based assessments with evaluations of physical facades to better understand the correlation between virtual experiences and real-world perceptions.
This comparison could help refine VR methodologies for architectural research.
Additionally, leveraging emerging technologies in Extended Reality (ER), such as Mixed Reality (MR) and Augmented Reality (AR), could further bridge the gap between virtual simulations and reality, enhancing the assessment and prediction of user preferences in complexity in architectural design.


%3 CICA System Metrics and Historical Analysis Dataset:

The CICA system presents opportunities for further development and enhancement.
Future works could improve the accuracy of the CICA system by incorporating additional metrics to encompass a wider array of complexity factors, such as color, texture, and contextual integration.
Future iterations should consider both the quantitative aspects of facade complexity and the cultural resonance and historical context to provide a comprehensive evaluation of architectural evolution.
A strategy to achieve this developing methodologies that integrate user feedback more directly into the design process.
This could lead to more personalized and culturally sensitive architectural solutions, providing insights into how cultural influences shape preferences for architectural complexity and informing culturally sensitive design practices.

%4 Focus on Facade Design

Future research should broaden its scope to encompass additional elements, such as interior design and spatial organization, to achieve a more holistic understanding of architectural complexity.
By extending the focus from facade design to include interior spaces and overall building organization, a more comprehensive perspective on architectural complexity can be attained.
Additionally, future studies should examine the relationship between architectural complexity and sustainability, exploring how complex designs can either contribute to or detract from sustainable building practices.

Future research should extend its scope beyond facade design to include interior design and spatial organization for a holistic understanding of architectural complexity.
Additionally, future studies should examine the relationship between architectural complexity and sustainability, exploring how complex designs can either contribute to or detract from sustainable building practices.
