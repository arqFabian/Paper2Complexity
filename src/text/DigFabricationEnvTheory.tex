
%!\subsection{Digital fabrication and Environmental Sustainability}
%!\label{subsec:DigitalFabricationAndEnvSustainability}

%!==========================
%8. **Environmental Sustainability:** Investigate how the integration of complex facades and digital fabrication aligns with contemporary sustainability principles. Examine how parametric design and data-driven approaches can enhance energy efficiency, reduce material waste, and contribute to sustainable construction practices.
%!==========================

Digital fabrication technologies, meanwhile, are interacting with the biological world on a daily basis.
Engineers, designers, and architects are combining computational design, additive manufacturing, materials engineering, and synthetic biology to pioneer a symbiosis between microorganisms, our bodies, the products we consume, and even the buildings we inhabit\cite{Schwab2016}.


    If articulation has taken over from ornament in the architecture of abstract expressionism, space is what displaced symbolism: space dramatized by an acrobatic use of light.
    Our heroic and original symbols, from carceri to Cape Kennedy, feed our late Romantic egos and satisfy our need for spectacular, expressionistic space for a new age in architecture.
    Today, however, most buildings need reasonably low ceilings and windows rather than glass walls for light, to contain the air conditioning and meet the budget.
    Therefore our aesthetic impact should come from sources other than light and space, more symbolic and less spatial sources\cite{Venturi1971}



