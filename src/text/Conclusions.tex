%!%\section{Conclusions}
%%\label{sec:Conclusion}
%%%!%\section{Conclusions}
%%\label{sec:Conclusion}
%%%!%\section{Conclusions}
%%\label{sec:Conclusion}
%%%!%\section{Conclusions}
%%\label{sec:Conclusion}
%%\input{Text/Conclusions}

%% The main conclusions of the study may be presented in a short Conclusions section, which may stand alone or form a subsection of a Discussion or Results and Discussion section.
%% Answer the Research question

%revision

This paper explores architectural design at the intersection of digital fabrication, virtual reality assessment, and computer vision to deepen our understanding of intricate facade design.
Our primary goal is to gauge user tolerance and acceptance of complex facades, offering insights into future construction practices.
A literature review confirms a resurgence of complexity in contemporary architecture.
We introduce the `Computational Image Complexity Analysis' (CICA) system, using computer vision to quantitatively analyzing building complexity across epochs, revealing an upward complexity trendline since the late 20th century.

In a virtual reality experiment, we quantify user responses to complex facades, shedding light on their tolerance and acceptance levels.
On average, participants favor complexity, with an average score of 3.82 (out of 10) and a \(40\%\) probability of selecting a score close to this value according to CICA.
Post-survey qualitative scores averaged 4.9 on a 7-point Likert scale, indicating favorable attitudes toward complex facade variations.
These results align with the historical analysis, confirming contemporary architecture's embrace of complexity.

%! conclusions from survey results
 %Q9
 Architects and designers can consider this result as a reflection of the delicate balance required in incorporating patterns and textures into facades.
 While complexity can be visually engaging, it should not overshadow other design elements or create visual clutter.
 A moderate perception of complexity suggests that the design struck a reasonable equilibrium in this regard.

%Q10

Furthermore, a moderate rating implies that the design may have struck a chord with a broad range of participants, appealing to those who appreciate both intricate and simpler ornamentation styles.
 This adaptability in the perception of detail suggests that the design achieved a versatile and inclusive approach to ornamentation.

%Q11

From an architectural standpoint, this finding underscores the importance of materials in creating visually appealing and complex facades.
 The materials used in construction play a crucial role in defining the facade's character and can significantly impact its aesthetic appeal.

%Q12
The participants' feedback on the aesthetic intricacy of the facade compositions aligns with the idea that a balanced and moderate level of complexity in composition can enhance the overall visual appeal of architectural facades.

%Q13
The results indicate that the majority of participants recognized the pivotal role played by the arrangement of architectural elements in enhancing the facade's complexity.

%Q14
While color does contribute to complexity, its role might be less pronounced compared to factors like shape arrangement.Color has a role in creating visually engaging facades, however other design elements may have a more substantial influence on complexity.

%!draft ideas
 Architects may be responding to a growing emphasis on user experience and well-being.
 Complex facades could be designed to engage occupants and create aesthetically pleasing environments.

%!Ideas for interpretations of upwards trajectory of trendline in historical analysis chart graph

%Resurgence of Ornamentation: This trend may reflect a resurgence of interest in architectural ornamentation and intricate facade design. After the stark minimalism of the mid-20th century, architects may be returning to more decorative and visually complex elements in building design.
%
%Advancements in Technology: The last half-century has witnessed significant technological advancements, particularly in computer-aided design and construction techniques. These technological leaps may have empowered architects to explore more complex and daring designs that were previously challenging to execute.
%
%Sustainability and Integration: Contemporary architectural practices emphasize sustainability and the integration of buildings with their surroundings. Complex facade designs may facilitate innovative approaches to sustainability, such as incorporating natural ventilation and shading strategies.
%
%Expression of Identity: In an era of globalization, architects may be using complex facade designs as a means of expressing regional or cultural identities. This could explain the resurgence of ornamentation and complex patterns.
%
%Human-Centric Design: Architects may be responding to a growing emphasis on user experience and well-being. Complex facades could be designed to engage occupants and create aesthetically pleasing environments.
%
%Economic Prosperity: Economic factors can also influence architectural trends. Periods of economic prosperity often lead to more ambitious and complex building designs, as resources are more readily available for experimentation.
%
%Post-Modernism and Pluralism: The rejection of the rigid principles of Modernism in favor of post-modernism and architectural pluralism has contributed to a more diverse range of design possibilities, including complex facades.
%
%Interdisciplinary Collaboration: Architects are increasingly collaborating with other disciplines, such as artists and engineers. This interdisciplinary approach may encourage more complex and innovative facade designs.
%
%Environmental Considerations: The focus on environmental sustainability may drive architects to explore facade designs that optimize energy efficiency, daylighting, and thermal comfort, often requiring intricate solutions.
%
%Global Connectivity: Architects today have access to a wealth of global architectural influences and references through the internet. This exposure to diverse architectural traditions may encourage experimentation with complex facades.

%% The main conclusions of the study may be presented in a short Conclusions section, which may stand alone or form a subsection of a Discussion or Results and Discussion section.
%% Answer the Research question

%revision

This paper explores architectural design at the intersection of digital fabrication, virtual reality assessment, and computer vision to deepen our understanding of intricate facade design.
Our primary goal is to gauge user tolerance and acceptance of complex facades, offering insights into future construction practices.
A literature review confirms a resurgence of complexity in contemporary architecture.
We introduce the `Computational Image Complexity Analysis' (CICA) system, using computer vision to quantitatively analyzing building complexity across epochs, revealing an upward complexity trendline since the late 20th century.

In a virtual reality experiment, we quantify user responses to complex facades, shedding light on their tolerance and acceptance levels.
On average, participants favor complexity, with an average score of 3.82 (out of 10) and a \(40\%\) probability of selecting a score close to this value according to CICA.
Post-survey qualitative scores averaged 4.9 on a 7-point Likert scale, indicating favorable attitudes toward complex facade variations.
These results align with the historical analysis, confirming contemporary architecture's embrace of complexity.

%! conclusions from survey results
 %Q9
 Architects and designers can consider this result as a reflection of the delicate balance required in incorporating patterns and textures into facades.
 While complexity can be visually engaging, it should not overshadow other design elements or create visual clutter.
 A moderate perception of complexity suggests that the design struck a reasonable equilibrium in this regard.

%Q10

Furthermore, a moderate rating implies that the design may have struck a chord with a broad range of participants, appealing to those who appreciate both intricate and simpler ornamentation styles.
 This adaptability in the perception of detail suggests that the design achieved a versatile and inclusive approach to ornamentation.

%Q11

From an architectural standpoint, this finding underscores the importance of materials in creating visually appealing and complex facades.
 The materials used in construction play a crucial role in defining the facade's character and can significantly impact its aesthetic appeal.

%Q12
The participants' feedback on the aesthetic intricacy of the facade compositions aligns with the idea that a balanced and moderate level of complexity in composition can enhance the overall visual appeal of architectural facades.

%Q13
The results indicate that the majority of participants recognized the pivotal role played by the arrangement of architectural elements in enhancing the facade's complexity.

%Q14
While color does contribute to complexity, its role might be less pronounced compared to factors like shape arrangement.Color has a role in creating visually engaging facades, however other design elements may have a more substantial influence on complexity.

%!draft ideas
 Architects may be responding to a growing emphasis on user experience and well-being.
 Complex facades could be designed to engage occupants and create aesthetically pleasing environments.

%!Ideas for interpretations of upwards trajectory of trendline in historical analysis chart graph

%Resurgence of Ornamentation: This trend may reflect a resurgence of interest in architectural ornamentation and intricate facade design. After the stark minimalism of the mid-20th century, architects may be returning to more decorative and visually complex elements in building design.
%
%Advancements in Technology: The last half-century has witnessed significant technological advancements, particularly in computer-aided design and construction techniques. These technological leaps may have empowered architects to explore more complex and daring designs that were previously challenging to execute.
%
%Sustainability and Integration: Contemporary architectural practices emphasize sustainability and the integration of buildings with their surroundings. Complex facade designs may facilitate innovative approaches to sustainability, such as incorporating natural ventilation and shading strategies.
%
%Expression of Identity: In an era of globalization, architects may be using complex facade designs as a means of expressing regional or cultural identities. This could explain the resurgence of ornamentation and complex patterns.
%
%Human-Centric Design: Architects may be responding to a growing emphasis on user experience and well-being. Complex facades could be designed to engage occupants and create aesthetically pleasing environments.
%
%Economic Prosperity: Economic factors can also influence architectural trends. Periods of economic prosperity often lead to more ambitious and complex building designs, as resources are more readily available for experimentation.
%
%Post-Modernism and Pluralism: The rejection of the rigid principles of Modernism in favor of post-modernism and architectural pluralism has contributed to a more diverse range of design possibilities, including complex facades.
%
%Interdisciplinary Collaboration: Architects are increasingly collaborating with other disciplines, such as artists and engineers. This interdisciplinary approach may encourage more complex and innovative facade designs.
%
%Environmental Considerations: The focus on environmental sustainability may drive architects to explore facade designs that optimize energy efficiency, daylighting, and thermal comfort, often requiring intricate solutions.
%
%Global Connectivity: Architects today have access to a wealth of global architectural influences and references through the internet. This exposure to diverse architectural traditions may encourage experimentation with complex facades.

%% The main conclusions of the study may be presented in a short Conclusions section, which may stand alone or form a subsection of a Discussion or Results and Discussion section.
%% Answer the Research question

%revision

This paper explores architectural design at the intersection of digital fabrication, virtual reality assessment, and computer vision to deepen our understanding of intricate facade design.
Our primary goal is to gauge user tolerance and acceptance of complex facades, offering insights into future construction practices.
A literature review confirms a resurgence of complexity in contemporary architecture.
We introduce the `Computational Image Complexity Analysis' (CICA) system, using computer vision to quantitatively analyzing building complexity across epochs, revealing an upward complexity trendline since the late 20th century.

In a virtual reality experiment, we quantify user responses to complex facades, shedding light on their tolerance and acceptance levels.
On average, participants favor complexity, with an average score of 3.82 (out of 10) and a \(40\%\) probability of selecting a score close to this value according to CICA.
Post-survey qualitative scores averaged 4.9 on a 7-point Likert scale, indicating favorable attitudes toward complex facade variations.
These results align with the historical analysis, confirming contemporary architecture's embrace of complexity.

%! conclusions from survey results
 %Q9
 Architects and designers can consider this result as a reflection of the delicate balance required in incorporating patterns and textures into facades.
 While complexity can be visually engaging, it should not overshadow other design elements or create visual clutter.
 A moderate perception of complexity suggests that the design struck a reasonable equilibrium in this regard.

%Q10

Furthermore, a moderate rating implies that the design may have struck a chord with a broad range of participants, appealing to those who appreciate both intricate and simpler ornamentation styles.
 This adaptability in the perception of detail suggests that the design achieved a versatile and inclusive approach to ornamentation.

%Q11

From an architectural standpoint, this finding underscores the importance of materials in creating visually appealing and complex facades.
 The materials used in construction play a crucial role in defining the facade's character and can significantly impact its aesthetic appeal.

%Q12
The participants' feedback on the aesthetic intricacy of the facade compositions aligns with the idea that a balanced and moderate level of complexity in composition can enhance the overall visual appeal of architectural facades.

%Q13
The results indicate that the majority of participants recognized the pivotal role played by the arrangement of architectural elements in enhancing the facade's complexity.

%Q14
While color does contribute to complexity, its role might be less pronounced compared to factors like shape arrangement.Color has a role in creating visually engaging facades, however other design elements may have a more substantial influence on complexity.

%!draft ideas
 Architects may be responding to a growing emphasis on user experience and well-being.
 Complex facades could be designed to engage occupants and create aesthetically pleasing environments.

%!Ideas for interpretations of upwards trajectory of trendline in historical analysis chart graph

%Resurgence of Ornamentation: This trend may reflect a resurgence of interest in architectural ornamentation and intricate facade design. After the stark minimalism of the mid-20th century, architects may be returning to more decorative and visually complex elements in building design.
%
%Advancements in Technology: The last half-century has witnessed significant technological advancements, particularly in computer-aided design and construction techniques. These technological leaps may have empowered architects to explore more complex and daring designs that were previously challenging to execute.
%
%Sustainability and Integration: Contemporary architectural practices emphasize sustainability and the integration of buildings with their surroundings. Complex facade designs may facilitate innovative approaches to sustainability, such as incorporating natural ventilation and shading strategies.
%
%Expression of Identity: In an era of globalization, architects may be using complex facade designs as a means of expressing regional or cultural identities. This could explain the resurgence of ornamentation and complex patterns.
%
%Human-Centric Design: Architects may be responding to a growing emphasis on user experience and well-being. Complex facades could be designed to engage occupants and create aesthetically pleasing environments.
%
%Economic Prosperity: Economic factors can also influence architectural trends. Periods of economic prosperity often lead to more ambitious and complex building designs, as resources are more readily available for experimentation.
%
%Post-Modernism and Pluralism: The rejection of the rigid principles of Modernism in favor of post-modernism and architectural pluralism has contributed to a more diverse range of design possibilities, including complex facades.
%
%Interdisciplinary Collaboration: Architects are increasingly collaborating with other disciplines, such as artists and engineers. This interdisciplinary approach may encourage more complex and innovative facade designs.
%
%Environmental Considerations: The focus on environmental sustainability may drive architects to explore facade designs that optimize energy efficiency, daylighting, and thermal comfort, often requiring intricate solutions.
%
%Global Connectivity: Architects today have access to a wealth of global architectural influences and references through the internet. This exposure to diverse architectural traditions may encourage experimentation with complex facades.

%% The main conclusions of the study may be presented in a short Conclusions section, which may stand alone or form a subsection of a Discussion or Results and Discussion section.
%% Answer the Research question


This paper explores architectural design at the intersection of digital fabrication, virtual reality assessment, and computer vision to deepen our understanding of intricate facade design.
Our primary goal is to gauge user tolerance and acceptance of complex facades, offering insights into future construction practices.
A literature review confirms a resurgence of complexity in contemporary architecture.
We introduce the `Computational Image Complexity Analysis' (CICA) system, using computer vision to quantitatively analyzing building complexity across epochs, revealing an upward complexity trendline since the late 20th century.

In a virtual reality experiment, we quantify user responses to complex facades, shedding light on their tolerance and acceptance levels.
On average, participants favor complexity, with an average score of 3.82 (out of 10) and a \(40\%\) probability of selecting a score close to this value according to CICA.
Post-survey qualitative scores averaged 4.9 on a 7-point Likert scale, indicating favorable attitudes toward complex facade variations.
These results align with the historical analysis, confirming contemporary architecture's embrace of complexity.

%! conclusions from survey results
 %Q9
 Architects and designers can consider this result as a reflection of the delicate balance required in incorporating patterns and textures into facades.
 While complexity can be visually engaging, it should not overshadow other design elements or create visual clutter.
 A moderate perception of complexity suggests that the design struck a reasonable equilibrium in this regard.

%Q10

Furthermore, a moderate rating implies that the design may have struck a chord with a broad range of participants, appealing to those who appreciate both intricate and simpler ornamentation styles.
 This adaptability in the perception of detail suggests that the design achieved a versatile and inclusive approach to ornamentation.

%Q11

From an architectural standpoint, this finding underscores the importance of materials in creating visually appealing and complex facades.
 The materials used in construction play a crucial role in defining the facade's character and can significantly impact its aesthetic appeal.

%Q12
The participants' feedback on the aesthetic intricacy of the facade compositions aligns with the idea that a balanced and moderate level of complexity in composition can enhance the overall visual appeal of architectural facades.

%Q13
The results indicate that the majority of participants recognized the pivotal role played by the arrangement of architectural elements in enhancing the facade's complexity.

%Q14
While color does contribute to complexity, its role might be less pronounced compared to factors like shape arrangement.Color has a role in creating visually engaging facades, however other design elements may have a more substantial influence on complexity.

%!draft ideas
 Architects may be responding to a growing emphasis on user experience and well-being.
 Complex facades could be designed to engage occupants and create aesthetically pleasing environments.

%!Ideas for interpretations of upwards trajectory of trendline in historical analysis chart graph

%Resurgence of Ornamentation: This trend may reflect a resurgence of interest in architectural ornamentation and intricate facade design. After the stark minimalism of the mid-20th century, architects may be returning to more decorative and visually complex elements in building design.
%
%Advancements in Technology: The last half-century has witnessed significant technological advancements, particularly in computer-aided design and construction techniques. These technological leaps may have empowered architects to explore more complex and daring designs that were previously challenging to execute.
%
%Sustainability and Integration: Contemporary architectural practices emphasize sustainability and the integration of buildings with their surroundings. Complex facade designs may facilitate innovative approaches to sustainability, such as incorporating natural ventilation and shading strategies.
%
%Expression of Identity: In an era of globalization, architects may be using complex facade designs as a means of expressing regional or cultural identities. This could explain the resurgence of ornamentation and complex patterns.
%
%Human-Centric Design: Architects may be responding to a growing emphasis on user experience and well-being. Complex facades could be designed to engage occupants and create aesthetically pleasing environments.
%
%Economic Prosperity: Economic factors can also influence architectural trends. Periods of economic prosperity often lead to more ambitious and complex building designs, as resources are more readily available for experimentation.
%
%Post-Modernism and Pluralism: The rejection of the rigid principles of Modernism in favor of post-modernism and architectural pluralism has contributed to a more diverse range of design possibilities, including complex facades.
%
%Interdisciplinary Collaboration: Architects are increasingly collaborating with other disciplines, such as artists and engineers. This interdisciplinary approach may encourage more complex and innovative facade designs.
%
%Environmental Considerations: The focus on environmental sustainability may drive architects to explore facade designs that optimize energy efficiency, daylighting, and thermal comfort, often requiring intricate solutions.
%
%Global Connectivity: Architects today have access to a wealth of global architectural influences and references through the internet. This exposure to diverse architectural traditions may encourage experimentation with complex facades.