%!%\section{Conclusions}
%%\label{sec:Conclusion}
%%%!%\section{Conclusions}
%%\label{sec:Conclusion}
%%%!%\section{Conclusions}
%%\label{sec:Conclusion}
%%%!%\section{Conclusions}
%%\label{sec:Conclusion}
%%\input{Text/Conclusions}

%% The main conclusions of the study may be presented in a short Conclusions section, which may stand alone or form a subsection of a Discussion or Results and Discussion section.
%% Answer the Research question

%revision

This research investigates architectural design at the intersection of digital fabrication, virtual reality (VR) assessment, and computer vision, aiming to deepen our understanding of complexity in facade design.
Our primary goal is to gauge user tolerance and acceptance of complex facades, offering insights into future construction practices.

A literature review confirmed that contemporary architecture is witnessing a trend towards increasing complexity in facade designs, moving away from the minimalist approach of the modernist movement, a trend also evidenced by the quantitative analysis across architectural history, provided by the `Computational Image Complexity Analysis' (CICA) system, which revealed an upward complexity trendline since the late 20th century (see Figure~\ref{fig:CICAscatterGraphRender}).
The historical analysis using the CICA system underscored the cultural and historical significance of facades, indicating that architectural complexity is not merely a matter of quantitative metrics but also involves cultural resonance and historical context.

Participants in the virtual reality experiment showed a preference for facades with moderate complexity, suggesting that future architectural trends may favor designs that balance intricacy with simplicity.
On average, participants favor a moderate level of complexity, with an average CICA complexity score of 3.82 (out of 10) and a 40\% probability of selecting a score close to this value according to CICA\@.

Discrepancies between participant perceptions and the CICA system's complexity rankings were particularly evident at higher complexity levels.
This highlights the subjective nature of complexity perception and the importance of integrating human feedback into architectural assessments.

The qualitative data suggests a shift towards customizable and user-responsive architectural solutions, with participants favoring form over materials and expressing a preference for facades that consider views and privacy.
This feedback suggests a strategic, view-dependent approach to facade complexity is crucial for user satisfaction.

In conclusion, this study underscores a shift in contemporary architecture towards embracing complexity in facade design, moving beyond the minimalist constraints of the modernist movement.
These insights could inform the development of nuanced and user-centric approaches in architectural design, catering to the evolving demands of modern society.


%% The main conclusions of the study may be presented in a short Conclusions section, which may stand alone or form a subsection of a Discussion or Results and Discussion section.
%% Answer the Research question

%revision

This research investigates architectural design at the intersection of digital fabrication, virtual reality (VR) assessment, and computer vision, aiming to deepen our understanding of complexity in facade design.
Our primary goal is to gauge user tolerance and acceptance of complex facades, offering insights into future construction practices.

A literature review confirmed that contemporary architecture is witnessing a trend towards increasing complexity in facade designs, moving away from the minimalist approach of the modernist movement, a trend also evidenced by the quantitative analysis across architectural history, provided by the `Computational Image Complexity Analysis' (CICA) system, which revealed an upward complexity trendline since the late 20th century (see Figure~\ref{fig:CICAscatterGraphRender}).
The historical analysis using the CICA system underscored the cultural and historical significance of facades, indicating that architectural complexity is not merely a matter of quantitative metrics but also involves cultural resonance and historical context.

Participants in the virtual reality experiment showed a preference for facades with moderate complexity, suggesting that future architectural trends may favor designs that balance intricacy with simplicity.
On average, participants favor a moderate level of complexity, with an average CICA complexity score of 3.82 (out of 10) and a 40\% probability of selecting a score close to this value according to CICA\@.

Discrepancies between participant perceptions and the CICA system's complexity rankings were particularly evident at higher complexity levels.
This highlights the subjective nature of complexity perception and the importance of integrating human feedback into architectural assessments.

The qualitative data suggests a shift towards customizable and user-responsive architectural solutions, with participants favoring form over materials and expressing a preference for facades that consider views and privacy.
This feedback suggests a strategic, view-dependent approach to facade complexity is crucial for user satisfaction.

In conclusion, this study underscores a shift in contemporary architecture towards embracing complexity in facade design, moving beyond the minimalist constraints of the modernist movement.
These insights could inform the development of nuanced and user-centric approaches in architectural design, catering to the evolving demands of modern society.


%% The main conclusions of the study may be presented in a short Conclusions section, which may stand alone or form a subsection of a Discussion or Results and Discussion section.
%% Answer the Research question

%revision

This research investigates architectural design at the intersection of digital fabrication, virtual reality (VR) assessment, and computer vision, aiming to deepen our understanding of complexity in facade design.
Our primary goal is to gauge user tolerance and acceptance of complex facades, offering insights into future construction practices.

A literature review confirmed that contemporary architecture is witnessing a trend towards increasing complexity in facade designs, moving away from the minimalist approach of the modernist movement, a trend also evidenced by the quantitative analysis across architectural history, provided by the `Computational Image Complexity Analysis' (CICA) system, which revealed an upward complexity trendline since the late 20th century (see Figure~\ref{fig:CICAscatterGraphRender}).
The historical analysis using the CICA system underscored the cultural and historical significance of facades, indicating that architectural complexity is not merely a matter of quantitative metrics but also involves cultural resonance and historical context.

Participants in the virtual reality experiment showed a preference for facades with moderate complexity, suggesting that future architectural trends may favor designs that balance intricacy with simplicity.
On average, participants favor a moderate level of complexity, with an average CICA complexity score of 3.82 (out of 10) and a 40\% probability of selecting a score close to this value according to CICA\@.

Discrepancies between participant perceptions and the CICA system's complexity rankings were particularly evident at higher complexity levels.
This highlights the subjective nature of complexity perception and the importance of integrating human feedback into architectural assessments.

The qualitative data suggests a shift towards customizable and user-responsive architectural solutions, with participants favoring form over materials and expressing a preference for facades that consider views and privacy.
This feedback suggests a strategic, view-dependent approach to facade complexity is crucial for user satisfaction.

In conclusion, this study underscores a shift in contemporary architecture towards embracing complexity in facade design, moving beyond the minimalist constraints of the modernist movement.
These insights could inform the development of nuanced and user-centric approaches in architectural design, catering to the evolving demands of modern society.


%% The main conclusions of the study may be presented in a short Conclusions section, which may stand alone or form a subsection of a Discussion or Results and Discussion section.
%% Answer the Research question

%revision

This research investigates architectural design at the intersection of digital fabrication, virtual reality (VR) assessment, and computer vision, aiming to deepen our understanding of complexity in facade design.
Our primary goal is to gauge user tolerance and acceptance of complex facades, offering insights into future construction practices.

A literature review confirmed that contemporary architecture is witnessing a trend towards increasing complexity in facade designs, moving away from the minimalist approach of the modernist movement, a trend also evidenced by the quantitative analysis across architectural history, provided by the 'Computational Image Complexity Analysis' (CICA) system, which revealed an upward complexity trendline since the late 20th century (see Figure \ref{fig:CICAscatterGraphRender}).
The historical analysis using the CICA system underscored the cultural and historical significance of facades, indicating that architectural complexity is not merely a matter of quantitative metrics but also involves cultural resonance and historical context.

Participants in the virtual reality experiment showed a preference for facades with moderate complexity, suggesting that future architectural trends may favor designs that balance intricacy with simplicity.
On average, participants favor a moderate level of complexity, with an average CICA complexity score of 3.82 (out of 10) and a 40\% probability of selecting a score close to this value according to CICA.

Discrepancies between participant perceptions and the CICA system's complexity rankings were particularly evident at higher complexity levels.
This highlights the subjective nature of complexity perception and the importance of integrating human feedback into architectural assessments.

The qualitative data suggests a shift towards customizable and user-responsive architectural solutions, with participants favoring form over materials and expressing a preference for facades that consider views and privacy.
This feedback suggests a strategic, view-dependent approach to facade complexity is crucial for user satisfaction.

In conclusion, this study underscores a shift in contemporary architecture towards embracing complexity in facade design, moving beyond the minimalist constraints of the modernist movement.
These insights could inform the development of nuanced and user-centric approaches in architectural design, catering to the evolving demands of modern society.


%In a VR experiment, we quantify user responses to complex facades, shedding light on their tolerance and acceptance levels.
%On average, participants favor complexity, with an average CICA complexity score of 3.82 (out of 10) and a 40\% probability of selecting a score close to this value according to CICA. Post-survey qualitative scores averaged 4.9 on a 7-point Likert scale, indicating favorable attitudes toward complex facade variations.
%These results align with the historical analysis, confirming contemporary architecture's embrace of complexity.

%The findings from this analysis indicate a preference among participants for facades with moderate complexity, hinting at a future architectural trend that favors a harmonious balance between intricacy and simplicity.
%Such designs are likely to be visually engaging without being overwhelming.
%Additionally, the diversity in participant responses suggests a move towards more customizable and personalized architectural solutions, tailored to meet individual preferences and needs.

%The differences between participant responses and the CICA system are most pronounced in Patterns 1 and 3, suggesting subjective nuances in complexity perception that the CICA system might not capture.
%These insights highlight the importance of integrating subjective human input with objective algorithmic assessments in the architectural design process.

%In post-experiment discussions, participants favored form significantly more than materials in facade design, recommending an 80:20 focus.
%They preferred simpler facades for areas with key views and more complex, privacy-enhancing designs where views were not a priority.
%This feedback suggests a strategic, view-dependent approach to facade complexity is crucial for user satisfaction.

In summary, the findings support the notion that contemporary architecture is embracing complexity in facade design, marking a departure from more simplistic interpretations of architecture associated with the guidelines of the modernist movement.
This shift is evident not only in the quantitative analysis of historical facade complexity trends but also in the results of our experimental phase, where participants displayed a clear inclination towards complexity.
