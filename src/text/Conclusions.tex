%!%\section{Conclusions}
%%\label{sec:Conclusion}
%%%!%\section{Conclusions}
%%\label{sec:Conclusion}
%%%!%\section{Conclusions}
%%\label{sec:Conclusion}
%%%!%\section{Conclusions}
%%\label{sec:Conclusion}
%%\input{Text/Conclusions}

%% The main conclusions of the study may be presented in a short Conclusions section, which may stand alone or form a subsection of a Discussion or Results and Discussion section.
%% Answer the Research question

%revision

This research investigates architectural design at the intersection of digital fabrication, virtual reality (VR) assessment, and computer vision, aiming to deepen our understanding of complexity in facade design.
Our primary goal is to gauge user tolerance and acceptance of complex facades, offering insights into future construction practices.

A literature review confirmed that contemporary architecture is witnessing a trend towards increasing complexity in facade designs, moving away from the minimalist approach of the modernist movement, a trend also evidenced by the quantitative analysis across architectural history, provided by the `Computational Image Complexity Analysis' (CICA) system, which revealed an upward complexity trendline since the late 20th century (see Figure~\ref{fig:CICAscatterGraphRender}).
The historical analysis using the CICA system underscored the cultural and historical significance of facades, indicating that architectural complexity is not merely a matter of quantitative metrics but also involves cultural resonance and historical context.

Participants in the virtual reality experiment showed a preference for facades with moderate complexity, suggesting that future architectural trends may favor designs that balance intricacy with simplicity.
On average, participants favor a moderate level of complexity, with an average CICA complexity score of 3.82 (out of 10) and a 40\% probability of selecting a score close to this value according to CICA\@.

Discrepancies between participant perceptions and the CICA system's complexity rankings were particularly evident at higher complexity levels.
This highlights the subjective nature of complexity perception and the importance of integrating human feedback into architectural assessments.

The qualitative data suggests a shift towards customizable and user-responsive architectural solutions, with participants favoring form over materials and expressing a preference for facades that consider views and privacy.
This feedback suggests a strategic, view-dependent approach to facade complexity is crucial for user satisfaction.

In conclusion, this study underscores a shift in contemporary architecture towards embracing complexity in facade design, moving beyond the minimalist constraints of the modernist movement.
These insights could inform the development of nuanced and user-centric approaches in architectural design, catering to the evolving demands of modern society.


%% The main conclusions of the study may be presented in a short Conclusions section, which may stand alone or form a subsection of a Discussion or Results and Discussion section.
%% Answer the Research question

%revision

This research investigates architectural design at the intersection of digital fabrication, virtual reality (VR) assessment, and computer vision, aiming to deepen our understanding of complexity in facade design.
Our primary goal is to gauge user tolerance and acceptance of complex facades, offering insights into future construction practices.

A literature review confirmed that contemporary architecture is witnessing a trend towards increasing complexity in facade designs, moving away from the minimalist approach of the modernist movement, a trend also evidenced by the quantitative analysis across architectural history, provided by the `Computational Image Complexity Analysis' (CICA) system, which revealed an upward complexity trendline since the late 20th century (see Figure~\ref{fig:CICAscatterGraphRender}).
The historical analysis using the CICA system underscored the cultural and historical significance of facades, indicating that architectural complexity is not merely a matter of quantitative metrics but also involves cultural resonance and historical context.

Participants in the virtual reality experiment showed a preference for facades with moderate complexity, suggesting that future architectural trends may favor designs that balance intricacy with simplicity.
On average, participants favor a moderate level of complexity, with an average CICA complexity score of 3.82 (out of 10) and a 40\% probability of selecting a score close to this value according to CICA\@.

Discrepancies between participant perceptions and the CICA system's complexity rankings were particularly evident at higher complexity levels.
This highlights the subjective nature of complexity perception and the importance of integrating human feedback into architectural assessments.

The qualitative data suggests a shift towards customizable and user-responsive architectural solutions, with participants favoring form over materials and expressing a preference for facades that consider views and privacy.
This feedback suggests a strategic, view-dependent approach to facade complexity is crucial for user satisfaction.

In conclusion, this study underscores a shift in contemporary architecture towards embracing complexity in facade design, moving beyond the minimalist constraints of the modernist movement.
These insights could inform the development of nuanced and user-centric approaches in architectural design, catering to the evolving demands of modern society.


%% The main conclusions of the study may be presented in a short Conclusions section, which may stand alone or form a subsection of a Discussion or Results and Discussion section.
%% Answer the Research question

%revision

This research investigates architectural design at the intersection of digital fabrication, virtual reality (VR) assessment, and computer vision, aiming to deepen our understanding of complexity in facade design.
Our primary goal is to gauge user tolerance and acceptance of complex facades, offering insights into future construction practices.

A literature review confirmed that contemporary architecture is witnessing a trend towards increasing complexity in facade designs, moving away from the minimalist approach of the modernist movement, a trend also evidenced by the quantitative analysis across architectural history, provided by the `Computational Image Complexity Analysis' (CICA) system, which revealed an upward complexity trendline since the late 20th century (see Figure~\ref{fig:CICAscatterGraphRender}).
The historical analysis using the CICA system underscored the cultural and historical significance of facades, indicating that architectural complexity is not merely a matter of quantitative metrics but also involves cultural resonance and historical context.

Participants in the virtual reality experiment showed a preference for facades with moderate complexity, suggesting that future architectural trends may favor designs that balance intricacy with simplicity.
On average, participants favor a moderate level of complexity, with an average CICA complexity score of 3.82 (out of 10) and a 40\% probability of selecting a score close to this value according to CICA\@.

Discrepancies between participant perceptions and the CICA system's complexity rankings were particularly evident at higher complexity levels.
This highlights the subjective nature of complexity perception and the importance of integrating human feedback into architectural assessments.

The qualitative data suggests a shift towards customizable and user-responsive architectural solutions, with participants favoring form over materials and expressing a preference for facades that consider views and privacy.
This feedback suggests a strategic, view-dependent approach to facade complexity is crucial for user satisfaction.

In conclusion, this study underscores a shift in contemporary architecture towards embracing complexity in facade design, moving beyond the minimalist constraints of the modernist movement.
These insights could inform the development of nuanced and user-centric approaches in architectural design, catering to the evolving demands of modern society.


%% The main conclusions of the study may be presented in a short Conclusions section, which may stand alone or form a subsection of a Discussion or Results and Discussion section.
%% Answer the Research question

 Architects may be responding to a growing emphasis on user experience and well-being.
 Complex facades could be designed to engage occupants and create aesthetically pleasing environments.

%!Ideas for interpretations of upwards trajectory of trendline in historical analysis chart graph

%Resurgence of Ornamentation: This trend may reflect a resurgence of interest in architectural ornamentation and intricate facade design. After the stark minimalism of the mid-20th century, architects may be returning to more decorative and visually complex elements in building design.
%
%Advancements in Technology: The last half-century has witnessed significant technological advancements, particularly in computer-aided design and construction techniques. These technological leaps may have empowered architects to explore more complex and daring designs that were previously challenging to execute.
%
%Sustainability and Integration: Contemporary architectural practices emphasize sustainability and the integration of buildings with their surroundings. Complex facade designs may facilitate innovative approaches to sustainability, such as incorporating natural ventilation and shading strategies.
%
%Expression of Identity: In an era of globalization, architects may be using complex facade designs as a means of expressing regional or cultural identities. This could explain the resurgence of ornamentation and complex patterns.
%
%Human-Centric Design: Architects may be responding to a growing emphasis on user experience and well-being. Complex facades could be designed to engage occupants and create aesthetically pleasing environments.
%
%Economic Prosperity: Economic factors can also influence architectural trends. Periods of economic prosperity often lead to more ambitious and complex building designs, as resources are more readily available for experimentation.
%
%Post-Modernism and Pluralism: The rejection of the rigid principles of Modernism in favor of post-modernism and architectural pluralism has contributed to a more diverse range of design possibilities, including complex facades.
%
%Interdisciplinary Collaboration: Architects are increasingly collaborating with other disciplines, such as artists and engineers. This interdisciplinary approach may encourage more complex and innovative facade designs.
%
%Environmental Considerations: The focus on environmental sustainability may drive architects to explore facade designs that optimize energy efficiency, daylighting, and thermal comfort, often requiring intricate solutions.
%
%Global Connectivity: Architects today have access to a wealth of global architectural influences and references through the internet. This exposure to diverse architectural traditions may encourage experimentation with complex facades.