%\section{Results}
%\label{sec:Results}
%% Results should be clear and concise.

%! Text
The experiment was carried out at Kyushu University, Fukuoka, Japan.
The study took place in two timeframes, from October 12 to October 30, 2023, and July 1 to July 12, 2024, with experiments held between 10:00 and 18:00.

\deleted{
 A total of 26 participants, comprising university students and faculty members, engaged in the experiment.
The demographic distribution of the participants is illustrated in Figure~\ref{fig:SurveyBackgroundChart}.
The majority (41\%) were students from various disciplines, while 27\% had a background in construction, and 20\% had prior experience in facade design, as depicted in Figure~\ref{fig:SurveyYearsExperienceChart}.
}

\added{
A total of 26 participants, comprising 13 males and 13 females, engaged in the experiment. The participants' ages ranged from 18 to 31, with 69.2\% of participants aged between 18 and 24, and 30.8\% aged between 25 and 31.
The demographic distribution of the participants is illustrated in Figure~\ref{fig:SurveyBackgroundChart}. The majority (41\%) were students from various disciplines, while 27\% had a background in construction, and 20\% had prior experience in facade design, as illustrated in Figure~\ref{fig:SurveyYearsExperienceChart}.
The participant pool consisted largely of university volunteers, which explains the limited professional experience among participants, as most were students.
}

\textbf{VR to measure user preference for complexity in facade design.}

In the VR Interaction stage, participants engaged with the facade selection task for all three patterns, resulting in 78 experiment sessions.
For each pattern, participants selected the facade variation they found most comfortable based on its perceived complexity level.

%!Complexity level chosen bar chart

The preferred complexity levels from the VR simulation stage were consolidated into the `Facade variatio selection and CICA score Chart,' a bar chart, shown in Figure~\ref{fig:ComplexityLevelChosenChart}.
Analysis of this chart reveals that most participants favored one of the first five facade variations, with only 1/3 of instances selecting options beyond this range.
 \deleted{On average, participants preferred a CICA score of \(Mean = 4.05\) with a standard deviation of \(SD = 1.2\), as determined by the CICA system.}
\added{
The results show a preference for mid-range complexity with a mean CICA score of \(Mean = 4.05\), but the standard deviation of \(SD = 1.2\) highlights significant variability. This deviation indicates that complexity perception is highly subjective and influenced by personal or contextual factors, such as visual tolerance or interaction with the VR environment.
}
Notably, `facade variation 3'  emerged as the most popular choice for all three patterns among the ten variations (see Figure~\ref{fig:modeling_flowchart}).


%!Complexity level chosen probability graph

The `Probability Distribution Graph of Preferred CICA Scores Across Patterns', showcased in Figure~\ref{fig:ProbabilityComplexitylevelChart}, provides a visual representation of the distribution of participant choices.
It accentuates that there is a \(40\%\) probability of the focus group selecting an answer proximate to the calculated complexity score average,\(Mean = 4.05\) with a modest standard deviation of \(SD = 12\%\) in predicting individual data points or outcomes.
 \deleted{This indicates a moderate level of predictability in participant choices based on the CICA system's complexity assessment.}
\added{This indicates a moderate level of predictability in participant choices suggesting that while most selections align near the average CICA score, there is still a notable range in individual preferences, indicating subjective differences in complexity perception.
}

%!Comparison chart of Average Chosen Facade and CICA scores by pattern

The `Comparison chart of Average Chosen Facade and CICA scores by pattern', displayed in Figure~\ref{fig:ComplexityLevelPerPattern}, underscores that the average choice of facade variation for each pattern hovers around the overall average complexity score, \(Mean = 4.05\) and the average choice of facade variation \(Mean = 4.4\), further supporting the alignment between participant preferences and the CICA system's complexity evaluation.

%Preliminary conclusions from quantitative results

The results from these preliminary analysis indicate a preference among participants for facades with moderate complexity, hinting at a future architectural trend that favors a harmonious balance between intricacy and simplicity.
Such designs are likely to be visually engaging without being overwhelming.
\deleted{Additionally, the diversity in participant responses suggests a move towards more customizable and personalized architectural solutions, tailored to meet individual preferences and needs.
}
\added{
Additionally, the observed deviations, though modest, suggest that individual preferences for complexity vary, reinforcing the need for flexible and customizable architectural designs tailored to meet individual preferences and needs.
}

\textbf{Alignment of user perception with CICA system evaluation of complexity}

%!Complexity perception accuracy per pattern with trendlines
The accuracy of the CICA system in assessing facade complexity, compared to participant perceptions, was analyzed in Stage 2 of the experiment, the Screen-Based Ranking Stage.
The results of this comparison are visually represented in the graphs in Figure~\ref{fig:AccuracyPatternMaster}.
These graphs illustrate the alignment between the trendlines of the overall participants' rankings and the CICA system's rankings for all three patterns, with an average standard deviation of 9\% \(SD = 0.9\) in complexity level categorization.

The results reveal varying degrees of accuracy across different patterns:

In Pattern 1, in Figure~\ref{fig:AccuracyPatternMaster}(a), participants' perception of complexity rises gradually and then sharply peaks at facade variation 8, which they rated the highest in terms of complexity.
The CICA system, however, peaks earlier at facade variation 4, suggesting that the system detected a higher level of complexity at an earlier stage than the participants.
The standard deviation \(SD1 = 1.0\) indicates that there was a considerable spread in participant responses, highlighting a divergence in complexity perception between the human participants and the CICA system, especially at higher complexity levels.

For Pattern 2, in Figure~\ref{fig:AccuracyPatternMaster}(b), the participant rankings show a peak at facade variation 9, rated as the most complex, and a near-peak score for facade variation 10.
Conversely, the CICA system also recognizes variation 9's complexity but assigns higher scores to variations 7 and 8 than to variation 10.
This discrepancy suggests that certain design elements in variation 10 might be perceived by users as contributing to complexity more than the CICA system's metrics capture.
The smaller standard deviation \(SD2 = 0.6\) here indicates a closer alignment between participants’ perceptions and the CICA scores, suggesting a more consistent agreement on complexity rankings for this pattern among the participants.

In Pattern 3, as illustrated in Figure~\ref{fig:AccuracyPatternMaster}(c), participant rankings highlight one peak in perceived complexity, with facade variation 9 rated highest and variation 8 closely behind.
However, the CICA system assigns the highest complexity score to variation 7 and ranks variation 5 as the second most complex, diverging significantly from participant rankings for variations 8, 9, and 10.
This mismatch, along with the standard deviation \(SD = 1.1\), similar to Pattern 1, underscores the variability in how participants perceive complexity as opposed to the CICA system, particularly at the upper end of the complexity scale.

The analysis across patterns demonstrates that while the CICA system provides a systematic approach to complexity measurement, it does not always reflect the human perception, particularly at higher complexity variations.
The differences between participant responses and the CICA system are most pronounced in Patterns 1 and 3, suggesting subjective nuances in complexity perception that the CICA system might not capture.
These insights highlight the importance of integrating subjective human input with objective algorithmic assessments in the architectural design process.



