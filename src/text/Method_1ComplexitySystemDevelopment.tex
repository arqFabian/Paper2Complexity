%    \subsection{VR system for Complexity Analysis in facade design}
%    \label{subsec:VRsystemDevelopment}
%    %    \subsection{VR system for Complexity Analysis in facade design}
%    \label{subsec:VRsystemDevelopment}
%    %    \subsection{VR system for Complexity Analysis in facade design}
%    \label{subsec:VRsystemDevelopment}
%    %    \subsection{VR system for Complexity Analysis in facade design}
%    \label{subsec:VRsystemDevelopment}
%    \input{text/Method_1ComplexitySystemDevelopment.tex}VR System development
%! Concise

The `Complexity Analysis' system addresses the challenge of quantifying complexity in architectural facade design, playing a pivotal role in our study.
To achieve this, we developed a process that integrates immersive VR experiences with CV algorithms embedded in the CICA system (see Figure~\ref{fig:MethodologyFlowchartComplexity}, element 3.1).
This approach enables real-time interaction with various facade designs while providing complexity data, offering comprehensive insights into the aesthetic and practical implications of architectural complexity.

The system comprises three integral components: `\textit{3D Modeling and Environment Setup}', `\textit{CICA System}', and `\textit{VR Integration and Simulation Tools}'.
These components are illustrated in Figure~\ref{fig:MethodologyFlowchartComplexity} (labeled 1 to 3) and are detailed in the following sections.
VR System development
%! Concise

The `Complexity Analysis' system addresses the challenge of quantifying complexity in architectural facade design, playing a pivotal role in our study.
To achieve this, we developed a process that integrates immersive VR experiences with CV algorithms embedded in the CICA system (see Figure~\ref{fig:MethodologyFlowchartComplexity}, element 3.1).
This approach enables real-time interaction with various facade designs while providing complexity data, offering comprehensive insights into the aesthetic and practical implications of architectural complexity.

The system comprises three integral components: `\textit{3D Modeling and Environment Setup}', `\textit{CICA System}', and `\textit{VR Integration and Simulation Tools}'.
These components are illustrated in Figure~\ref{fig:MethodologyFlowchartComplexity} (labeled 1 to 3) and are detailed in the following sections.
VR System development
%! Concise

The `Complexity Analysis' system addresses the challenge of quantifying complexity in architectural facade design, playing a pivotal role in our study.
To achieve this, we developed a process that integrates immersive VR experiences with CV algorithms embedded in the CICA system (see Figure~\ref{fig:MethodologyFlowchartComplexity}, element 3.1).
This approach enables real-time interaction with various facade designs while providing complexity data, offering comprehensive insights into the aesthetic and practical implications of architectural complexity.

The system comprises three integral components: `\textit{3D Modeling and Environment Setup}', `\textit{CICA System}', and `\textit{VR Integration and Simulation Tools}'.
These components are illustrated in Figure~\ref{fig:MethodologyFlowchartComplexity} (labeled 1 to 3) and are detailed in the following sections.
VR System development
%! Concise

The `Complexity Analysis' system addresses the challenge of quantifying complexity in architectural facade design, playing a pivotal role in our study.
It aims to measure and understand the impact of facade complexity on user perception and architectural aesthetics.
To achieve this, we developed a process that integrates immersive VR experiences with CV algorithms embedded in the CICA system (see Figure~\ref{fig:MethodologyFlowchart}, element 3.1).
This approach enables real-time interaction with various facade designs while providing complexity data, offering comprehensive insights into the aesthetic and practical implications of architectural complexity.
By bridging the gap between theoretical analysis and practical application, this system provides a valuable tool for quantifying complexity and has the potential for optimizing facade designs.

The system comprises three integral components: `\textit{3D Modeling and Environment Setup}', `\textit{CICA System}', and `\textit{VR Integration and Simulation Tools}'.
These components are illustrated in Figure~\ref{fig:MethodologyFlowchart} (labeled 1 to 3) and are detailed in the following sections.

%!Original
%begin{enumerate}
%\item \textit{3D Modeling and Environment Setup:} Detailed in Section~\ref{subsubsec:3DModeling} (element 1 in Figure~\ref{fig:MethodologyFlowchart}), this component is responsible for creating realistic 3D representations of the experiment site and ten distinct facade variations across three patterns for user interaction.
%
%\item \textit{CICA System:} Detailed in Section~\ref{subsubsec:CICAsystem} (Element 2 in Figure~\ref{fig:MethodologyFlowchart}), this system generates complexity scores for 3D-modeled facades and photographs of historic and existing buildings.
%This systematic approach evaluates and selects facade designs for the VR experience and conducts a historical complexity analysis of buildings across various styles.
%
%\item \textit{VR Integration and Simulation Tools:} Detailed in Section~\ref{subsubsec:VR_integration} (element 3 in Figure~\ref{fig:MethodologyFlowchart}), this component offers an immersive simulation with a data visualization interface.
%Participants can navigate the virtual space, experiencing the building and its surroundings, and interact with various facade designs.
%Real-time feedback enhances this interaction by showing the impact of different facade variations.
%\end{enumerate}