%%Methodology
%% methodology Intro
The methodology of this study, comprises three main components (as illustrated in  Figure~\ref{fig:MethodologyFlowchart}):

\begin{itemize}
    \item Building Complexity Analysis System Development
    \item Experiment Execution
    \item Data Analysis and Validation
\end{itemize}


\textbf{Building Complexity Analysis (BCA) System Development:}  Outlined in Section~\ref{subsec:VRsystemDevelopment} and represented as (3.1) in Figure~\ref{fig:MethodologyFlowchart}, this component integrates the VR and CICA systems to create an immersive environment for evaluating facade complexities.
Users can interact with variable facade designs in VR while the CICA system uses computer vision algorithms to generate complexity scores, providing quantitative assessments of the designs for DBD\@.

\textbf{Experiment Execution:} Detailed in Section~\ref{subsec:Experiment_execution} and illustrated as (3.2) in Figure~\ref{fig:MethodologyFlowchart}, involves the deployment of the VR integration of the BCA system in a controlled scenario to evaluate users' perception and response to different complexity levels in facade designs.

\textbf{Data Analysis and Validation:} Detailed in Section~\ref{subsec:Data_analysis} and illustrated as (3.3) in Figure~\ref{fig:MethodologyFlowchart}, this component analyzes the collected data to validate the VR system and CICA framework's effectiveness in measuring complexity and
users' acceptance of building complexity.


these components are combined to integrate computational analysis with immersive VR experiments, exploring user preferences in facade design and providing insights into architectural trends.

With this methodology outlined, the following sections will now delve into a comprehensive breakdown of each component, highlighting their objectives, methodologies, and significance in achieving our research goals.


%\textbf{CICA system:} Detailed in Section~\ref{subsec:CICAsystem} and depicted as (3.1) in Figure~\ref{fig:MethodologyFlowchart}, involves developing a Python script using computer vision algorithms to process images and yield complexity scores, quantitatively gauging the intricacy of architectural designs.
%It supports theoretical analyses and provides a ranking framework for facade designs in the VR system.
%
%\textbf{VR System Development:}  Outlined in Section~\ref{subsec:VRsystemDevelopment} and represented as (3.2) in Figure~\ref{fig:MethodologyFlowchart}, this component focuses on creating an immersive experience where participants can explore and interact with a building's interior and exterior, manipulating facade designs with variable complexity levels and experiencing their impact.
%
%\textbf{Experiment Execution:} Detailed in Section~\ref{subsec:Experiment_execution} and illustrated as (3.3) in Figure~\ref{fig:MethodologyFlowchart}, following a similar approach to previous studies~\cite{Wolfartsberger2019}, this component outlines the method to evaluate users' acceptance of building complexity.
%It consists of three stages: a VR interaction stage, a screen-based stage, and a post-interaction survey.
%These stages allow for the collection of quantitative and qualitative data on participants' perceptions and responses to different complexity levels in facade designs.
%
%As illustrated in the methodology flowchart (Figure~\ref{fig:MethodologyFlowchart}), these components are combined to integrate computational analysis with immersive VR experiments, exploring user preferences in facade design and providing insights into architectural trends.
%
%With this methodology outlined, the following sections will now delve into a comprehensive breakdown of each component, highlighting their objectives, methodologies, and significance in achieving our research goals.