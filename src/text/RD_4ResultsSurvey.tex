%!\subsection{Qualitative Analysis on Users Perception to Complex Facades}
%    \label{subsec:ResultsSurvey}
%    %!\section{Discussion}
%\label{sec:Discussion}
%%% This should explore the significance of the results of the work, not repeat them. A combined Results and Discussion section is often appropriate. Avoid extensive citations and discussion of published literature.
%%%!\section{Discussion}
%\label{sec:Discussion}
%%% This should explore the significance of the results of the work, not repeat them. A combined Results and Discussion section is often appropriate. Avoid extensive citations and discussion of published literature.
%%%!\section{Discussion}
%\label{sec:Discussion}
%%% This should explore the significance of the results of the work, not repeat them. A combined Results and Discussion section is often appropriate. Avoid extensive citations and discussion of published literature.
%%\input{Text/Discussion}


%!% Summary: A brief recap of your key results

Having laid out the results of our comprehensive study, encompassing the intricate analysis of facade complexity throughout history and the user responses to complex facades in contemporary design, we now transition to the discussion phase.

In this section, we interpret our findings, consider their implications, and reflect on their significance in the broader context of architectural trends and design preferences.
This discussion aims to shed light on the evolving relationship between users and the complexity of building facades, offering valuable insights for future construction practices and architectural design.

The multifaceted nature of our data, derived from both historical analysis and user interaction, positions us to answer the core research question of this study: ``What degree of complexity within facade design would users tolerate and accept for a building, and what insights do their preferences provide for future architectural trends?''

%!% Interpretations: What do your results mean?

    %! Historical Analysis graph
    %Discussion on results of history analysis and how theory aligns with the computational image analysis.

Our initial hypothesis, grounded in a rigorous examination of architectural styles from a theoretical standpoint, posited that the evolution of architecture comprises a dynamic interplay between simplicity and complexity (see Section\ref{subsec:TimelineArchitectureStyles}).
We anticipated that in the post-modernist era, contemporary architecture would experience a resurgence of complexity and ornamentation, exemplified by the emergence of five prominent architectural styles: Deconstructivism, Neofuturism, High-tech modernism, Parametricism, and Pragmatic utopianism(see Figure\ref{fig:contemporarytimeline}).

These styles, we contended, were shaped by advancements in technology, the proliferation of computer-aided design tools, and an overarching commitment to sustainability, collectively steering architectural design toward a trajectory characterized by heightened complexity.

Establishing the fundamental principle of architectural complexity is undeniably a complex undertaking, one that is deeply intertwined with numerous socio-economic factors shaping urban development.
Nonetheless, the development of the `Computational Image Complexity Analysis' (CICA) system was underpinned by a visionary premise: that the accumulation of a vast and diverse dataset, spanning architectural creations from centuries past to the contemporary era, could unveil discernible patterns.

These patterns, we hypothesized, would validate our initial proposition concerning the evolutionary essence of architecture - an enduring dialogue between simplicity and complexity, indicating an inexorable shift toward increased complexity in contemporary architecture.

In its quest to do so, the CICA system transcends disparities in building scale, geographic locations, and socio-economic drivers shaping urban development.
Instead, it seeks to reveal an overarching pattern that substantiates the resurgence of architectural complexity within the present epoch.

In accordance with our expectations, our analysis of building complexity throughout history, employing the CICA system, and depicted in the `Historical Complexity Analysis' Chart (Figure\ref{fig:HistoricalComplexityGraph}), meticulously organized by architectural style and year, was complemented by the inclusion of a polynomial trendline.
We opted for this specific trendline due to its versatility in accommodating the intricate data patterns that naturally emerge when assessing historical building complexity scores.
The outcome of this choice was the emergence of a curve characterized by an intriguing cyclic pattern.

This distinctive pattern, akin to an undulating curve traversing centuries, uncovers a continual oscillation between architectural complexity and simplicity.
It resembles the paradigmatic shifts in architectural design discussed within our theoretical analysis (see section\ref{subsec:FacadeandOrnament}), a phenomenon that has endured throughout architectural history.

The dynamic oscillation of the polynomial curve, illustrated in the `Historical Complexity Analysis' Chart in Figure\ref{fig:HistoricalComplexityGraph}, can be interpreted as peaks of ornamental richness giving way to troughs of minimalist restraint, relative to the standards of each era since each historical period has witnessed its unique interpretation of architectural complexity.

Furthermore, our analysis of the last 50 years of data within the trendline curve conspicuously illustrates an upward trajectory in architectural complexity.
This trend signifies a departure from the stark minimalism that characterized the decline of the Modernist movement between the 1950s and 1960s, thereby affirming our initial hypothesis.

These findings not only provide empirical validation of our hypothesis but also emphasize the cyclical nature of architectural complexity, reflecting the continuous dialogues within architectural practice across the ages.

    %! Experiment results discussion
    %Discussion on how the experiment results show a preference or positive take on complexity.



        %Consistent with the hypothesis, a discernible pattern emerges in the accuracy graphs (see Figure \ref{fig:Accuracyscattergraph}) when transitioning from a "screen-based" approach to a "VR-based" approach. The majority of users demonstrate increased proximity to the best-recommended solution provided by the system.

        %These results suggest a deeper understanding of the site and a heightened trust in the system. These findings align with the survey responses, indicating that users perceive the system's recommendations as highly valuable and express a strong likelihood of implementing them (see Figures \ref{fig:UsabilityChartSurvey} and \ref{fig:PerceptionSatisfactionSurveyResults}).

        %Moreover, similar studies, such as AstanehAsl et al. (2022) \cite{AstanehAsl2022}, which compared screen-based approaches with VR in resolving building system conflicts, also reported a greater comprehension of technical issues in immersive VR compared to screen-based approaches.

        %However, it is important to note that the wide standard deviation (\(SD = 44.1\%\)) observed in the analysis of VR system improvements can be attributed to the individual instances. These deviations encompass both substantial improvements of over  \(95\%\) in at least 12 sessions and negative results in 5 sessions, with one test subject experiencing a decline of up to \(-48\%\).

        %These individual variations highlight the potential and current limitations of the selected VR system, emphasizing the need for further refinement of the data visualization interface to enhance the understanding of system recommendations.

        %Effective communication plays a pivotal role in bridging the gap between large amounts of data and user perception, ensuring that instances of decline in identifying the best location for SLP are minimized or eliminated altogether.

        %As suggested by Marsden et al. \cite{Marsden2008}, to create a user-centered design for VR, an integrated approach is needed, which balances the influences of MR technology with SE (Software Engineering) and UE (User Experience) considerations.

        %Such an approach considers the technology-driven procedures of Mixed Reality development, like is the case of VR while maintaining the systematic, controllable, and manageable processes advocated by SE and integrating appropriate methods and procedures from UE to develop usable solutions that are applicable in practice.

        %Upon closer examination of the sites individually, a clear pattern emerges: the more heterogeneous the topography, the easier it is to identify suitable locations (see Figure \ref{fig:ImprovementPerSite}). This observation is supported by the lowest rate of improvement in Site 1, which had a relatively homogeneous terrain compared to Site 2, where the most pronounced level variations occurred (see Table \ref{tab:SiteParametersAndPreview}).

        %It appears that when dealing with topography, complexity is preferred, as it accentuates the differences between different locations, thereby expediting the decision-making process in filtering out undesired options. This principle, which is primarily based on visual cues from the simulation, can be translated into data visualization techniques, where even highlighting differences in homogeneous topography could result in an overall improvement in decision-making

%! Historical Analysis graph

%Followed by a tuning of the level of complexity how we interpret the survey results to propose a graphical conclusion of an apparent complex facade inside the boundaries of tolerance given by the experiment participants.



%!% Implications: Why do your results matter?
%% Your overall aim is to show the reader exactly what your research has contributed, and why they should care.


        %And while previous research has focused on small details \cite{AlSaggaf2021}, these results demonstrate that an overview of the whole scale of the project can also be impacted by the integration of VR techniques in a positive matter and perhaps show a larger impact that may tilt the preferential adoption of VR when solving this issue.

%!% limitations: What can’t your results tell us?

        %The findings of this research provide valuable insights into the potential of VR immersion in data-driven design for Site Layout Planning, as well as potential implications for other areas of the building design process. However, it is important to interpret the results with caution due to certain limitations. The generalizability of the findings is limited by the small sample size and the fact that the participants were limited to individuals affiliated with the university.

        %Furthermore, the experiment design aimed to simulate the current methodology used in addressing SLP through a "screen-based interaction" stage, which involved CAD plans, perspectives, and expert recommendations typically provided to design teams.

        %However, the introduction of data visualization techniques and charts was exclusive to the "VR stage" (see Figure \ref{fig:VRinterface}). As a result, it is challenging to determine the exact contribution of VR-based interaction versus more efficient data visualization techniques to the observed accuracy improvement. It can be speculated that if the same graphs and interface used in the VR stage were introduced to the screen-based stage, the observed accuracy improvement may have been different.

%!% recommendations: Avenues for further studies or analyses
%% Based on the discussion of your results, you can make recommendations for practical implementation or further research. Sometimes, the recommendations are saved for the conclusion. Suggestions for further research can lead directly from the limitations. Don’t just state that more studies should be done—give concrete ideas for how future work can build on areas that your own research was unable to address.

        %The limitations identified in this research present valuable opportunities for further investigation. Future studies can delve into the specific impact of VR immersion on data visualization techniques, aiming to understand how VR influences the adoption of data-driven design methods. To achieve this, an experiment can be designed to provide the same interface for both screen-based and VR interactions, facilitating a comprehensive analysis of the effects of VR on data visualization.

        %Moreover, it is essential for this experiment to explore innovative ways of presenting data, pushing the boundaries of traditional 2D constraints. The true potential of VR can be fully realized when data visualization techniques transcend the limitations of physical reality. By projecting data as spatial representations that generate visual feedback and alter the perception of reality, VR has the capacity to eliminate the constraints of traditional design approaches.

        %By embracing these possibilities, future research can unlock the true potential of VR in data-driven design. This can lead to advancements in how data is visualized, enhancing decision-making processes and paving the way for more immersive and interactive experiences.




%!% Summary: A brief recap of your key results

Having laid out the results of our comprehensive study, encompassing the intricate analysis of facade complexity throughout history and the user responses to complex facades in contemporary design, we now transition to the discussion phase.

In this section, we interpret our findings, consider their implications, and reflect on their significance in the broader context of architectural trends and design preferences.
This discussion aims to shed light on the evolving relationship between users and the complexity of building facades, offering valuable insights for future construction practices and architectural design.

The multifaceted nature of our data, derived from both historical analysis and user interaction, positions us to answer the core research question of this study: ``What degree of complexity within facade design would users tolerate and accept for a building, and what insights do their preferences provide for future architectural trends?''

%!% Interpretations: What do your results mean?

    %! Historical Analysis graph
    %Discussion on results of history analysis and how theory aligns with the computational image analysis.

Our initial hypothesis, grounded in a rigorous examination of architectural styles from a theoretical standpoint, posited that the evolution of architecture comprises a dynamic interplay between simplicity and complexity (see Section\ref{subsec:TimelineArchitectureStyles}).
We anticipated that in the post-modernist era, contemporary architecture would experience a resurgence of complexity and ornamentation, exemplified by the emergence of five prominent architectural styles: Deconstructivism, Neofuturism, High-tech modernism, Parametricism, and Pragmatic utopianism(see Figure\ref{fig:contemporarytimeline}).

These styles, we contended, were shaped by advancements in technology, the proliferation of computer-aided design tools, and an overarching commitment to sustainability, collectively steering architectural design toward a trajectory characterized by heightened complexity.

Establishing the fundamental principle of architectural complexity is undeniably a complex undertaking, one that is deeply intertwined with numerous socio-economic factors shaping urban development.
Nonetheless, the development of the `Computational Image Complexity Analysis' (CICA) system was underpinned by a visionary premise: that the accumulation of a vast and diverse dataset, spanning architectural creations from centuries past to the contemporary era, could unveil discernible patterns.

These patterns, we hypothesized, would validate our initial proposition concerning the evolutionary essence of architecture - an enduring dialogue between simplicity and complexity, indicating an inexorable shift toward increased complexity in contemporary architecture.

In its quest to do so, the CICA system transcends disparities in building scale, geographic locations, and socio-economic drivers shaping urban development.
Instead, it seeks to reveal an overarching pattern that substantiates the resurgence of architectural complexity within the present epoch.

In accordance with our expectations, our analysis of building complexity throughout history, employing the CICA system, and depicted in the `Historical Complexity Analysis' Chart (Figure\ref{fig:HistoricalComplexityGraph}), meticulously organized by architectural style and year, was complemented by the inclusion of a polynomial trendline.
We opted for this specific trendline due to its versatility in accommodating the intricate data patterns that naturally emerge when assessing historical building complexity scores.
The outcome of this choice was the emergence of a curve characterized by an intriguing cyclic pattern.

This distinctive pattern, akin to an undulating curve traversing centuries, uncovers a continual oscillation between architectural complexity and simplicity.
It resembles the paradigmatic shifts in architectural design discussed within our theoretical analysis (see section\ref{subsec:FacadeandOrnament}), a phenomenon that has endured throughout architectural history.

The dynamic oscillation of the polynomial curve, illustrated in the `Historical Complexity Analysis' Chart in Figure\ref{fig:HistoricalComplexityGraph}, can be interpreted as peaks of ornamental richness giving way to troughs of minimalist restraint, relative to the standards of each era since each historical period has witnessed its unique interpretation of architectural complexity.

Furthermore, our analysis of the last 50 years of data within the trendline curve conspicuously illustrates an upward trajectory in architectural complexity.
This trend signifies a departure from the stark minimalism that characterized the decline of the Modernist movement between the 1950s and 1960s, thereby affirming our initial hypothesis.

These findings not only provide empirical validation of our hypothesis but also emphasize the cyclical nature of architectural complexity, reflecting the continuous dialogues within architectural practice across the ages.

    %! Experiment results discussion
    %Discussion on how the experiment results show a preference or positive take on complexity.



        %Consistent with the hypothesis, a discernible pattern emerges in the accuracy graphs (see Figure \ref{fig:Accuracyscattergraph}) when transitioning from a "screen-based" approach to a "VR-based" approach. The majority of users demonstrate increased proximity to the best-recommended solution provided by the system.

        %These results suggest a deeper understanding of the site and a heightened trust in the system. These findings align with the survey responses, indicating that users perceive the system's recommendations as highly valuable and express a strong likelihood of implementing them (see Figures \ref{fig:UsabilityChartSurvey} and \ref{fig:PerceptionSatisfactionSurveyResults}).

        %Moreover, similar studies, such as AstanehAsl et al. (2022) \cite{AstanehAsl2022}, which compared screen-based approaches with VR in resolving building system conflicts, also reported a greater comprehension of technical issues in immersive VR compared to screen-based approaches.

        %However, it is important to note that the wide standard deviation (\(SD = 44.1\%\)) observed in the analysis of VR system improvements can be attributed to the individual instances. These deviations encompass both substantial improvements of over  \(95\%\) in at least 12 sessions and negative results in 5 sessions, with one test subject experiencing a decline of up to \(-48\%\).

        %These individual variations highlight the potential and current limitations of the selected VR system, emphasizing the need for further refinement of the data visualization interface to enhance the understanding of system recommendations.

        %Effective communication plays a pivotal role in bridging the gap between large amounts of data and user perception, ensuring that instances of decline in identifying the best location for SLP are minimized or eliminated altogether.

        %As suggested by Marsden et al. \cite{Marsden2008}, to create a user-centered design for VR, an integrated approach is needed, which balances the influences of MR technology with SE (Software Engineering) and UE (User Experience) considerations.

        %Such an approach considers the technology-driven procedures of Mixed Reality development, like is the case of VR while maintaining the systematic, controllable, and manageable processes advocated by SE and integrating appropriate methods and procedures from UE to develop usable solutions that are applicable in practice.

        %Upon closer examination of the sites individually, a clear pattern emerges: the more heterogeneous the topography, the easier it is to identify suitable locations (see Figure \ref{fig:ImprovementPerSite}). This observation is supported by the lowest rate of improvement in Site 1, which had a relatively homogeneous terrain compared to Site 2, where the most pronounced level variations occurred (see Table \ref{tab:SiteParametersAndPreview}).

        %It appears that when dealing with topography, complexity is preferred, as it accentuates the differences between different locations, thereby expediting the decision-making process in filtering out undesired options. This principle, which is primarily based on visual cues from the simulation, can be translated into data visualization techniques, where even highlighting differences in homogeneous topography could result in an overall improvement in decision-making

%! Historical Analysis graph

%Followed by a tuning of the level of complexity how we interpret the survey results to propose a graphical conclusion of an apparent complex facade inside the boundaries of tolerance given by the experiment participants.



%!% Implications: Why do your results matter?
%% Your overall aim is to show the reader exactly what your research has contributed, and why they should care.


        %And while previous research has focused on small details \cite{AlSaggaf2021}, these results demonstrate that an overview of the whole scale of the project can also be impacted by the integration of VR techniques in a positive matter and perhaps show a larger impact that may tilt the preferential adoption of VR when solving this issue.

%!% limitations: What can’t your results tell us?

        %The findings of this research provide valuable insights into the potential of VR immersion in data-driven design for Site Layout Planning, as well as potential implications for other areas of the building design process. However, it is important to interpret the results with caution due to certain limitations. The generalizability of the findings is limited by the small sample size and the fact that the participants were limited to individuals affiliated with the university.

        %Furthermore, the experiment design aimed to simulate the current methodology used in addressing SLP through a "screen-based interaction" stage, which involved CAD plans, perspectives, and expert recommendations typically provided to design teams.

        %However, the introduction of data visualization techniques and charts was exclusive to the "VR stage" (see Figure \ref{fig:VRinterface}). As a result, it is challenging to determine the exact contribution of VR-based interaction versus more efficient data visualization techniques to the observed accuracy improvement. It can be speculated that if the same graphs and interface used in the VR stage were introduced to the screen-based stage, the observed accuracy improvement may have been different.

%!% recommendations: Avenues for further studies or analyses
%% Based on the discussion of your results, you can make recommendations for practical implementation or further research. Sometimes, the recommendations are saved for the conclusion. Suggestions for further research can lead directly from the limitations. Don’t just state that more studies should be done—give concrete ideas for how future work can build on areas that your own research was unable to address.

        %The limitations identified in this research present valuable opportunities for further investigation. Future studies can delve into the specific impact of VR immersion on data visualization techniques, aiming to understand how VR influences the adoption of data-driven design methods. To achieve this, an experiment can be designed to provide the same interface for both screen-based and VR interactions, facilitating a comprehensive analysis of the effects of VR on data visualization.

        %Moreover, it is essential for this experiment to explore innovative ways of presenting data, pushing the boundaries of traditional 2D constraints. The true potential of VR can be fully realized when data visualization techniques transcend the limitations of physical reality. By projecting data as spatial representations that generate visual feedback and alter the perception of reality, VR has the capacity to eliminate the constraints of traditional design approaches.

        %By embracing these possibilities, future research can unlock the true potential of VR in data-driven design. This can lead to advancements in how data is visualized, enhancing decision-making processes and paving the way for more immersive and interactive experiences.



%\textbf{Qualitative analysis on user perception of complexity}
%! Post experiment survey results

We gathered additional insights through a survey and interviews, presenting a multifaceted view of user perceptions regarding architectural complexity.
The responses to the `complexity perception' section of the survey have been summarized in Figure\ref{fig:SurveyQuestions6-10} and Figure\ref{fig:SurveyQuestions11-15}, with evaluations conducted using a 7-point Likert scale.


Survey responses show a moderate to high endorsement of complexity in facade designs, with average ratings above \(3.5\) and mean scores around \(Mean = 4.9\).
A standard deviation of \(SD = 1.5\), however, reflects a range of views among participants.
These variations in responses underscore the diverse perspectives and preferences among participants, enriching our understanding of how users perceive facade complexity.

\textbf{User-Centric Design Preferences: Survey Insights and Post-Experiment Reflections}

Key takeaways from the survey include:

%! Detailed analysis of survey questions

%Q6 To what extent do you find the overall complexity of this facade design appealing?
\textit{Q6- Appeal of Complexity:}
Participants rated the appeal of facade complexity positively , with an average score indicating moderate to high appeal \(Mean=4.3\).
This suggests that complex facade designs have the potential to attract and satisfy user preferences, this finding aligns with the broader architectural discourse, suggesting a growing acceptance of architectural complexity in contemporary design and highlights the potential for the integration of such designs in future architectural practices (Question 6 in Figure \ref{fig:SurveyQuestions6-10}).

%Q7.How do you rate the intricacy of the patterns and textures used in this facade design?
\textit{Q7-  Intricacy of Patterns and Textures:}
The survey revealed that the intricacy of patterns and textures used in the facade designs was well-received, scoring above the midpoint on the Likert scale \(Mean=4.6\).
This indicates that such design elements are important for user satisfaction and can significantly contribute to the visual engagement of a building.
(Question 7 in Figure \ref{fig:SurveyQuestions6-10}).
It encourages the exploration and incorporation of such elements into future architectural projects, knowing that they are likely to resonate with users and contribute positively to the overall appeal of the design.

%Q8. To what extent do you think the arrangement of architectural elements on this facade adds to its visual interest?
\textit{Q8 - Architectural Element Arrangement:}
Participants highly rated the contribution of architectural element arrangement to the facade's visual interest \(Mean=5.5\).
This suggests that how elements are composed can greatly impact a facade's appeal, hinting at the importance of thoughtful design in creating engaging architectural experiences (Question 8 in Figure \ref{fig:SurveyQuestions6-10}).

%Q9. How complex do you perceive the facade's use of patterns and textures?
\textit{Q9 - Perception of Facade's Pattern and Texture Complexity:}
Participants perceived the complexity of patterns and textures as moderate \(Mean=3.9\).
This reflects a balanced approach to facade design, where complexity is appreciated but not overwhelming, pointing to the importance of finding the right level of complexity that resonates with users (Question 9 in Figure \ref{fig:SurveyQuestions6-10}).

%Q10. How detailed do you find the ornamentation on this facade design?
\textit{Q10 - Detail in Ornamentation:}
The detail in ornamentation received a score that indicates users found it moderately detailed \(Mean=5.0\).
This suggests a user preference for ornamentation that contributes to the visual richness of a facade without dominating the design (Question 10 in Figure \ref{fig:SurveyQuestions6-10}).

%Q11. How much do the combination of materials contribute to the overall complexity of the facade?
\textit{Q11 - Material Combination Contribution to Complexity:}
The survey participants regarded the combination of materials as an important factor contributing to facade complexity\(Mean=5.0\).
This finding underscores the importance of materials in creating visually appealing and complex facades, highlighting how materials play a crucial role in defining the facade's character and can significantly impact its aesthetic appeal.(Question 11 in Figure \ref{fig:SurveyQuestions11-15}).

%Q12. To what degree does the composition of the facade strike you as aesthetically intricate?
\textit{Q12 - Composition's Aesthetic Intricacy:}
The aesthetic intricacy of the composition was rated as moderately high \(Mean=4.9\), emphasizing that users value the thoughtful arrangement of design elements that contribute to a facade's overall aesthetic complexity(Question 12 in Figure \ref{fig:SurveyQuestions11-15}). This underscores the notion that a well-calibrated and moderately complex composition can augment the overall visual appeal of architectural facades.

%Q13. How much do you believe that the arrangement of shapes and forms on the facade contributes to its complexity?
\textit{Q13 - Contribution of Shapes and Forms to Complexity:}
This question scored highly \(Mean=6.3\), indicating that users place a significant value on the role of shapes and forms in adding to a facade's complexity.
This highlights the need for architects to consider the geometric aspects of design when aiming to create complex facades (Question 13 in Figure \ref{fig:SurveyQuestions11-15}). It solidifies the notion that the strategic placement of visual elements holds substantial sway over how a facade is perceived, highlighting its essential role in architectural design endeavors aimed at achieving complexity.

%Q14. How significantly does the use of color enhance the facade's visual complexity?
\textit{Q14 - Enhancement of Complexity by Color:}
The use of color was considered to moderately enhance the facade's visual complexity \(Mean=5.1\).
While not as impactful as form or texture, color is still an important design tool that can subtly influence the perception of complexity(Question 14 in Figure \ref{fig:SurveyQuestions11-15}).
This insight is valuable for architects and designers, indicating that while color is a consideration in creating visually engaging facades, other design elements may have a more substantial influence on complexity.

%Q15. How much depth and layering do you observe in the design of this facade?
\textit{Q15 - Observation of Depth and Layering:}
Depth and layering in design were perceived as contributing moderately to facade complexity\(Mean=4.3\).
While not rated as highly significant as other factors like the arrangement of shapes and forms, it still contributes to the overall perception of complexity.
This implies that three-dimensionality and the interplay of different design layers can enhance the perception of complexity and should be considered in facade design (Question 15 in Figure \ref{fig:SurveyQuestions11-15}).

From the insights from the questions, we can see a pattern where participants appreciate complexity that is intelligently integrated into design through form, texture, and color, but still desire a certain level of clarity and not be overwhelmed by excessive details.
These insights can guide architects and designers in creating facades that are complex yet coherent, appealing to a broad spectrum of users.

\textit{Post-experiment interview:}
In post-experiment interviews, participants articulated a clear preference for the role of form in facade design, assigning it significantly more importance than materials at an 80:20 ratio.
This perspective underlines form as a dominant influence in their assessment of facade complexity and aesthetic value.

When discussing intricate facades, there was a shared viewpoint that the design should complement and enhance the surrounding views.
Participants preferred simpler facades for areas with significant vistas, such as the front view of the campus, to ensure these views were unobstructed and appreciated.

Conversely, for building aspects with less prominent views, like those facing adjacent structures, there was a preference for more complex designs.
Participants felt that such designs could effectively balance the need for privacy with an enhanced aesthetic presence.
This insight suggests that intricate facade designs are seen as beneficial when they serve specific purposes, such as enhancing privacy or contributing to the building’s visual interest, rather than being applied uniformly.

These interviews reveal a nuanced understanding of facade design among participants, highlighting the need for context-sensitive approaches that align architectural form and complexity with the functional and aesthetic needs of building users.



%!% Implications: Why do your results matter?
%% Your overall aim is to show the reader exactly what your research has contributed, and why they should care.


        %And while previous research has focused on small details \cite{AlSaggaf2021}, these results demonstrate that an overview of the whole scale of the project can also be impacted by the integration of VR techniques in a positive matter and perhaps show a larger impact that may tilt the preferential adoption of VR when solving this issue.








%! survey results

We gathered additional insights through a survey and interviews, presenting a multifaceted view of user perceptions regarding architectural complexity.
The responses to the `complexity perception' section of the survey have been summarized in Figure~\ref{fig:SurveyQuestions6-10} and Figure~\ref{fig:SurveyQuestions11-15}, with evaluations conducted using a 7-point Likert scale.

Survey responses show a moderate to high endorsement of complexity in facade designs, with average ratings above \(3.5\) and mean scores around \(Mean = 5.2\).
However, the standard deviation of \(SD = 1.3\) reveals considerable variations in participant responses, underscoring the subjective nature of architectural complexity.
These deviations reflect the diverse perspectives and preferences among participants, which is significant in understanding that the perception of facade complexity is not uniform across users.
This range of responses suggests that while some participants are more attuned to intricate designs, others may prefer simpler structures, thus reinforcing the need for adaptable design approaches in architecture.

\textbf{Survey Insights and Post-Experiment Reflections}

%! Detailed analysis of survey questions

%Q6 To what extent do you find the overall complexity of this facade design appealing?
%\textit{Q6- Appeal of Complexity:}
Participants rated the appeal of facade complexity positively, with an average score indicating moderate to high appeal (\(Q\)\textsubscript{\small{6mean}} = 4.8; Figure~\ref{fig:SurveyQuestions6-10}).
This suggests that complex facade designs have the potential to attract and satisfy user preferences, aligning with broader architectural discourse and highlighting the potential for integrating such designs into future practices.

%Q7.How do you rate the intricacy of the patterns and textures used in this facade design?
%\textit{Q7-  Intricacy of Patterns and Textures:}
The survey also revealed that the intricacy of patterns and textures in the facade designs was well-received, scoring above the midpoint on the Likert scale~(\(Q\)\textsubscript{\small{7 mean}} = 5.0; Figure~\ref{fig:SurveyQuestions6-10}). This indicates that these design elements significantly contribute to user satisfaction and visual engagement, encouraging their exploration in future projects.
%Q8. To what extent do you think the arrangement of architectural elements on this facade adds to its visual interest?
%\textit{Q8 - Architectural Element Arrangement:}
Participants rated the contribution of architectural element arrangement to the facade's visual interest highly (\(Q\)\textsubscript{\small{8 mean}}=5.8; Figure~\ref{fig:SurveyQuestions6-10}), suggesting that thoughtful composition can greatly enhance a facade's appeal.
%Q9. How complex do you perceive the facade's use of patterns and textures?
%\textit{Q9 - Perception of Facade's Pattern and Texture Complexity:}
The complexity of patterns and textures was perceived as moderate to high appeal (\(Q\)\textsubscript{\small{9 mean}}=4.7; Figure~\ref{fig:SurveyQuestions6-10}), reflecting a balanced approach where complexity is appreciated but not overwhelming.
This points to the importance of finding the right complexity level that resonates with users.

%Q10. How detailed do you find the ornamentation on this facade design?
%\textit{Q10 - Detail in Ornamentation:}
The detail in ornamentation received a score that indicates users found it moderately detailed  (\(Q\)\textsubscript{\small{10 mean}}=5.0~; Figure~\ref{fig:SurveyQuestions6-10}).
This suggests a user preference for ornamentation that contributes to the visual richness of a facade without dominating the design.
%Q11. How much do the combination of materials contribute to the overall complexity of the facade?
%\textit{Q11 - Material Combination Contribution to Complexity:}
The combination of materials was seen as an important factor in contributing to facade complexity (\(Q\)\textsubscript{\small{11 mean}}=5.4; Figure~\ref{fig:SurveyQuestions11-15}), underscoring the role of materials in defining a facade's character and aesthetic appeal.

%Q12. To what degree does the composition of the facade strike you as aesthetically intricate?
%\textit{Q12 - Composition's Aesthetic Intricacy:}
The aesthetic intricacy of the composition received a moderately high rating (\(Q\)\textsubscript{\small{12 mean}}=5.2; Figure~\ref{fig:SurveyQuestions11-15}), emphasizing the value of thoughtful arrangement of design elements in enhancing a facade's visual complexity.
%Q13. How much do you believe that the arrangement of shapes and forms on the facade contributes to its complexity?
%\textit{Q13 - Contribution of Shapes and Forms to Complexity:}
Participants placed significant value on the role of shapes and forms in adding to facade complexity (\(Q\)\textsubscript{\small{13 mean}}=6.3; Figure~\ref{fig:SurveyQuestions11-15}).
It solidifies the notion that the strategic placement of visual elements holds substantial sway over how a facade is perceived, highlighting the need for architects to consider geometric aspects when designing complex facades.
%Q14. How significantly does the use of color enhance the facade's visual complexity?
%\textit{Q14 - Enhancement of Complexity by Color:}
The use of color was considered to moderately enhance visual complexity (\(Q\)\textsubscript{\small{14 mean}}=5.1; Figure~\ref{fig:SurveyQuestions11-15}). While not as impactful as form or texture, color is still an important design tool influencing complexity perception.
%Q15. How much depth and layering do you observe in the design of this facade?
%\textit{Q15 - Observation of Depth and Layering:}
Depth and layering were perceived as contributing moderately to facade complexity (\(Q\)\textsubscript{\small{15 mean}}=4.6; Figure~\ref{fig:SurveyQuestions11-15}), indicating that while not rated as highly significant as other factors, three-dimensionality and interplay of different design layers can enhance perceived complexity.

These insights suggest that participants appreciate complexity that is intelligently integrated into design through form, texture, and color, yet still desire a certain level of clarity without being overwhelmed by excessive details.
These findings can guide architects in creating facades that are complex yet coherent, appealing to a broad spectrum of users.
However, while participants were prompted to assess shapes and forms on the facade, the survey did not address the overall volumetric complexity or building massing.
The focus remained on surface details such as patterns and textures, rather than the three-dimensional geometry of the entire building.
This suggests that the impressions reflect mostly two-dimensional visual factors, potentially missing how volumetric complexity—like the building's size, shape, and articulation—affects perception.

%!\textit{Post-experiment interview:}
In post-experiment interviews, participants articulated a clear preference for the role of form in facade design, assigning it significantly more importance than materials at an 80:20 ratio, emphasizing form as a dominant influence in their assessment of facade complexity and aesthetic value.

Interestingly, when discussing complex facades, participants gained a more comprehensive understanding of how complexity is perceived in different contexts, thanks to the VR experiment, which allowed them to assess facades from both interior and exterior perspectives.
They agreed that intricate designs were appreciated when viewed from the exterior and generally favored more complex facades from an outside viewpoint.
However, after experiencing them from the inside, participants tended to prefer simpler designs—particularly when enjoying the simulated unobstructed views of the campus.
This distinction highlights that users may desire more visual complexity on facades facing non-critical or less scenic areas, while favoring simplicity on facades that interact with prominent external views to maintain visual comfort and openness from within.
The results of the VR experiment align with these findings, reinforcing the notion that the perception of facade complexity shifts depending on the viewer's position.
This underscores the need for context-sensitive design strategies that account for both indoor and outdoor experiences.

These interviews reveal a nuanced understanding of facade design among participants, highlighting the need for context-sensitive approaches that align architectural form and complexity with the functional and aesthetic needs of building users.
Complex designs can enhance privacy or visual interest where appropriate, while simpler facades can preserve the aesthetic experience of key external views.

%!% Implications: Why do your results matter?







