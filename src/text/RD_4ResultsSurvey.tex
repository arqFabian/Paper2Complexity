%!\subsection{Qualitative Analysis on Users Perception to Complex Facades}
%    \label{subsec:ResultsSurvey}
%    \input{Text/RD_ResultsSurvey}

%! survey results

We gathered additional insights through a survey and interviews, presenting a multifaceted view of user perceptions regarding architectural complexity.
The responses to the `complexity perception' section of the survey have been summarized in Figure~\ref{fig:SurveyQuestions6-10} and Figure~\ref{fig:SurveyQuestions11-15}, with evaluations conducted using a 7-point Likert scale.

Survey responses show a moderate to high endorsement of complexity in facade designs, with average ratings above \(3.5\) and mean scores around \(Mean = 5.2\).
\deleted{
A standard deviation of \(SD = 1.3\), however, reflects a range of views among participants. These variations in responses underscore the diverse perspectives and preferences among participants, enriching our understanding of how users perceive facade complexity.}
\added{
However, the standard deviation of \(SD = 1.3\) reveals considerable variations in participant responses, underscoring the subjective nature of architectural complexity. These deviations reflect the diverse perspectives and preferences among participants, which is significant in understanding that the perception of facade complexity is not uniform across users. This range of responses suggests that while some participants are more attuned to intricate designs, others may prefer simpler structures, thus reinforcing the need for adaptable design approaches in architecture.
}

\textbf{Survey Insights and Post-Experiment Reflections}

%! Detailed analysis of survey questions

%Q6 To what extent do you find the overall complexity of this facade design appealing?
%\textit{Q6- Appeal of Complexity:}
Participants rated the appeal of facade complexity positively, with an average score indicating moderate to high appeal (\(Q\)\textsubscript{\small{6mean}} = 4.8; Figure~\ref{fig:SurveyQuestions6-10}).
This suggests that complex facade designs have the potential to attract and satisfy user preferences, aligning with broader architectural discourse and highlighting the potential for integrating such designs into future practices.

%Q7.How do you rate the intricacy of the patterns and textures used in this facade design?
%\textit{Q7-  Intricacy of Patterns and Textures:}
The survey also revealed that the intricacy of patterns and textures in the facade designs was well-received, scoring above the midpoint on the Likert scale~(\(Q\)\textsubscript{\small{7 mean}} = 5.0; Figure~\ref{fig:SurveyQuestions6-10}). This indicates that these design elements significantly contribute to user satisfaction and visual engagement, encouraging their exploration in future projects.
%Q8. To what extent do you think the arrangement of architectural elements on this facade adds to its visual interest?
%\textit{Q8 - Architectural Element Arrangement:}
Participants rated the contribution of architectural element arrangement to the facade's visual interest highly (\(Q\)\textsubscript{\small{8 mean}}=5.8; Figure~\ref{fig:SurveyQuestions6-10}), suggesting that thoughtful composition can greatly enhance a facade's appeal.
%Q9. How complex do you perceive the facade's use of patterns and textures?
%\textit{Q9 - Perception of Facade's Pattern and Texture Complexity:}
The complexity of patterns and textures was perceived as moderate to high appeal (\(Q\)\textsubscript{\small{9 mean}}=4.7; Figure~\ref{fig:SurveyQuestions6-10}), reflecting a balanced approach where complexity is appreciated but not overwhelming.
This points to the importance of finding the right complexity level that resonates with users.

%Q10. How detailed do you find the ornamentation on this facade design?
%\textit{Q10 - Detail in Ornamentation:}
The detail in ornamentation received a score that indicates users found it moderately detailed  (\(Q\)\textsubscript{\small{10 mean}}=5.0~; Figure~\ref{fig:SurveyQuestions6-10}).
This suggests a user preference for ornamentation that contributes to the visual richness of a facade without dominating the design.
%Q11. How much do the combination of materials contribute to the overall complexity of the facade?
%\textit{Q11 - Material Combination Contribution to Complexity:}
The combination of materials was seen as an important factor in contributing to facade complexity (\(Q\)\textsubscript{\small{11 mean}}=5.4; Figure~\ref{fig:SurveyQuestions11-15}), underscoring the role of materials in defining a facade's character and aesthetic appeal.

%Q12. To what degree does the composition of the facade strike you as aesthetically intricate?
%\textit{Q12 - Composition's Aesthetic Intricacy:}
The aesthetic intricacy of the composition received a moderately high rating (\(Q\)\textsubscript{\small{12 mean}}=5.2; Figure~\ref{fig:SurveyQuestions11-15}), emphasizing the value of thoughtful arrangement of design elements in enhancing a facade's visual complexity.
%Q13. How much do you believe that the arrangement of shapes and forms on the facade contributes to its complexity?
%\textit{Q13 - Contribution of Shapes and Forms to Complexity:}
Participants placed significant value on the role of shapes and forms in adding to facade complexity (\(Q\)\textsubscript{\small{13 mean}}=6.3; Figure~\ref{fig:SurveyQuestions11-15}).
It solidifies the notion that the strategic placement of visual elements holds substantial sway over how a facade is perceived, highlighting the need for architects to consider geometric aspects when designing complex facades.
%Q14. How significantly does the use of color enhance the facade's visual complexity?
%\textit{Q14 - Enhancement of Complexity by Color:}
The use of color was considered to moderately enhance visual complexity (\(Q\)\textsubscript{\small{14 mean}}=5.1; Figure~\ref{fig:SurveyQuestions11-15}). While not as impactful as form or texture, color is still an important design tool influencing complexity perception.
%Q15. How much depth and layering do you observe in the design of this facade?
%\textit{Q15 - Observation of Depth and Layering:}
Depth and layering were perceived as contributing moderately to facade complexity (\(Q\)\textsubscript{\small{15 mean}}=4.6; Figure~\ref{fig:SurveyQuestions11-15}), indicating that while not rated as highly significant as other factors, three-dimensionality and interplay of different design layers can enhance perceived complexity.

These insights suggest that participants appreciate complexity that is intelligently integrated into design through form, texture, and color, yet still desire a certain level of clarity without being overwhelmed by excessive details.
These findings can guide architects in creating facades that are complex yet coherent, appealing to a broad spectrum of users.
\added{However, while participants were prompted to assess shapes and forms on the facade, the survey did not address the overall volumetric complexity or building massing. The focus remained on surface details such as patterns and textures, rather than the three-dimensional geometry of the entire building.
This suggests that the impressions reflect mostly two-dimensional visual factors, potentially missing how volumetric complexity—like the building's size, shape, and articulation—affects perception.}

%!\textit{Post-experiment interview:}
In post-experiment interviews, participants articulated a clear preference for the role of form in facade design, assigning it significantly more importance than materials at an 80:20 ratio, emphasizing form as a dominant influence in their assessment of facade complexity and aesthetic value.
\deleted{When discussing intricate facades, participants agreed that the design should complement and enhance surrounding views.
Simpler facades were preferred for areas with significant views, such as the front view of the campus as in the case of the experiment, ensuring these views were unobstructed and appreciated.
Conversely, for building aspects with less prominent views, like those facing adjacent structures, there was a preference for more complex designs to balance privacy needs with enhanced aesthetic presence.
This insight suggests intricate facade designs are seen as beneficial when serving specific purposes, such as enhancing privacy or contributing to visual interest, rather than being applied uniformly.}

\added{Interestingly, when discussing complex facades, participants gained a more comprehensive understanding of how complexity is perceived in different contexts, thanks to the VR experiment, which allowed them to assess facades from both interior and exterior perspectives. They agreed that intricate designs were appreciated when viewed from the exterior and generally favored more complex facades from an outside viewpoint. However, after experiencing them from the inside, participants tended to prefer simpler designs—particularly when enjoying the simulated unobstructed views of the campus. This distinction highlights that users may desire more visual complexity on facades facing non-critical or less scenic areas, while favoring simplicity on facades that interact with prominent external views to maintain visual comfort and openness from within. The results of the VR experiment align with these findings, reinforcing the notion that the perception of facade complexity shifts depending on the viewer's position. This underscores the need for context-sensitive design strategies that account for both indoor and outdoor experiences.}

These interviews reveal a nuanced understanding of facade design among participants, highlighting the need for context-sensitive approaches that align architectural form and complexity with the functional and aesthetic needs of building users.\added{Complex designs can enhance privacy or visual interest where appropriate, while simpler facades can preserve the aesthetic experience of key external views.}

%!% Implications: Why do your results matter?

%% Your overall aim is to show the reader exactly what your research has contributed, and why they should care.


        %And while previous research has focused on small details \cite{AlSaggaf2021}, these results demonstrate that an overview of the whole scale of the project can also be impacted by the integration of VR techniques in a positive matter and perhaps show a larger impact that may tilt the preferential adoption of VR when solving this issue.






