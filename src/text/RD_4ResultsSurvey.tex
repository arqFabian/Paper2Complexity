%!\subsection{Qualitative Analysis on Users Perception to Complex Facades}
%    \label{subsec:ResultsSurvey}
%    \input{Text/RD_ResultsSurvey}

%! Post experiment survey results

We gathered additional insights through a survey and interviews, presenting a multifaceted view of user perceptions regarding architectural complexity.
The responses to the `complexity perception' section of the survey have been summarized in Figure\ref{fig:SurveyQuestions6-10} and Figure\ref{fig:SurveyQuestions11-15}, with evaluations conducted using a 7-point Likert scale.


Survey responses show a moderate to high endorsement of complexity in facade designs, with average ratings above \(3.5\) and mean scores around \(Mean = 4.9\).
A standard deviation of \(SD = 1.5\), however, reflects a range of views among participants.
These variations in responses underscore the diverse perspectives and preferences among participants, enriching our understanding of how users perceive facade complexity.

\textbf{User-Centric Design Preferences: Survey Insights and Post-Experiment Reflections}

Key takeaways from the survey include:

%! Detailed analysis of survey questions

%Q6 To what extent do you find the overall complexity of this facade design appealing?
\textit{Q6- Appeal of Complexity:}
Participants rated the appeal of facade complexity positively , with an average score indicating moderate to high appeal (\(Q\)\textsubscript{\small{6 mean}}=4.3).
This suggests that complex facade designs have the potential to attract and satisfy user preferences, this finding aligns with the broader architectural discourse, suggesting a growing acceptance of architectural complexity in contemporary design and highlights the potential for the integration of such designs in future architectural practices (Question 6 in Figure~\ref{fig:SurveyQuestions6-10}).

%Q7.How do you rate the intricacy of the patterns and textures used in this facade design?
\textit{Q7-  Intricacy of Patterns and Textures:}
The survey revealed that the intricacy of patterns and textures used in the facade designs was well-received, scoring above the midpoint on the Likert scale (\(Q\)\textsubscript{\small{7 mean}}=4.6).
This indicates that such design elements are important for user satisfaction and can significantly contribute to the visual engagement of a building.
(Question 7 in Figure~\ref{fig:SurveyQuestions6-10}).
It encourages the exploration and incorporation of such elements into future architectural projects, knowing that they are likely to resonate with users and contribute positively to the overall appeal of the design.

%Q8. To what extent do you think the arrangement of architectural elements on this facade adds to its visual interest?
\textit{Q8 - Architectural Element Arrangement:}
Participants highly rated the contribution of architectural element arrangement to the facade's visual interest  (\(Q\)\textsubscript{\small{8 mean}}=5.5).
This suggests that how elements are composed can greatly impact a facade's appeal, hinting at the importance of thoughtful design in creating engaging architectural experiences (Question 8 in Figure~\ref{fig:SurveyQuestions6-10}).

%Q9. How complex do you perceive the facade's use of patterns and textures?
\textit{Q9 - Perception of Facade's Pattern and Texture Complexity:}
Participants perceived the complexity of patterns and textures as moderate  (\(Q\)\textsubscript{\small{9 mean}}=3.9).
This reflects a balanced approach to facade design, where complexity is appreciated but not overwhelming, pointing to the importance of finding the right level of complexity that resonates with users (Question 9 in Figure~\ref{fig:SurveyQuestions6-10}).

%Q10. How detailed do you find the ornamentation on this facade design?
\textit{Q10 - Detail in Ornamentation:}
The detail in ornamentation received a score that indicates users found it moderately detailed  (\(Q\)\textsubscript{\small{10 mean}}=5.0).
This suggests a user preference for ornamentation that contributes to the visual richness of a facade without dominating the design (Question 10 in Figure~\ref{fig:SurveyQuestions6-10}).

%Q11. How much do the combination of materials contribute to the overall complexity of the facade?
\textit{Q11 - Material Combination Contribution to Complexity:}
The survey participants regarded the combination of materials as an important factor contributing to facade complexity (\(Q\)\textsubscript{\small{11 mean}}=5.0).
This finding underscores the importance of materials in creating visually appealing and complex facades, highlighting how materials play a crucial role in defining the facade's character and can significantly impact its aesthetic appeal.(Question 11 in Figure~\ref{fig:SurveyQuestions11-15}).

%Q12. To what degree does the composition of the facade strike you as aesthetically intricate?
\textit{Q12 - Composition's Aesthetic Intricacy:}
The aesthetic intricacy of the composition was rated as moderately high (\(Q\)\textsubscript{\small{12 mean}}=4.9), emphasizing that users value the thoughtful arrangement of design elements that contribute to a facade's overall aesthetic complexity(Question 12 in Figure~\ref{fig:SurveyQuestions11-15}). This underscores the notion that a well-calibrated and moderately complex composition can augment the overall visual appeal of architectural facades.

%Q13. How much do you believe that the arrangement of shapes and forms on the facade contributes to its complexity?
\textit{Q13 - Contribution of Shapes and Forms to Complexity:}
This question scored highly (\(Q\)\textsubscript{\small{13 mean}}=6.3), indicating that users place a significant value on the role of shapes and forms in adding to a facade's complexity.
This highlights the need for architects to consider the geometric aspects of design when aiming to create complex facades (Question 13 in Figure~\ref{fig:SurveyQuestions11-15}). It solidifies the notion that the strategic placement of visual elements holds substantial sway over how a facade is perceived, highlighting its essential role in architectural design endeavors aimed at achieving complexity.

%Q14. How significantly does the use of color enhance the facade's visual complexity?
\textit{Q14 - Enhancement of Complexity by Color:}
The use of color was considered to moderately enhance the facade's visual complexity (\(Q\)\textsubscript{\small{14 mean}}=5.1).
While not as impactful as form or texture, color is still an important design tool that can subtly influence the perception of complexity(Question 14 in Figure~\ref{fig:SurveyQuestions11-15}).
This insight is valuable for architects and designers, indicating that while color is a consideration in creating visually engaging facades, other design elements may have a more substantial influence on complexity.

%Q15. How much depth and layering do you observe in the design of this facade?
\textit{Q15 - Observation of Depth and Layering:}
Depth and layering in design were perceived as contributing moderately to facade complexity(\(Q\)\textsubscript{\small{15 mean}}=4.3).
While not rated as highly significant as other factors like the arrangement of shapes and forms, it still contributes to the overall perception of complexity.
This implies that three-dimensionality and the interplay of different design layers can enhance the perception of complexity and should be considered in facade design (Question 15 in Figure~\ref{fig:SurveyQuestions11-15}).

From the insights from the questions, we can see a pattern where participants appreciate complexity that is intelligently integrated into design through form, texture, and color, but still desire a certain level of clarity and not be overwhelmed by excessive details.
These insights can guide architects and designers in creating facades that are complex yet coherent, appealing to a broad spectrum of users.

\textit{Post-experiment interview:}
In post-experiment interviews, participants articulated a clear preference for the role of form in facade design, assigning it significantly more importance than materials at an 80:20 ratio.
This perspective underlines form as a dominant influence in their assessment of facade complexity and aesthetic value.

When discussing intricate facades, there was a shared viewpoint that the design should complement and enhance the surrounding views.
Participants preferred simpler facades for areas with significant vistas, such as the front view of the campus, to ensure these views were unobstructed and appreciated.

Conversely, for building aspects with less prominent views, like those facing adjacent structures, there was a preference for more complex designs.
Participants felt that such designs could effectively balance the need for privacy with an enhanced aesthetic presence.
This insight suggests that intricate facade designs are seen as beneficial when they serve specific purposes, such as enhancing privacy or contributing to the building’s visual interest, rather than being applied uniformly.

These interviews reveal a nuanced understanding of facade design among participants, highlighting the need for context-sensitive approaches that align architectural form and complexity with the functional and aesthetic needs of building users.



%!% Implications: Why do your results matter?
%% Your overall aim is to show the reader exactly what your research has contributed, and why they should care.


        %And while previous research has focused on small details \cite{AlSaggaf2021}, these results demonstrate that an overview of the whole scale of the project can also be impacted by the integration of VR techniques in a positive matter and perhaps show a larger impact that may tilt the preferential adoption of VR when solving this issue.






