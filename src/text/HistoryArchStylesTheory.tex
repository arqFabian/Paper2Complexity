
%\subsection{Human-Centric Design Philosophy across the history of architecture styles}
%\label{subsec:TimelineArchitectureStyles}

%===========================
%%Human-Centric Design Philosophy: Explore the historical evolution of architectural philosophies that prioritize human experience and well-being. Discuss how different architectural styles and movements have addressed the balance between ornamentation, functionality, and human comfort.
%===========================

Architecture stands as a unique art form, setting itself apart from other creative mediums.
It requires not only the transformation of the ordinary into the extraordinary, akin to painting and sculpture, but also the imperative to fulfill the purpose and functionality of a building\cite{Hnin2022}.

At the core of architectural evolution resides a dynamic interplay between simplicity and complexity, often guided by the intersection of societal values and technological advancements\cite{Economakis2023}.
However, it's important to clarify that within the context of this research, neither simplicity nor complexity carry any inherent negative or positive connotations.
Using the term ``simplicity`` here doesn't imply a condemnation of one style in favor of another;
rather, it highlights that throughout the history of architecture and its various styles, there have been periods characterized by evident complexities as well as phases where designs embraced apparent simplicities that hid their intricacies into themselves.
Undoubtedly, every prevailing architectural style of its time has contributed masterpieces to the built environment.

Consider, for instance, the transition from the robust Romanesque classic style of the 10th century, notably exhibited in churches, to the Gothic style brought by groundbreaking advancements of the 12th century that introduced buttresses, revolutionizing load distribution\cite{Arora2023}(see Figure\ref{fig:RomanesquevsGothic}).
This innovation propelled churches skyward, inviting luminous interplays to embellish the interiors through stunning stained-glass windows bedecked in intricate design\cite{Stacbond2020}.

This oscillation between complexity and simplicity persists through time— transitioning from the intricate Gothic style to the revival of Greek and Roman ideals, exemplified by the symmetrical perfection of the Renaissance era in the 14th century.

This resurgence was succeeded by the opulent ornamentation and exuberance of the Baroque style in the 16th century, essentially a creation of the Chatolic church, that started in Europe and later spread across the New world in other Chatolic nations.
It was characterized by the preference for curves over the straight line, an interest in complex plans and volumes, overlapping architectural forms, and an interest for combining the three arts of painting, sculpture and architecture\cite{Economakis2023}.

The progression continued with the classical revival of the 18th century, heavily influenced by the architectural principles of classical Greece and the Palladian style.
This revival aimed to create picturesque compositions and sought to reestablish the rational simplicity that defined ancient Greece and Rome\cite{Economakis2023}.

The late 18th century marked the beginning of globalization and an information explosion, which enabled scientific advancements but also blurred the geographical origins of architectural forms.
As a result, architectural expressions from diverse cultures became acceptable options in various contexts.
This phenomenon led to the incorporation of multiple historical references into buildings, soon Gothic, Oriental or even Egyptian styles, to name a few, were integrated into victorian houses, resulting in stylistic confusion known as Relativism and Subjectivism.
Architects grappled with a multitude of options and no clear consensus on architectural expression\cite{Economakis2023}.

In response to this architectural chaos, the Neoclassical style would appear, envisioned under the conservative academicism of the 19th century that aimed to bring order by consolidating the architectural profession under the teachings of classical architecture as idealized during the Renaissance.
Characterized by being bilaterally symmetrical and sel-referential or a-contextual with little regard to how they integrated to urban settings.
This heavily inspired greek and Palladian architecture would integrate with new technologies like reinforced concrete and cast iron.
Led by the École des Beaux-Arts in Paris, the Neoclassic style gained prominence and persisted until the early 20th century\cite{Economakis2023}.

However, by the late 19th century, the increasingly rigid academicism of the École des Beaux-Arts gave way to a shift towards ostentation.
The lack of volumetric hierarchy in building designs led to a departure from the sobriety and principles of the Renaissance.
Their monumentality originally praised was now resulting in a tendency towards overelaboration as buildings competed for attention\cite{Economakis2023}.

During this period The Art Deco movement would also make its appearance, flourishing in the 1920s and 30s, on a style that celebrated technological progress through luxurious materials and intricate patterns.
It blended modern design with artistic craftsmanship, embracing geometric shapes, bold colors, and streamlined forms.
Art Deco architecture often featured sleek lines, zigzags, and stylized motifs, capturing the spirit of the era's dynamism and opulence.\cite{Arora2023}

In response to the attitudes of the Ecole of Beaux Arts and its cult towards exuberance, an antagonism was formed specially among the more socially-minded ones and during the first half of the 20th century, society would witness the emergence of Modern Architecture and rationalism with their increasing radical approaches towards the built environment (see Figure\ref{fig:NeoclassicalvsModernism}).

%%Figure neoclassicim vs modernism
     \begin{figure}[htb]
          \centering
          \includegraphics[width= \linewidth]{Images/NeoclassicismVsModernism}
          \caption{Neoclassic building "Paris Opera" 19th AC (left) vs Modernist house "Villa Savoye" 20th AC (right). From Complexity to simplicity. (\textit{Images edited from source:\cite{Stacbond2020}})}
          \label{fig:NeoclassicalvsModernism}
        \end{figure}

This architectural ethos adopted the maxim ``Form follows function'', emphasizing functionalism and characterized by a rejection of traditional ornament in favor of new forms of more subtles intricacies like the “aesthetics of machinery” that showcased architecture  enriched  with  only  the  beauty of its lines and the use of new-age materials such as steel, glass, and concrete\cite{Gage2015}.
Venturi\cite{Venturi1972} further reflects that modern architecture considered as progressive, If not revolutionary, utopian, and puristic;
it  is  dissatisfied  with existing conditions and its architects would prefer to change rather than enhance what is there.

%% add some extra references to the modernist section specially about the urban configuration

The postmodernism style of the late 60's marks a radical return of ornament in form recognizing that even simplified modern elements serve as ornamentation focusing on the thought of freeing design element from oppresive modern constraints.



The late 20th century embraced imagination and expression through architects like Frank Gehry, Zaha Hadid, and Rem Koolhaas.
Their constructions stood as monumental expressions of ornament, enabled by digital technologies (Figure\ref{fig:Modernismvspostmodernism}).

%%Figure neoclassicim vs modernism
     \begin{figure}[htb]
          \centering
          \includegraphics[width= \linewidth]{Images/modernism vs postmodernism}
          \caption{Modernist building "Bauhaus School" 20th AC (left) vs Postmodernist "Guggenheim museum" 1997 (right). From simplicity to Complexity. (\textit{Images edited from source:\cite{Arora2023}})}
          \label{fig:Modernismvspostmodernism}
        \end{figure}

Intricate shapes and structures have materialized, spanning from juxtaposed ornaments to innovative transformative structural ornamentation.
This pursuit of complexity is a global phenomenon, prompting a competitive quest among leading contemporary architectural firms to harness parametric design as a tool to conceive groundbreaking new buildings.\cite{Burlando2019}.

However, due to their complexity, these structures remained exceptional, not integrated into the urban fabric.
Now at the beginning of the 21st century and the advent of the 4th industrial revolution, characterized by a fusion of technologies that is blurring the lines between the physical, digital, and biological spheres\cite{Schwab2016}, forecasts the democratization of digital fabrication which in turn will bring a paradigm shift, offering globally to all cultures the means to express authenticity through complex parametric designs, signaling a contemporary era embracing complexity and ornament.

Amidst this historical exploration, it becomes evident that the architecture of the future is poised to harmonize ornamentation, functionality, and human comfort.
The trajectory points towards a style of complexity—a style that crafts a delicate equilibrium, resonating with the values of our time and the technological possibilities at our fingertips.

In this intricate interplay, architecture emerges as a tangible synthesis of human experience and creative expression, balancing ornament's aesthetic allure, the essential functionality of the built environment, and the crucial comfort of its inhabitants.