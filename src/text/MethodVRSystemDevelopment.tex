%    \subsection{VR system for Complexity Analysis in facade design}
%    \label{subsec:VRsystemDevelopment}
%    %    \subsection{VR system for Complexity Analysis in facade design}
%    \label{subsec:VRsystemDevelopment}
%    %    \subsection{VR system for Complexity Analysis in facade design}
%    \label{subsec:VRsystemDevelopment}
%    %    \subsection{VR system for Complexity Analysis in facade design}
%    \label{subsec:VRsystemDevelopment}
%    \input{text/MethodVRSystemDevelopment.tex}VR System development
%! Concise
The VR system developed for this study plays a pivotal role in our experiment, enabling participants to immerse themselves in a world where they can explore and interact with facades of varying complexity levels.

The system comprises three integral components (see Figure \ref{fig:MethodologyFlowchart}):

\begin{enumerate}
\item \textbf{3D Modeling:} Detailed in Section \ref{subsubsec:3DModeling} and labeled as (1) in Figure \ref{fig:MethodologyFlowchart}, this component is responsible for creating realistic 3D representations of the experiment site and ten distinct facade variations across three patterns for user interaction.

\item \textbf{CICA System for 3D-Modeled Facades Analysis:} Detailed in Section \ref{subsubsec:CICAfor3DmodeledFacades} and labeled as (2) in Figure \ref{fig:MethodologyFlowchart}, the CICA system (as detailed in Section \ref{subsec:Computational Image Complexity analysis}) provides complexity scores for the 3D-modeled facades.
This ensures a systematic approach to evaluating and selecting facade designs for the VR experience.

\item \textbf{VR Integration:} Detailed in Section \ref{subsubsec:VR_integration} and labeled as (3) in Figure \ref{fig:MethodologyFlowchart}, this component, developed using Unity, offers an immersive simulation with a data visualization interface.
Participants can navigate the virtual space, experiencing the building and its surroundings, and interact with various facade designs.
Real-time feedback enhances this interaction by showing the impact of different facade variations.
\end{enumerate}

Together, these components form a comprehensive system that leverages 3D modeling, data visualization, and virtual reality to recreate varying levels of facade complexity within building designs.
This system enables a thorough exploration of architectural complexity preferences.

Having outlined the VR system for complexity analysis in facade design, we will now delve into the specifics of its three fundamental components.


%!Original
%As we delve into the realm of virtual reality (VR), our next undertaking involves the development of the VR Complexity Analysis system for facade design.
%This innovative system will play a pivotal role in our experiment, enabling participants to immerse themselves in a world where they can explore and interact with facades of varying complexity levels.
%
%The primary objective of the VR Complexity Analysis system is to capture user preferences and tolerance thresholds related to facade complexity.
%It achieves this by creating an immersive virtual environment designed to investigate architectural complexity preferences.
%
%To accomplish this ambitious goal, the system integrates three key components, as depicted in Figure\ref{fig:MethodologyFlowchart}: 3D modeling, the application of the Computational Image Complexity Analysis (CICA) system, and VR integration.
%
%The ``3D modeling'' module, developed in Blender (v3.6), serves as the initial component.
%Its role is to simulate the site and building where our experiment will take place and generate ten distinct facade variations for each of the three distinctive patterns.
%This capability empowers the system to create realistic, detailed 3D models of the facades in question, allowing participants to gain a deeper understanding of their spatial impact within the environment.
%
%The second vital component is the application of the Computational Image Complexity Analysis (CICA) system, implemented in Python and seamlessly integrated into the Blender (v3.6) environment.
%The CICA system plays a central role by providing complexity scores for various iterations of 3D-modeled facade variations (see section\ref{subsec:Image Complexity analysis}).
%This systematic approach establishes an organized framework for evaluating facade design iterations, creating a selection process and hierarchy for the facade designs to be integrated into the VR experience.
%
%Finally, the `Virtual reality integration' module, developed in Unity, serves as the third critical component.
%This module includes an immersive simulation and a data visualization interface that transports users into the VR simulation of the experiment's location for facade complexity analysis.
%
%Within this dynamic virtual environment, participants can explore and interact with the building from both inside and outside, visualize its context, and manipulate the facade variations through the user interface.
%
%Seamlessly integrated with the simulation, the interface provides real-time feedback on the impact of different facade variations on the building, facilitating more effective and informed decision-making when selecting a specific level of complexity.
%
%Together, these components form a comprehensive system that leverages 3D modeling, data visualization, and virtual reality to recreate varying levels of facade complexity within building designs, enabling a thorough exploration of architectural complexity preferences.
%
%Now that we have outlined the overarching vision for our VR Complexity Analysis system in facade design, let's delve into the specifics of its three fundamental components.VR System development
%! Concise
The VR system developed for this study plays a pivotal role in our experiment, enabling participants to immerse themselves in a world where they can explore and interact with facades of varying complexity levels.

The system comprises three integral components (see Figure \ref{fig:MethodologyFlowchart}):

\begin{enumerate}
\item \textbf{3D Modeling:} Detailed in Section \ref{subsubsec:3DModeling} and labeled as (1) in Figure \ref{fig:MethodologyFlowchart}, this component is responsible for creating realistic 3D representations of the experiment site and ten distinct facade variations across three patterns for user interaction.

\item \textbf{CICA System for 3D-Modeled Facades Analysis:} Detailed in Section \ref{subsubsec:CICAfor3DmodeledFacades} and labeled as (2) in Figure \ref{fig:MethodologyFlowchart}, the CICA system (as detailed in Section \ref{subsec:Computational Image Complexity analysis}) provides complexity scores for the 3D-modeled facades.
This ensures a systematic approach to evaluating and selecting facade designs for the VR experience.

\item \textbf{VR Integration:} Detailed in Section \ref{subsubsec:VR_integration} and labeled as (3) in Figure \ref{fig:MethodologyFlowchart}, this component, developed using Unity, offers an immersive simulation with a data visualization interface.
Participants can navigate the virtual space, experiencing the building and its surroundings, and interact with various facade designs.
Real-time feedback enhances this interaction by showing the impact of different facade variations.
\end{enumerate}

Together, these components form a comprehensive system that leverages 3D modeling, data visualization, and virtual reality to recreate varying levels of facade complexity within building designs.
This system enables a thorough exploration of architectural complexity preferences.

Having outlined the VR system for complexity analysis in facade design, we will now delve into the specifics of its three fundamental components.


%!Original
%As we delve into the realm of virtual reality (VR), our next undertaking involves the development of the VR Complexity Analysis system for facade design.
%This innovative system will play a pivotal role in our experiment, enabling participants to immerse themselves in a world where they can explore and interact with facades of varying complexity levels.
%
%The primary objective of the VR Complexity Analysis system is to capture user preferences and tolerance thresholds related to facade complexity.
%It achieves this by creating an immersive virtual environment designed to investigate architectural complexity preferences.
%
%To accomplish this ambitious goal, the system integrates three key components, as depicted in Figure\ref{fig:MethodologyFlowchart}: 3D modeling, the application of the Computational Image Complexity Analysis (CICA) system, and VR integration.
%
%The ``3D modeling'' module, developed in Blender (v3.6), serves as the initial component.
%Its role is to simulate the site and building where our experiment will take place and generate ten distinct facade variations for each of the three distinctive patterns.
%This capability empowers the system to create realistic, detailed 3D models of the facades in question, allowing participants to gain a deeper understanding of their spatial impact within the environment.
%
%The second vital component is the application of the Computational Image Complexity Analysis (CICA) system, implemented in Python and seamlessly integrated into the Blender (v3.6) environment.
%The CICA system plays a central role by providing complexity scores for various iterations of 3D-modeled facade variations (see section\ref{subsec:Image Complexity analysis}).
%This systematic approach establishes an organized framework for evaluating facade design iterations, creating a selection process and hierarchy for the facade designs to be integrated into the VR experience.
%
%Finally, the `Virtual reality integration' module, developed in Unity, serves as the third critical component.
%This module includes an immersive simulation and a data visualization interface that transports users into the VR simulation of the experiment's location for facade complexity analysis.
%
%Within this dynamic virtual environment, participants can explore and interact with the building from both inside and outside, visualize its context, and manipulate the facade variations through the user interface.
%
%Seamlessly integrated with the simulation, the interface provides real-time feedback on the impact of different facade variations on the building, facilitating more effective and informed decision-making when selecting a specific level of complexity.
%
%Together, these components form a comprehensive system that leverages 3D modeling, data visualization, and virtual reality to recreate varying levels of facade complexity within building designs, enabling a thorough exploration of architectural complexity preferences.
%
%Now that we have outlined the overarching vision for our VR Complexity Analysis system in facade design, let's delve into the specifics of its three fundamental components.VR System development
%! Concise
The VR system developed for this study plays a pivotal role in our experiment, enabling participants to immerse themselves in a world where they can explore and interact with facades of varying complexity levels.

The system comprises three integral components (see Figure \ref{fig:MethodologyFlowchart}):

\begin{enumerate}
\item \textbf{3D Modeling:} Detailed in Section \ref{subsubsec:3DModeling} and labeled as (1) in Figure \ref{fig:MethodologyFlowchart}, this component is responsible for creating realistic 3D representations of the experiment site and ten distinct facade variations across three patterns for user interaction.

\item \textbf{CICA System for 3D-Modeled Facades Analysis:} Detailed in Section \ref{subsubsec:CICAfor3DmodeledFacades} and labeled as (2) in Figure \ref{fig:MethodologyFlowchart}, the CICA system (as detailed in Section \ref{subsec:Computational Image Complexity analysis}) provides complexity scores for the 3D-modeled facades.
This ensures a systematic approach to evaluating and selecting facade designs for the VR experience.

\item \textbf{VR Integration:} Detailed in Section \ref{subsubsec:VR_integration} and labeled as (3) in Figure \ref{fig:MethodologyFlowchart}, this component, developed using Unity, offers an immersive simulation with a data visualization interface.
Participants can navigate the virtual space, experiencing the building and its surroundings, and interact with various facade designs.
Real-time feedback enhances this interaction by showing the impact of different facade variations.
\end{enumerate}

Together, these components form a comprehensive system that leverages 3D modeling, data visualization, and virtual reality to recreate varying levels of facade complexity within building designs.
This system enables a thorough exploration of architectural complexity preferences.

Having outlined the VR system for complexity analysis in facade design, we will now delve into the specifics of its three fundamental components.


%!Original
%As we delve into the realm of virtual reality (VR), our next undertaking involves the development of the VR Complexity Analysis system for facade design.
%This innovative system will play a pivotal role in our experiment, enabling participants to immerse themselves in a world where they can explore and interact with facades of varying complexity levels.
%
%The primary objective of the VR Complexity Analysis system is to capture user preferences and tolerance thresholds related to facade complexity.
%It achieves this by creating an immersive virtual environment designed to investigate architectural complexity preferences.
%
%To accomplish this ambitious goal, the system integrates three key components, as depicted in Figure\ref{fig:MethodologyFlowchart}: 3D modeling, the application of the Computational Image Complexity Analysis (CICA) system, and VR integration.
%
%The ``3D modeling'' module, developed in Blender (v3.6), serves as the initial component.
%Its role is to simulate the site and building where our experiment will take place and generate ten distinct facade variations for each of the three distinctive patterns.
%This capability empowers the system to create realistic, detailed 3D models of the facades in question, allowing participants to gain a deeper understanding of their spatial impact within the environment.
%
%The second vital component is the application of the Computational Image Complexity Analysis (CICA) system, implemented in Python and seamlessly integrated into the Blender (v3.6) environment.
%The CICA system plays a central role by providing complexity scores for various iterations of 3D-modeled facade variations (see section\ref{subsec:Image Complexity analysis}).
%This systematic approach establishes an organized framework for evaluating facade design iterations, creating a selection process and hierarchy for the facade designs to be integrated into the VR experience.
%
%Finally, the `Virtual reality integration' module, developed in Unity, serves as the third critical component.
%This module includes an immersive simulation and a data visualization interface that transports users into the VR simulation of the experiment's location for facade complexity analysis.
%
%Within this dynamic virtual environment, participants can explore and interact with the building from both inside and outside, visualize its context, and manipulate the facade variations through the user interface.
%
%Seamlessly integrated with the simulation, the interface provides real-time feedback on the impact of different facade variations on the building, facilitating more effective and informed decision-making when selecting a specific level of complexity.
%
%Together, these components form a comprehensive system that leverages 3D modeling, data visualization, and virtual reality to recreate varying levels of facade complexity within building designs, enabling a thorough exploration of architectural complexity preferences.
%
%Now that we have outlined the overarching vision for our VR Complexity Analysis system in facade design, let's delve into the specifics of its three fundamental components.VR System development
%! Concise
As we delve into the realm of virtual reality (VR), our next undertaking involves the development of the VR Complexity Analysis system for facade design.
This innovative system will play a pivotal role in our experiment, enabling participants to immerse themselves in a world where they can explore and interact with facades of varying complexity levels.

The system comprises three integral components:

The ``3D modeling'' module, utilizing Blender (v3.6), this module creates realistic 3D models of the experiment site and ten distinct facade variations for each of the three patterns.
It ensures that participants have a detailed and spatially accurate representation of the facades to interact with.

The second vital component is the application of the Computational Image Complexity Analysis (CICA) system, integrated with Blender, the CICA system (as detailed in section\ref{subsec:Image Complexity analysis}) provides complexity scores for the 3D-modeled facades.
This step ensures an organized and systematic approach to evaluating and selecting facade designs for the VR experience.

Finally, the `Virtual reality integration' module,developed using Unity, this module offers an immersive simulation with a data visualization interface.
Participants can navigate the virtual space, experiencing the building and its surroundings, and interact with various facade designs.
This interaction is enhanced by real-time feedback on the impact of different facade variations.

Together, these components form a comprehensive system that leverages 3D modeling, data visualization, and virtual reality to recreate varying levels of facade complexity within building designs, enabling a thorough exploration of architectural complexity preferences.

Now that we have outlined the overarching vision for our VR Complexity Analysis system in facade design, let's delve into the specifics of its three fundamental components.


%!Original
%As we delve into the realm of virtual reality (VR), our next undertaking involves the development of the VR Complexity Analysis system for facade design.
%This innovative system will play a pivotal role in our experiment, enabling participants to immerse themselves in a world where they can explore and interact with facades of varying complexity levels.
%
%The primary objective of the VR Complexity Analysis system is to capture user preferences and tolerance thresholds related to facade complexity.
%It achieves this by creating an immersive virtual environment designed to investigate architectural complexity preferences.
%
%To accomplish this ambitious goal, the system integrates three key components, as depicted in Figure\ref{fig:MethodologyFlowchart}: 3D modeling, the application of the Computational Image Complexity Analysis (CICA) system, and VR integration.
%
%The ``3D modeling'' module, developed in Blender (v3.6), serves as the initial component.
%Its role is to simulate the site and building where our experiment will take place and generate ten distinct facade variations for each of the three distinctive patterns.
%This capability empowers the system to create realistic, detailed 3D models of the facades in question, allowing participants to gain a deeper understanding of their spatial impact within the environment.
%
%The second vital component is the application of the Computational Image Complexity Analysis (CICA) system, implemented in Python and seamlessly integrated into the Blender (v3.6) environment.
%The CICA system plays a central role by providing complexity scores for various iterations of 3D-modeled facade variations (see section\ref{subsec:Image Complexity analysis}).
%This systematic approach establishes an organized framework for evaluating facade design iterations, creating a selection process and hierarchy for the facade designs to be integrated into the VR experience.
%
%Finally, the `Virtual reality integration' module, developed in Unity, serves as the third critical component.
%This module includes an immersive simulation and a data visualization interface that transports users into the VR simulation of the experiment's location for facade complexity analysis.
%
%Within this dynamic virtual environment, participants can explore and interact with the building from both inside and outside, visualize its context, and manipulate the facade variations through the user interface.
%
%Seamlessly integrated with the simulation, the interface provides real-time feedback on the impact of different facade variations on the building, facilitating more effective and informed decision-making when selecting a specific level of complexity.
%
%Together, these components form a comprehensive system that leverages 3D modeling, data visualization, and virtual reality to recreate varying levels of facade complexity within building designs, enabling a thorough exploration of architectural complexity preferences.
%
%Now that we have outlined the overarching vision for our VR Complexity Analysis system in facade design, let's delve into the specifics of its three fundamental components.