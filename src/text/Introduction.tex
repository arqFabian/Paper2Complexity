%%Introduction

%=================================
%%Reference
%%https://www.scribbr.com/research-paper/research-paper-introduction/
%%State the objectives of the work and provide an adequate background, avoiding a detailed literature survey or a summary of the results.

%Step1. Introduce your topic.
     %This is generally accomplished with a strong opening hook.
%Step2. Describe the background.
     %For a paper describing original research, you’ll instead provide an overview of the most relevant research that has already been conducted.
%Step3. Establish your research problem.
     %In an empirical research paper, try to lead into the problem on the basis of your discussion of the literature.
%Step4. Specify your objective(s).
     %The research question is the question you want to answer in an empirical research paper. If your research involved testing hypotheses, these should be stated along with your research question.
%Step 5: Map out your paper.
     %The final part of the introduction is often dedicated to a brief overview of the rest of the paper.
%=================================

Recent advancements in computer-aided design (CAD) and digital fabrication are significantly influencing architectural practice.
These technologies facilitate the creation of complex architectural forms that were previously unfeasible, challenging the minimalist aesthetics traditionally dominated by modernist principles.
As a result, architects now have the tools to depart from prevailing blank walls or glass facades towards more richly detailed environments, potentially redefining the relationship between form and function as traditionally understood in the modernist context \cite{Gage2015}.
However, the indiscriminate pursuit of popular trends using new technologies can lead to designs that quickly become outdated.
Fashion-like gimmicks, often adopted without a deeper understanding of their long-term impact, do not necessarily resonate with users over time.
Consequently, many buildings constructed in the last fifty years are now facing demolition, lacking enduring qualities that foster a genuine connection with their inhabitants \cite{Aesthetic2022}.
This highlights a critical need for architectural practices that not only embrace innovation but also ensure relevance and sustainability through designs that engage and inspire \cite{Brielmann2022}.

Despite significant research aimed at optimizing resource use and boosting environmental sustainability, the disconnect between data-driven design and user satisfaction remains pronounced.
While computational advancements hold potential to transform urban landscapes, neglecting the psychological connection between people and their built environment can doom even sustainably designed buildings to obsolescence, incurring substantial societal and environmental costs \cite{Aesthetic2022}.
Neuroscientists and other researchers have been working to define the aspects that stimulate our satisfaction with the built environment that transcend subjective aesthetic judgments, suggesting that certain structural elements resonate universally on a neurophysiological level, offering emotional nourishment and promoting urban well-being \cite{Brielmann2022}.
These themes are often found in Biophilic design patterns that articulate the relationships between nature, human biology, and the design of the built environment.

Historical and contemporary architectures have frequently incorporated natural patterns and complexities into their designs, underscoring a fundamental human affinity for biophilic elements that enhance psychological and physical well-being in urban environments.
These design strategies, grounded in biophilic principles, align closely with the ingrained human need for interaction with nature-like settings, supporting a vibrant existence within urban landscapes \cite{Browning2014}.
The recurring use of fractal geometries, seen in the repetitive patterns of elements like clouds, trees, and fern leaves, exemplifies nature's inherent complexity that has historically influenced architectural aesthetics.
These natural complexities, when incorporated into building designs, place observers in a 'comfort zone' of sensory engagement, balancing stimulation and comfort through visually rich yet orderly environments \cite{Browning2014}.

This well-established relationship between natural patterns and human well-being was mathematically formalized by Birkhoff (1933), who introduced a ratio of order to complexity as a measure of aesthetic value \cite{Birkhoff1933}.
Despite some uncertainties, there is a widely held consensus within the field that exploring the scientific dimensions of aesthetics and appealing stimuli can offer profound insights into human brain functionality and behavior, significantly enriching our comprehension of these processes \cite{Redies2015}.
This belief supports the quantitative frameworks established by Birkhoff in 1933, which assess and integrate complexity into design, emphasizing a balanced and engaging complexity as essential for effective architectural expression \cite{Birkhoff1933}.
Despite these insights, the architectural field has not fully developed a framework that allows us to leverage these principles in complexity analysis.
Although many tools exist for environmental optimization of buildings, there is still a gap in establishing a balance between complexity and order.
This research seeks to bridge this gap by incorporating a methodology for measuring the complexity outputted during the design phase of building facade design, generating quantifiable data on the order of complexity of iterations of facade design that can be used to improve data-driven optimization models for building design.

To achieve our research objectives, we employ a mixed-methods approach that combines quantitative analysis with qualitative insights, leveraging digital fabrication techniques to create facades distinct from the ornament-bare, post-World War II architecture and urbanism.
The methodology is structured around four primary components:


\begin{enumerate}
    \item Literature review: Significant studies that have assessed aesthetic preferences related to complexity, such as the impact of chaotic patterns and fractal geometry on user perceptions, and an exploration of the fluctuation between simplicity and complexity in architectural history to infer future directions.
    \item Computational Image Complexity Analysis (CICA): created using computer vision algorithms to quantitatively assess the complexity of facade designs, providing data to support theoretical discussions on architectural complexity.
    \item Virtual Reality (VR) Simulation Development: Implements a VR framework for interactive exploration with varied degrees of complexity in facade design, collecting user perceptions and preferences.
    \item Experiment Design: involving VR to facilitate participant interaction with complex facades, augmented by surveys and interviews for qualitative insight.
\end{enumerate}

The research aims to identify the balance between simplicity and complexity in architecture by integrating historical insights with modern technological applications.
We hypothesize that this immersive experience will not only enrich our understanding of facade complexity but also reveal the relationship between user preferences and the evolving trends in the Architectural, Engineering, and Construction (AEC) industry for more elaborate designs.
Therefore, the main aims of this study are as follows:

\begin{itemize}
    \item Evaluate the evolution of architectural styles through historical trends, focusing on the interplay between complexity and simplicity.
    \item Apply the CICA system to quantitatively assess facade complexity across architectural eras.
    \item Employ VR technology to gauge user preferences and tolerance levels for complexity in contemporary facade designs.
    \item Investigate the alignment between user perception and the CICA system evaluation of complexity.
    \item Provide data-driven insights for future architectural design and construction practices.
\end{itemize}

In summary, this research aims to bridge the theoretical understanding of architectural complexity with practical insights gained through contemporary simulation technologies.
By examining architectural complexity through the lens of user preferences, this study seeks to inform and influence the future direction of architectural design and construction practices, ensuring they are aligned with both aesthetic values and functional demands.

%
%!Notes
Beauty is too difficult to translate into number so why bother?
Is beauty really subjective?
Why would we care? there is an

A 2011 survey in the United States found the strongest correlation between a place's physical beauty, and peoples satisfaction out of any other attributes!

Design disconnect. Architects and the public each seem to like different kinds of buildings. This effect was discovered by psychologist David Halpern. Study in 1987.

Beauty is nature way of

According to Denis Dutton, all things that we find beautiful have three things in common: Firstly: they have a shape or characteristics we inherently like.
Secondly, they are fit for their purpose.
And thirdly, they are well-made and display skill in their making.

Organised complexity. As humans, we seem to need some complexity or diversity of form but not too much. Only order is boring, but only complexity is chaos. We seem to like things that are somewhere in the middle. A plain facade too ordered, so we ignore it. A facade like this on the other hand is too

Evidence based design

%!Sustainability
%cause gimmicks, like fashion, get outdated at some point. Many buildings built in the last 50 years already need to be torn down, as they did not have the qualities that made people connect to them. All this renovating and rebuilding requires massive amounts of new concrete, glass and steel. All at a huge cost for society. And, of course, for the environment.


%!Thesis
%we link fractals and organized complexity to environmental optimization to combine them in a theory od data driven beauty environental design. We established that the complexity theory and the vr influence as representative of MR tehconologies serves to built on the aspect of creating a paneling system with organized complexity since it has proven that beauty is perceived by it and we guarantee to add value and improtance to cities based on the papers and video "What makes a building beuatiful".
%!Video article

Architecture:\cite{Aesthetic2022}



Organise your facade in a clear, readable way, so people can make sense of how load bearing features connect to each other
Apply some form of ornament to connect separate parts of the building and to offer the fractal & symmetrical qualities people subconsciously connect to
Prevent the creation of large blank walls or glass facades at eye level. Glass is hard to ‘read’ – people can’t focus their eyes well on it as it is partly reflective, partly translucent.
Apply (local) symmetry in your design. The building can be asymmetrical if building volumes on both sides of a central axis are ‘balanced’
Richly detail the facade if possible, with details on various levels of scale, or utilise materials with some pattern to offer fractal qualities in the surface
Build according to local preferences, history, culture – study the ‘Genius Loci’.
Use curves in your design wherever possible
Urbanism:

Utilise street trees wherever possible
Apply ‘gentle density’: “In contrast to high density, which includes mid- and high-rise residential buildings, gentle density refers to development of duplexes, triplexes, accessory dwelling units (ADUs), stacked townhouses, semi-detached homes and small-scale apartment and condominium buildings within and among single-family zoned neighborhoods. Gentle density promotes mixed-use development, with single-family homes alongside small multi-family homes, businesses and commercial buildings. Gentle density aims to retain neighborhoods’ residential identity and feel while alleviating housing crises.”

modern brain imaging methods have empowered studies on how aesthetic experience
correlates with the activation of brain regions in the human
observer and advanced computational methods permit the investigation of statistical
image properties that are associated with aesthetically pleasing images\cite{Redies2015}

Despite the uncertainties, there seems to be a general agreement in the field that the scien-tific study of aesthetic experience and of beautiful stimuli holds
great promise and may contribute significantly to our under-
standing of human brain function and behavior\cite{Redies2015}

Birkhoff (1933)\cite{Birkhoff1933} first formalized aesthetic value as the ratio of order and complexity.
Complexity, to Birkhoff (1933), was a
physical, measurable characteristic that could be described
mathematically \cite{Bies2016}

Birkhoff\cite{Birkhoff1933} defines an aesthetic measure and applies it to several types of objects with different modes of perception – visual, including
3D objects, and auditory (music, poetry).
The measure is defined in relation to the effort which the object demands of the perceiver (complexity), and the pleasing or displeasing features which can be recognised in the object (order).\cite{Douchova2016}
%!aesthetics

Recent experimental studies began examining the neuroaesthetics of the built envi-
ronment [166–170]. But most offer a perspective almost completely limited to modernist
aesthetics and fail to consider traditional design let alone the fractal mathematics of per-
ception [3,171–180]. \cite{Brielmann2022}

%!Complexity

Complexity & Order is rich sensory information that adheres to a spatial hierarchy similar to those encountered in nature. \cite{Browning2014}

A space with good Complexity & Order feels engaging and information-rich, as an intriguing balance between boring and overwhelming.\cite{Browning2014}

The Complexity & Order pattern has evolved from research on fractal geometries and preferred views; the perceptual and physiological responses to the complexity of fractals in nature, art and architecture; and the predictability of the occurrence of design flows and patterns in nature.\cite{Browning2014}




Researches have found that
The consistency of natural themes in historic structures and places suggests that
biophilic design is not a new phenomenon; rather, as a field of applied science,
it is the codification of history, human intuition and neural sciences showing that
connections with nature are vital to maintaining a healthful and vibrant existence
as an urban species.\cite{Browning2014}

Fractals are patterns that repeat at increasingly fine sizes and so create shapes of rich visual complexity.
Prevalent in nature, clouds, trees, and mountains are common examples, as are cauliflowers and fern leaves.
This paper dives into the complex minutiae of vision and fractals, and how they can place the observer in the “comfort zone”.\cite{Browning2014}

Richly detail the facade if possible, with details on various levels of scale, or utilise materials with some pattern to offer fractal qualities in the surface\cite{Aesthetic2022}

Complexity & Order is rich sensory information that adheres to a spatial hierarchy similar to those encountered in nature. \cite{Browning2014}

A space with good Complexity & Order feels engaging and information-rich, as an intriguing balance between boring and overwhelming.\cite{Browning2014}

The Complexity & Order pattern has evolved from research on fractal geometries and preferred views; the perceptual and physiological responses to the complexity of fractals in nature, art and architecture; and the predictability of the occurrence of design flows and patterns in nature.\cite{Browning2014}

Birkhoff (1933)\cite{Birkhoff1933} first formalized aesthetic value as the ratio of order and complexity.
Complexity, to Birkhoff (1933), was a
physical, measurable characteristic that could be described
mathematically \cite{Bies2016}

Birkhoff\cite{Birkhoff1933} defines an aesthetic measure and applies it to several types of objects with different modes of perception – visual, including
3D objects, and auditory (music, poetry).
The measure is defined in relation to the effort which the object demands of the perceiver (complexity), and the pleasing or displeasing features which can be recognised in the object (order).\cite{Douchova2016}




Traditional architecture knew how to explore \cite{Browning2014}



Despite extensive research in data-driven building design aimed at optimizing resource use and environmental sustainability, many buildings still fail to resonate with the people who inhabit them, leading to premature demolition.
Computational advancements have the potential to transform urban landscapes; however, without addressing the psychological connection between people and their built environment, even sustainably designed buildings may suffer the same fate as their predecessors, cause fashion-like gimmicks become obsolete over time, leading to significant societal and environmental costs\cite{Aesthetic2022}.
Beauty significantly contributes to emotional attachment and happiness in urban areas\cite{Brielmann2022}, yet beauty is largely considered as subjective and relinquished to aesthetic ideologies generating short-fashion urban fabric, however neuroscientists and anthropologists have found certain universal preferences organized in built structures on neurophysiological criteria and measurements that allow to circunvent old and inconclusive debates in aesthetics and architectural history\cite{Brielmann2022}.

Citations by brielmann 2022

The data reviewed here show that we fundamentally need to change how we design
the built environment.
It is time for people to place urban well-being above aesthetic ideology; the durability and long-term reusability of urban fabric above fashion. Changing
built environments—from “old-fashioned” to new, slick, and shiny—is a huge mistake, as people all over the world realize that we need emotional nourishment from the environment around us.
“Old-fashioned” surroundings could likely contain essential healing qualities.
We deserve to walk down a street that is beautiful, as judged by our biology, not by
some imposed stylistic criterion.
We deserve to be protected from the negative psychological reactions caused by experiencing neuroaesthetically inappropriate if not harmful design.
Based on our latest scientific understanding of perception, we need to reclaim
the appropriate and practical design toolkits of past generations to accomplish these ends.
Only then will we be able to build places for the greatest well-being of the greatest number of users.


Yet a new issue arises while extensive research in data-driven building design has been done in pursuit of optimization of resources and environmental engineering, buildings continue to fall out of grace and being demolished by failing to connect with the people that inhabit them.
The computational advancements provide the chance to transform the city scape however if no attention is directed to bring about that attachment with people regardless of it sustainability these new buildings will follow the same lifespan as their predecessors, cause gimmicks like fashion, get outdated at some point All at a huge cost for society. And, of course, for the environment. \cite{Brielmann2022}

