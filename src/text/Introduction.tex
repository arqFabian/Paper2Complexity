%%Introduction

%=================================
%%Reference
%%https://www.scribbr.com/research-paper/research-paper-introduction/
%%State the objectives of the work and provide an adequate background, avoiding a detailed literature survey or a summary of the results.

%Step1. Introduce your topic.
     %This is generally accomplished with a strong opening hook.
%Step2. Describe the background.
     %For a paper describing original research, you’ll instead provide an overview of the most relevant research that has already been conducted.
%Step3. Establish your research problem.
     %In an empirical research paper, try to lead into the problem on the basis of your discussion of the literature.
%Step4. Specify your objective(s).
     %The research question is the question you want to answer in an empirical research paper. If your research involved testing hypotheses, these should be stated along with your research question.
%Step 5: Map out your paper.
     %The final part of the introduction is often dedicated to a brief overview of the rest of the paper.
%=================================

In an era where computer-aided design and digital fabrication are revolutionizing the field of architecture, our study embarks on a timely exploration of society's receptiveness to embracing more intricate and ornamented architectural designs.
This potential shift away from the prevalent uniformity in urban landscapes poses a challenge to the traditional principle that `form follows function'\cite{Gage2015}, suggesting a rekindled interest in architectural complexity and ornamentation.

This research aims to evaluate contemporary receptivity to architectural complexity, leveraging digital fabrication to propose a departure from uniform urban design towards uniquely intricate structures.
We question the extent to which modern audiences are prepared to accept and value such complexity in the built environment.
To achieve our research objectives, we employ a mixed-methods approach that combines quantitative analysis with qualitative insights.
The methodology is structured around four primary components:

\begin{enumerate}
    \item Literature review: to contextualize the fluctuation between simplicity and complexity in architectural history to infer future directions.
    \item Computational Image Complexity Analysis (CICA): created using computer vision algorithms to quantitatively assess the complexity of facade designs, providing data to support theoretical discussions on architectural complexity.
    \item Virtual Reality (VR) Simulation Development: Implements a VR framework for interactive exploration with varied degrees of complexity in facade design, collecting user perceptions and preferences.
    \item Experiment Design: involving VR to facilitate participant interaction with complex facades, augmented by surveys and interviews for qualitative insight.
\end{enumerate}

The research aims to identify the balance between simplicity and complexity in architecture by integrating historical insights with modern technological applications.
We hypothesized that, this immersive experience will not only enrich our understanding of facade complexity but also reveal the relationship between user preferences and the evolving trends in the Architectural, Engineering and Construction (AEC) industry for more elaborate designs.

Therefore, the main aims of this study will be as follows:

\begin{itemize}
    \item Analyze historical architectural trends to discern the oscillation between complexity and simplicity.
    \item Apply the CICA system to quantitatively assess facade complexity across architectural eras.
    \item Utilize VR to measure user tolerance and preference for complexity in facade design.
    \item Investigate the alignment of user preferences with current trends in the AEC industry, especially concerning digital fabrication.
    \item Provide actionable insights on the future of architectural design and construction, guided by empirical data on user preferences and technological capabilities.
\end{itemize}



In summary, this research aims to bridge the theoretical understanding of architectural complexity with practical insights gained through contemporary simulation technologies.
By examining architectural complexity through the lens of user preferences, this study seeks to inform and influence the future direction of architectural design and construction practices, ensuring they are aligned with both aesthetic values and functional demands.



