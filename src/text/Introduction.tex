%%Introduction

%%State the objectives of the work and provide an adequate background, avoiding a detailed literature survey or a summary of the results.

The advent of computer Aid design paired with the already established industrialization of construction, brought in during the creation of the rationalism movement on the 20th century, has provoked a new paradigm shift in the architecture design. The cult of the oversimplification and the ideas of functionality brought by their most canonical representative "Le Corbusier" that birthed cities alienated from the human centric design that had created urban spaces and society \cite{Stacbond2020} in favor for minimalistic approches following an industrial design that ppointed to the machine as the new model for building design \cite{Economakis2023}

I want to describe the opportunity to bring back ornament and identity to the urban space and buildings by the integration of digital fabrication. as a response to the one size fits all state of current building design that stills considers above all eficiency.

Also do a quick timeline of the progress of styles or evolution of paradigm in architecture throughout history with the purpose of showing how simplicity follows complexity and how society responds in a bit of a cyclical way to it.

talk about biomimetics too since might be a comunion between performance and aesthetics while contextualizing architecture.

mention the oportunity of mixed reality to convey all this information to the stakeholders so that they can feel the advantages of great design with the empowering tools of digital fabrication that could reshape the very understanding of common construction with the soon to come democratization of 3d printing houses and other digital fabrication techniques.


Is it possible to reconcile and balance the heuristic approach to problem-solving in Site Layout Planning (SLP) commonly applied in the Architecture, Engineering, and Construction (AEC) industry with the Data-driven Building Design (DBD) strategy? The goal of the DBD strategy is to optimize the outcome of defining the initial schematic model for SLP. Finding a way to integrate these two approaches effectively could lead to more efficient and informed decision-making processes in SLP.

This research aims to explore the potential of virtual reality (VR) simulations in enhancing the assimilation of recommended performance-based design solutions among stakeholders involved in SLP. The study also aims to accommodate stakeholders' input and experiences in the decision-making process.

Is the advent of computer-aided design and digital fabrication serving as a prelude to the revitalization of intricate and ornate design within the social tapestry of our human environment, thus offering an antidote to the monotony of the contemporary urban landscape?

%%%%%% Start of Chatgpt


The emergence of computer-aided design and digital fabrication could herald a resurgence of intricate and ornate architectural design within our societal fabric, potentially offering a remedy to the uniformity plaguing today's urban landscapes.

Upon delving into the annals of influential architectural styles, a discernible pattern emerges—an oscillation between simplicity and complexity. This recurrent cycle in architectural paradigms is not only reflective of the values ingrained in the societies they house, but also closely tied to pivotal technological advancements.

For instance, consider the transition from the robust Romanesque classic style of the 10th century, prominently displayed in churches\cite{Arora2023}, all the way up to the 12th century breakthrough in load distribution with the introduction of buttresses. This innovation propelled churches towards the heavens, enabling the interplay of light to create magnificent stained-glass windows adorned with intricate patterns \cite{Stacbond2020}.
We will see this trend emerge again from the complex Gothic style to the return of Greek and Roman Ideals in the simmetry and perfection focused of the Renaissance style of the 14th century which would again be replaced on the 16th century by the exagerated ornamentation of the Baroque Style, folowed by the neoclassical architecture on the 18th century heavily inspired by classical Greek and Palladian style architecture.
followed by the complex Art Deco in the 1920s and 30s that celebrates techonologial progress with luxurios matterials and patterns hadcrafted with modern technology and design.
Reactive to it on the first half of the 20th century Modern Architecture and rationalism would make its appearance with the maxim "Form follows function"\cite{Gage2015} emphasizing functionalism and minimalistic architecture that showcased new-age materials steel, glass, and concrete. \cite{Arora2023}
This shift encapsulates the essence of architectural evolution—a dynamic interplay between simplicity and complexity, often guided by the confluence of societal values and technological breakthroughs.

The advent of computer-aided design, coupled with the industrialization of construction, ushered in a new paradigm shift in architectural design.
Previously, Rationalism considered as the main architectural style of the 20th century, epitomized by figures like Le Corbusier, emphasized oversimplification and functionality, with focus in simple, symmetrical shapes and concrete as the material of choice, leading to cities alienated from human-centric design in favor of fast transportation, rupturing urban spaces that once defined societies \cite{Stacbond2020}.
As industrial design became the hallmark of future construction \cite{Economakis2023}, architecture embraced minimalism, forsaking ornate beauty and individuality in favor of mass-produced uniformity.

We evidence this when observing the Romanesque classical style on the 10 th century found mostly in churches ith thick wall structures, followed by the breakthrough in load distribution during the Gothic style of the 12th to the 16th century which abandons the robustness to erect high buildings with complex great stained windows.
The Renaissance follows at the end of the middle ages and with it the return of the classical order with round arch and classical order columns that brings ornaments back to the interior in favour of more simplified exteriors. In reaction to this style Baroque style surges with more dynamic forms , irregular shapes and exagerated ornamentation in bold combinations.

In response to this prevailing state of one-size-fits-all building design, this paper aims to explore an opportunity to reintroduce ornament and identity to urban spaces and buildings by embracing the integration of digital fabrication. By harnessing the boundless potential of digital fabrication technology, architecture can transcend the limitations of mass production and rediscover its intricate, awe-inspiring roots.

Main Question of the Paper:

"What is the user response to complex facades created with digital fabrication, measured through mixed reality, and how does it inform future construction trends?"

Goals of the Paper:

Refine understanding of complex Facade Design: To achieve this goal, the paper employs mixed reality exploration as a powerful tool to delve into the realm of complex facade design. By immersing users in virtual environments, the study quantifies user tolerance and acceptance of complex facades. Through this exploration, the paper seeks to gain valuable insights into the factors that influence users' responses, thereby refining the understanding of this intricate facet of architectural design.

Investigate correlations between user preferences and predictions in Shaping Future Construction Trends: By analyzing user preferences and responses to complex facades, the paper endeavors to establish correlations between these insights and predictions for future construction trends. Understanding how users' preferences align with the envisioned future of architecture will provide critical guidance for shaping design principles that resonate with the desires and aspirations of both individuals and societies.

In this quest for reimagining architecture, the concept of biomimetics emerges as an inspiring avenue to explore. Nature's wisdom and ingenuity offer architectural insights that combine performance and aesthetics seamlessly. By learning from the organic structures and systems found in nature, we can create buildings that not only respond effectively to environmental challenges but also captivate with their inherent beauty and elegance.

Realizing the potential of this transformative approach, mixed reality technology provides a gateway to communicate the vision of future architectural design effectively. By immersing stakeholders in a virtual realm where digital fabrication intersects with artistry and functionality, we can instill a shared understanding of the benefits of embracing complexity. This immersive experience empowers stakeholders to witness firsthand the redefinition of conventional construction methods, especially with the imminent democratization of 3D printing houses and other digital fabrication techniques.

In conclusion, this paper embarks on a journey to explore the integration of digital fabrication in architecture as a catalyst for redefining the essence of design.
By embracing complexity and letting go of the confines of mass-produced uniformity, we have an opportunity to breathe new life into our urban spaces, rediscover ornament and identity, and embrace a sustainable aesthetic inspired by the brilliance of nature.
Through mixed reality experiences, we can usher in a future where architecture is a testament to human ingenuity, artistic expression, and harmonious coexistence with our environment.
%%%%%

In the past, Sir Richard Rogers (1989) \cite{Rogers1989}, in his paper entitled “The artist and the Scientist”, highlighted the persistent issue in the AEC field of not keeping pace with science and technology. Efforts have been made to bridge the gap between human-machine interaction, digital fabrication processes, and building performance simulation (BPS) \cite{Naboni2015} in an attempt to harness the full potential of digital technologies and lay the foundation to modernize design and construction in a holistic and integrative computational approach \cite{Knippers2021}. 

To achieve this integration between human and machine, research in data-driven building design, utilizing multi-objective optimization (MOO), seeks to manage trade-offs and tackle the complex problem of building design.

Data-driven design focuses on generating design alternatives, rather than definitive results, from established and known variables, emphasizing the importance of data visualization in describing these variables in the context of each design alternative, particularly during the model development stage \cite{Burton2018}.
However, Seyed et al. \cite{Seyed2022} point out that effective visualization often involves dynamically linked graphs spanning across multiple dimensions, making it challenging to communicate effectively through traditional 2D screens. In this regard, Virtual Reality (VR) applications offer promising potential to bridge technological gaps and more accurately portray data-driven design visualization and analysis. 

Regarding Site Layout Planning (SLP), studies have emphasized the importance of performance-oriented thinking and the role of architects in designing algorithmic processes that lead to stable solutions \cite{AlSaggaf2021, Singh2019, Yang2022, Scheer2014}. Data-driven approaches and BPS have proven effective in enabling more efficient decision-making, but it is crucial to consider input from multiple stakeholders and their heuristic experiences.

Research studies in the field of data-driven building design and building performance simulation (BPS) consistently demonstrate their effectiveness in enabling users to make more efficient decisions \cite{Naboni2015}. However, it is important to note that this was only possible when the model ensure enough flexibility to allow for input from multiple stakeholders and their personal experiences (heuristic approach). While the design team is ultimately responsible for achieving the best solution, the interdisciplinary review process poses the risk of negatively affecting the final outcome of a project. This risk is also relevant in the context of site layout planning (SLP) design, as deviation from the recommended solution given by the data-driven optimized model may occur. According to Hemsath et al. \cite{Hemsath2012}, this can be due to the lack of standardization and structure of the data collected for the project, which impedes quick and effective querying and creates knowledge gaps among team members, increasing the probability of errors. However, the authors suggest that maintaining a higher degree of certainty is possible by querying the data and reviewing the results in an interactive fashion, supported by instantaneous data feedback.

Despite the many studies on performance-based design using multi-objective optimization (MOO) strategies, insufficient attention has been paid to analyzing the impact of deep immersion through virtual reality (VR) simulation and instantaneous feedback rendering data, provided through a heads-up display (HUD) by a Multi-Objective Optimization model could have in potentially mitigate the effects of the traditional heuristic process and achieve an outcome closer to the one predicted by the data-driven optimization approach. And while there has been research into “rendering resulting data in a live heads-up display (HUD) […] with construction scheduling, sequencing, and cost estimating […]” \cite{Hemsath2012} there has been no previous attempt to evaluate the results of a VR immersion on Data-driven Building Design optimization for Site Layout Planning (SLP).

According to the experts \cite{Augenbroe2012}, it is generally agreed that completely eliminating heuristic processes from the building process would be illogical, as in real-life scenarios, the final execution of a Site Layout Planning (SLP) and building as a whole will be primarily driven by heuristics, with formal decision-making informed by performance measures being limited to a small number of predefined decision moments.

Therefore, the question this study decided to explore was related to the potential of VR immersion to efficiently inform stakeholders responsible for the final decision in Site Layout Planning (SLP) design, and to determine whether they would respond more favorably to a Data-driven Design recommendation obtained from a Multi-objective optimization model. We hypothesized that, regardless of their professional background and prior knowledge of SLP design, participants in a VR experiment would be more likely to favor a Data-driven building Design recommendation, and that this would improve the "data connection between their knowledge and the design model" \cite{Augenbroe2012}, reducing the deviation error between the SLP design predicted result by the MOO model and the design ultimately chosen by participants in the experiment.
%%%%%%%%%%%%%%%%%%%%%%%

        

