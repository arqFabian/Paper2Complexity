%%Introduction

%=================================
%%Reference
%%https://www.scribbr.com/research-paper/research-paper-introduction/
%%State the objectives of the work and provide an adequate background, avoiding a detailed literature survey or a summary of the results.

%Step1. Introduce your topic.
     %This is generally accomplished with a strong opening hook.
%Step2. Describe the background.
     %For a paper describing original research, you’ll instead provide an overview of the most relevant research that has already been conducted.
%Step3. Establish your research problem.
     %In an empirical research paper, try to lead into the problem on the basis of your discussion of the literature.
%Step4. Specify your objective(s).
     %The research question is the question you want to answer in an empirical research paper. If your research involved testing hypotheses, these should be stated along with your research question.
%Step 5: Map out your paper.
     %The final part of the introduction is often dedicated to a brief overview of the rest of the paper.

%recommended limit 500 words
%=================================

Recent advancements in Building Information Modeling~(BIM) and digital fabrication are transforming architectural practice.
These technologies enable architects to design intricate and complex forms, moving beyond the uniformity of barren walls and fully glazed facades that often dominate contemporary streetscapes.
By leveraging these advancements, architects can introduce complexity and detail into their designs, enhancing both the visual and functional aspects of buildings, and creating more engaging and dynamic environments that potentially redefine the relationship between form and function~\cite{Leach2016}.

%1. Set up the motivation for the study (importance of facade complexity in design and sustainability)

Understanding facade complexity is crucial, as facades are the most visible part of a building and significantly impact urban aesthetics and user perception.
Designs that balance simplicity and complexity can create environments that are visually stimulating, functional, and comfortable for occupants~\cite{Browning2014}.~Additionally, facades contribute to energy efficiency and material optimization, particularly when combined with advanced technologies like digital fabrication and parametric design.

%2. Discuss the challenge of balancing complexity with sustainability and long-term adaptability:

However, the pursuit of complexity in architectural design must be balanced with sustainability and user satisfaction.
Overly complex designs, when not thoughtfully integrated, can quickly become obsolete, leading to construction waste, a significant contributor to carbon emissions~\cite{Oberfrancova2021}.
By optimizing and controlling facade complexity, architects can create visually engaging designs that are adaptable, long-lasting, and reduce the need for frequent renovations and replacements.

%3. Introduce the research gap and previous work:
Previous research has explored the mathematical relationships between complexity and aesthetic value~\cite{Bies2016, Douchova2016, Redies2015}.
Despite these insights, the architectural field has yet to develop frameworks that leverage these principles for practical design applications, especially considering modern technological advancements such as digital fabrication and parametric design.
These technologies not only enable the creation of complex forms but, when paired with Data-driven Building Design (DBD) optimization, support energy efficiency, material reduction, and sustainability.

%4. Present the study’s objective and contribution:
This study aims to bridge the gap between theory and practice by developing a methodology to measure facade complexity.
The goal is to generate data that enhances DBD through a complexity scoring function, helping to find the optimal balance between simplicity and complexity based on historical analysis and user preferences.
By integrating these insights with modern technologies, we seek to provide actionable, data-driven recommendations for sustainable architectural practices.

%4. Present the study’s hypothesis and methodology:
We hypothesize that by analyzing facade complexity across time and architectural styles through a computational model, a discernible pattern could emerge.
These trends, derived from the analysis of historical data, can then be compared with real-time user perceptions collected via a VR experiment.
By aligning modern user preferences with historical patterns, we aim to validate the model's effectiveness in predicting and assessing facade complexity, offering a framework for informed design decisions.

To support this, a comprehensive literature review was conducted, focusing on foundational theories of complexity and the historical evolution of architectural styles.
By identifying key trends in complexity over time, we can connect these findings with user perception studies to explore how people today interact with complex facades.
Comparing these trends with modern perceptions helps determine whether current preferences align with or diverge from historical patterns.
This review established the theoretical basis for developing our methodology, focusing on the relationship between complexity, aesthetic value, and user satisfaction.

Building upon the conclusions drawn from the literature, the methodology of this study is structured around three core components: the development of the `Complexity Analysis System' using Virtual Reality (VR) and the Computational Image Complexity Analysis (CICA) system, supported by Computer Vision (CV) algorithms, both specifically designed for this research; the `Experiment Execution,' aimed at assessing user perceptions of facade complexity; and a rigorous `Data Analysis' phase to validate the system's effectiveness.
Together, these components create a comprehensive framework for understanding the influence of complexity on architectural design and user satisfaction, through both historical analysis of an image database and a contemporary approach utilizing virtual environments (see Figure~\ref{fig:MethodologyFlowchartComplexity}).
Further details of the methodology can be found in Section~\ref{sec:Methodology}.

%Finally, mention the study’s contribution to sustainability and its practical applications:

This study proposes a system to measure and adjust facade complexity, which could be integrated with tools for energy efficiency, material optimization, and environmental comfort.
Such an approach addresses sustainability challenges while minimizing environmental impact, emphasizing the importance of balancing complexity with long-term adaptability and user satisfaction in modern architectural practices.

This comprehensive approach aims to enrich our understanding of facade complexity and its role in the contemporary Architectural, Engineering, and Construction (AEC) industry.

