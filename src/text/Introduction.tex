%%Introduction

%=================================
%%Reference
%%https://www.scribbr.com/research-paper/research-paper-introduction/
%%State the objectives of the work and provide an adequate background, avoiding a detailed literature survey or a summary of the results.

%Step1. Introduce your topic.
     %This is generally accomplished with a strong opening hook.
%Step2. Describe the background.
     %For a paper describing original research, you’ll instead provide an overview of the most relevant research that has already been conducted.
%Step3. Establish your research problem.
     %In an empirical research paper, try to lead into the problem on the basis of your discussion of the literature.
%Step4. Specify your objective(s).
     %The research question is the question you want to answer in an empirical research paper. If your research involved testing hypotheses, these should be stated along with your research question.
%Step 5: Map out your paper.
     %The final part of the introduction is often dedicated to a brief overview of the rest of the paper.

%recommended limit 500 words
%=================================

Recent advancements in Building Information Modeling (BIM) and digital fabrication are transforming architectural practice.
These technologies enable architects to design intricate and complex forms, moving beyond the uniformity of barren walls and fully glazed facades that often dominate contemporary streetscapes.
By leveraging these advancements, architects can introduce complexity and detail into their designs, enhancing both the visual and functional aspects of buildings, and creating more engaging and dynamic environments that potentially redefine the relationship between form and function~\cite{Leach2016}.

\deleted{
However, the pursuit of complexity in architectural design must be balanced with sustainability and user satisfaction.
Designs that are overly complex without consideration of these factors can quickly become outdated and disconnected from their inhabitants, leading to issues of obsolescence and lack of relevance~\cite{Oberfrancova2021}.
Understanding how complexity can enhance both environmental sustainability and user satisfaction is therefore crucial for modern architectural practice.
}

\added{
However, the pursuit of complexity in architectural design must be balanced with sustainability and user satisfaction.
Overly complex designs, when not thoughtfully integrated, can quickly become outdated, contributing to obsolescence and construction waste, a major source of carbon emissions~\cite{Oberfrancova2021}.
By incorporating complexity analysis into the design process and controlling and optimizing facade complexity, architects can create designs that are not only visually engaging but also adaptable and long-lasting, reducing the need for frequent renovations and replacements.
}

\added{
Understanding the complexity of facade designs is crucial because facades are the most visible part of a building, playing a significant role in urban aesthetics and user perception. Designs that strike the right balance between simplicity and complexity can create environments that are not only visually stimulating but also comfortable and functional for occupants~\cite{Browning2014}. Complexity can enhance the user experience, making facades more engaging and improving user satisfaction with the built environment. Moreover, facade design can contribute to energy efficiency and material optimization, especially when combined with advanced technologies like digital fabrication and parametric design.
}

\added{
This study proposes the development of a system to measure and adjust facade complexity, which could be integrated with existing tools for energy efficiency, material optimization, and environmental comfort.
Such an approach could significantly minimize environmental impact while addressing the sustainability challenges in modern construction.
Understanding how complexity can enhance both environmental sustainability and user satisfaction is therefore crucial for modern architectural practice.
}

\deleted{
Previous research has extensively explored the impact of complexity in architectural design, identifying mathematical relationships between complexity and aesthetic value ~\cite{Bies2016, Douchova2016, Redies2015}.
Despite these insights, the architectural field has yet to develop frameworks that leverage these principles for practical design applications, especially considering modern technological advancements aimed at sustainability.
}

\added{
Previous research has extensively explored the impact of complexity in architectural design, identifying mathematical relationships between complexity and aesthetic value ~\cite{Bies2016, Douchova2016, Redies2015}. Despite these insights, the architectural field has yet to develop frameworks that leverage these principles for practical design applications, especially considering modern technological advancements such as digital fabrication and parametric design. These technologies not only enable the creation of complex forms but when paired with `Data-driven Building Design' (DBD) optimization they also support energy efficiency, material reduction, and long-term sustainability.
}

This study aims to bridge the gap between theoretical understanding and practical application by developing a methodology to measure facade complexity.
The objectives are to generate data that can improve DBD by integrating a complexity scoring function that can inform on the optimal rate between simplicity and complexity based on historical analysis and user preferences.
By integrating complexity insights with modern technological applications, we seek to provide actionable, data-driven insights for future architectural practices promoting the advancement aimed at sustainability.

\deleted{The methodology is structured around 4 primary components:}

\deleted{- Literature review: Significant studies on the foundational theories of complexity, and an exploration of the fluctuation between simplicity and complexity in architectural history.}

%deleted list
    %\begin{itemize}
        \deleted{**Literature review: Significant studies on the foundational theories of complexity, and an exploration of the fluctuation between simplicity and complexity in architectural history.}

        \deleted{**Complexity Analysis System Development: Implements a Virtual Reality (VR) framework, and combines it with a Computational Image Complexity Analysis (CICA) component using computer vision (CV) algorithms to quantitatively assess the complexity of facade designs.}

        \deleted{**Experiment Execution: involving VR to facilitate participant interaction with complex facades, augmented by surveys and interviews for qualitative insight.}

        \deleted{**Data Analysis and Validation: Assessing the data collected during the experiment to evaluate the effectiveness of the Complexity Analysis System and CICA framework in measuring complexity and user preferences.}
    %\end{itemize}

\added{
A comprehensive literature review was conducted to examine significant research on the foundational theories of complexity, as well as the historical shifts between simplicity and complexity in architectural design.
By analyzing historical architectural styles and their evolving complexity, this study identifies key patterns and trends that inform contemporary approaches to facade design.
The integration of historical analysis with human perception studies allows for a deeper exploration of how users today interact with complex architectural designs. By comparing the complexity of historical facades with contemporary designs, we can investigate whether modern perceptions align with or diverge from historical trends.
This review established a solid theoretical foundation for the study and informed the research focus on understanding how complexity affects both aesthetic value and user satisfaction. These insights laid the groundwork for the development of the study’s methodology.
}

\added{
Building upon the conclusions drawn from the literature, the methodology of this study is structured around three core components: The development of the `Complexity Analysis System' using Virtual Reality (VR) and the Computational Image Complexity Analysis (CICA) system, supported by Computer Vision (CV) algorithms, both specifically designed for this research; the `Experiment Execution,' aimed at assessing user perceptions of facade complexity; and a rigorous `Data Analysis' phase to validate the system's effectiveness. Together, these components create a comprehensive framework for understanding the influence of complexity on architectural design and user satisfaction, through both historical analysis of an image database and a contemporary approach utilizing virtual environments (see Figure~\ref{fig:MethodologyFlowchartComplexity}).
Further details of the methodology can be found in Section~\ref{sec:Methodology}.
}

This comprehensive approach aims to enrich our understanding of facade complexity and its role in the contemporary Architectural, Engineering, and Construction (AEC) industry.

