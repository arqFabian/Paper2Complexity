%%Introduction

%%State the objectives of the work and provide an adequate background, avoiding a detailed literature survey or a summary of the results.

Is it possible to reconcile and balance the heuristic approach to problem-solving in Site Layout Planning (SLP) commonly applied in the Architecture, Engineering, and Construction (AEC) industry with the Data-driven Building Design (DBD) strategy? The goal of the DBD strategy is to optimize the outcome of defining the initial schematic model for SLP. Finding a way to integrate these two approaches effectively could lead to more efficient and informed decision-making processes in SLP.

This research aims to explore the potential of virtual reality (VR) simulations in enhancing the assimilation of recommended performance-based design solutions among stakeholders involved in SLP. The study also aims to accommodate stakeholders' input and experiences in the decision-making process.

In the past, Sir Richard Rogers (1989) \cite{Rogers1989}, in his paper entitled “The artist and the Scientist”, highlighted the persistent issue in the AEC field of not keeping pace with science and technology. Efforts have been made to bridge the gap between human-machine interaction, digital fabrication processes, and building performance simulation (BPS) \cite{Naboni2015} in an attempt to harness the full potential of digital technologies and lay the foundation to modernize design and construction in a holistic and integrative computational approach \cite{Knippers2021}. 

To achieve this integration between human and machine, research in data-driven building design, utilizing multi-objective optimization (MOO), seeks to manage trade-offs and tackle the complex problem of building design.

Data-driven design focuses on generating design alternatives, rather than definitive results, from established and known variables, emphasizing the importance of data visualization in describing these variables in the context of each design alternative, particularly during the model development stage \cite{Burton2018}.
However, Seyed et al. \cite{Seyed2022} point out that effective visualization often involves dynamically linked graphs spanning across multiple dimensions, making it challenging to communicate effectively through traditional 2D screens. In this regard, Virtual Reality (VR) applications offer promising potential to bridge technological gaps and more accurately portray data-driven design visualization and analysis. 

Regarding Site Layout Planning (SLP), studies have emphasized the importance of performance-oriented thinking and the role of architects in designing algorithmic processes that lead to stable solutions \cite{AlSaggaf2021, Singh2019, Yang2022, Scheer2014}. Data-driven approaches and BPS have proven effective in enabling more efficient decision-making, but it is crucial to consider input from multiple stakeholders and their heuristic experiences.

Research studies in the field of data-driven building design and building performance simulation (BPS) consistently demonstrate their effectiveness in enabling users to make more efficient decisions \cite{Naboni2015}. However, it is important to note that this was only possible when the model ensure enough flexibility to allow for input from multiple stakeholders and their personal experiences (heuristic approach). While the design team is ultimately responsible for achieving the best solution, the interdisciplinary review process poses the risk of negatively affecting the final outcome of a project. This risk is also relevant in the context of site layout planning (SLP) design, as deviation from the recommended solution given by the data-driven optimized model may occur. According to Hemsath et al. \cite{Hemsath2012}, this can be due to the lack of standardization and structure of the data collected for the project, which impedes quick and effective querying and creates knowledge gaps among team members, increasing the probability of errors. However, the authors suggest that maintaining a higher degree of certainty is possible by querying the data and reviewing the results in an interactive fashion, supported by instantaneous data feedback.

Despite the many studies on performance-based design using multi-objective optimization (MOO) strategies, insufficient attention has been paid to analyzing the impact of deep immersion through virtual reality (VR) simulation and instantaneous feedback rendering data, provided through a heads-up display (HUD) by a Multi-Objective Optimization model could have in potentially mitigate the effects of the traditional heuristic process and achieve an outcome closer to the one predicted by the data-driven optimization approach. And while there has been research into “rendering resulting data in a live heads-up display (HUD) […] with construction scheduling, sequencing, and cost estimating […]” \cite{Hemsath2012} there has been no previous attempt to evaluate the results of a VR immersion on Data-driven Building Design optimization for Site Layout Planning (SLP).

According to the experts \cite{Augenbroe2012}, it is generally agreed that completely eliminating heuristic processes from the building process would be illogical, as in real-life scenarios, the final execution of a Site Layout Planning (SLP) and building as a whole will be primarily driven by heuristics, with formal decision-making informed by performance measures being limited to a small number of predefined decision moments.

Therefore, the question this study decided to explore was related to the potential of VR immersion to efficiently inform stakeholders responsible for the final decision in Site Layout Planning (SLP) design, and to determine whether they would respond more favorably to a Data-driven Design recommendation obtained from a Multi-objective optimization model. We hypothesized that, regardless of their professional background and prior knowledge of SLP design, participants in a VR experiment would be more likely to favor a Data-driven building Design recommendation, and that this would improve the "data connection between their knowledge and the design model" \cite{Augenbroe2012}, reducing the deviation error between the SLP design predicted result by the MOO model and the design ultimately chosen by participants in the experiment.
%%%%%%%%%%%%%%%%%%%%%%%

        

