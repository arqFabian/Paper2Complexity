%% Research Background

%% A Theory section should extend, not repeat, the background to the article already dealt with in the Introduction and lay the foundation for further work. In contrast, a Calculation section represents a practical development from a theoretical basis.


The act of building our environment has its roots in our attempt to emulate nature. As Baker \cite{Baker2003} defines it, architecture creates a universe crafted by humans for humans. While buildings may originate from nature and try to blend with their surroundings, they remain distinct entities in the continuous environment, serving as private domains dedicated to humanity. However, the challenges of the future, as pointed out by Knippers et al. \cite{Knippers2021}, lie in constructing more buildings while minimizing pollutant emissions, conserving non-renewable resources, and ensuring the creation of high-quality and sustainable living spaces.

Regardless of the scale of the building project, whether it be a single housing unit or a towering skyscraper, Site Layout Planning (SLP) design remains a crucial and acknowledged step in the construction process across the entire Architecture, Engineering, and Construction (AEC) field. 

        %The act of building our environment started with our attempt to emulate nature and as Baker \cite{Baker2003} defined it, a universe created by man for man, and while it may have originated from nature in the semblance of the world it is by itself a separate entity from the otherwise continuous environment, and even though a building can attempt to blend with its surroundings it remains as a private domain dedicated to humanity. Knippers et al. \cite{Knippers2021} believe that the challenge for building in the future will be to build more, while emitting fewer pollutants, consuming less non-renewable resources, and still ensuring the outcome of a high-quality and liveable built environment. 
        
        

\subsection{Site Layout Planning}
\label{subsec:SLP}
As described by Ning \cite{Ning2011}, SLP involves a decision-making process encompassing problem identification, recognizing opportunities, developing solutions, selecting the best alternatives, and executing them. This critical phase sets the foundation for successful building construction and plays a pivotal role in achieving optimized outcomes that balance various design objectives, economic factors, and environmental considerations.
        %The process to build successfully a building includes several stages and regardless of the scale of that endeavor, be it in the form of a single housing unit or the planning of a skyscraper, Site Layout Planning (SLP) design continues to be recognized as a critical step in the construction process all across the AEC field. Ning \cite{Ning2011} defines SLP as a “decision-making process, which involves identifying problems and opportunities, developing solutions, choosing the best alternative and implementing it.” 

Kulabi et al. \cite{Kulabi2020} identified that the problem of Site Layout Planning (SLP) has been solved by two main approaches which are heuristic methods and mathematical optimization. 
Heuristic methods depend on knowledge-based systems and they are adopted to give good but not optimal results \cite{Kulabi2020}. Traditionally the construction industry has established rules and regulations to help establish these guidelines to achieve a successful Site Layout Planning design and in real life, the execution of most projects continues to be guided by heuristics   \cite{Augenbroe2012} and is done based on the experience of planners, and as a consequence, the level of accuracy and reliability vary among different projects \cite{Yi2018}.

Mathematical optimization methods appear to improve the current circumstances with the objective of improving the reliability of the results and a more stable outcome across field striving to achieve optimal results, however, Kulabi et al. \cite{Kulabi2020} explains that due to the complexity in computations derived by large projects, mathematical optimizations are not being applied to them and heuristic methods dominate on these projects. 

\subsection{Data-driven Building Design Optimization}
\label{subsec:PBD}

The continuous research towards mathematical optimization methods proves the allure of the promise for optimal design. The recurrent venture into this method has created a trend in data-driven building design optimization where a solution to a building project is no longer a single static result instead it is seen as a process from which an assort of solutions can be presented with various degrees of optimization, while still given the final choice to the stakeholders.

It is understood that traditionally design solutions were driven by prescriptive terms, rather than the expected performance of the solution, with building codes and regulations being the main contributors to prescriptive specifications \cite{Augenbroe2012} and as Hemsath \cite{Hemsath2012} argues, data-driven design, such as Performance-based Design, requires a deliberate approach with a focus on front-loading information and reducing the time between critical feedback.

Aside from the data which sits at the core of the data-driven building design, an accurate building performance simulation is needed to support the design evolution, not only for reviewing purposes but on the role of a virtual experiment \cite{Augenbroe2012}.

An example that explores the data-driven strategy in the field of Site Layout Planning (SLP) is given by Yi et al. \cite{Yi2018} who propose a mathematical model to solve the problem of allocating the temporary facilities needed for a construction Site Layout considering a multi-objective optimization approach that covers several safety, health and environmental concerns as well as transportation costs. Leveraging on the potential of computation to solve a non-linear optimization model that deals with different criteria, the team verified that compared to traditional heuristic methods the proposed optimization mathematical model could outperform the traditional approach improvement up to 19\% even when they tested it on a real scenario, proving the advantages of data-driven design optimization. 

The benefits of this approach in practice, recognize “that predicting and analyzing building behavior is more efficient and economical than resolving these issues once the building is built” \cite{Hemsath2012}. These benefits extend to Site Layout Planning.

Beyond the analysis of the strategies to solve the process of Site Layout Planning, it is important to recognize the way a building is considered in relation to its site, what Baker \cite{Baker2003} referred to as the site forces (which include orientation, views, access), and how they interact with the organizational forces identified within the building.

To include these variants implied for this exercise a combination between performance-based design and simulation with the traditional heuristic approach with a renewed interaction that takes advantage of Virtual reality as a new frontier, with the purpose of accelerating the adoption of data-driven design by the general public. 

\subsection{Virtual Reality (VR) and its impact on the AEC field}
\label{subsec:VR and AEC}

The Architecture, Engineering, and Construction (AEC) industry are striving to keep pace with the rapid advancements in technology \cite{Moonhubwebsite}. Despite innovations in building materials and existing CAD tools, the industry continues to face hazards and inefficiencies, making it increasingly challenging to attract new workers, leading to a labor shortage. For instance, countries like the UK are experiencing a record high of 37,000 job vacancies in the construction sector \cite{ConstructionVacancies}. In this competitive landscape, AEC firms that invest in modernization and embrace the latest technologies position themselves ahead of the competition due to their adaptability \cite{Alizadehsalehi2020}.

    %The Architecture, Engineering and Construction (AEC) industry is also trying to keep up with the current advancements of technology \cite{Moonhubwebsite}, regardless of the innovation in building materials and existing CAD tools the work in the AEC industry remains hazardous, plagued with inefficiencies which have made so that continuously more difficult to attract new workers creating the current environment of labor shortage which in countries like UK reach a new record high of 37000 vacancies\cite{ConstructionVacancies}. That is why the AEC firms that invest in modernization and continuously embrace the latest technologies  will position themselves ahead of the competition due to their willingness to adapt \cite{Alizadehsalehi2020}.  

To address safety, labor, and efficiency issues, the AEC industry is actively exploring various strategies. A pivotal technology in this pursuit is Virtual Reality (VR), as highlighted by a 2016 poll conducted by ARC Document Solutions \cite{ARCdocument}, where 65.7\% of respondents recognized VR as the most critical technology impacting key aspects of the industry. As Schnabel \cite{Schnabel2009} accurately points out, the emergence of VR and other forms of Mixed Reality will have a profound impact on the way we perceive the world, where reality redefines itself anew and will stand as one realm among others on the spectrum ranging from reality to virtuality. 
        %And as Schnabels \cite{Schnabel2009} accurately points out, the influence of the emergence of VR and other forms of Mixed Reality, will have a profound impact on the way we perceive the world, where reality redefines itself anew and will stand as one realm among others on the spectrum ranging from reality to virtuality.
        
VR offers a range of unique advantages, including enhanced visualization, faster project completion, reduced labor requirements, and minimized material usage. One of the most remarkable features of VR technology is its ability to provide an immersive experience of a building before it is physically constructed—a capability unmatched by traditional schematics or concept renderings \cite{Moonhubwebsite}. These computer-generated realities merge with the perception of reality and are specifically tailored to enhance and communicate key aspects guiding a building's design \cite{Schnabel2009}\cite{wang-2009}. By immersing stakeholders in a virtual environment, VR allows for interaction with designed spaces at a human scale, and facilitates a deeper understanding of building designs, enabling more informed decision-making and efficient project execution \cite{Castronovo2013}. As the AEC industry embraces VR technology, it becomes better equipped to address its challenges, drive innovation, and revolutionize the way buildings are designed and constructed. 
    
An example of this technology is the "VR smart" tool \cite{Wolfartsberger2019}, a VR-based tool created to support engineering design review. The “VR smart” team evaluated the usability of this app by comparing it to the mainstream approach that uses 2D documentations paired with a PC that shows on-screen CAD data. The “VR smart” app was then tested in a real industrial environment and an authentic engineering review. The results show that users of the VR system could identify slightly more errors in a design and the interaction with the system showed a much faster entry into the design review, which aligns with the evidence that the information becomes more accessible to groups with less knowledge of engineering documentation. 

For the specific case of Site Layout Planning (SLP), there have also been research studies on integrating Virtual reality into solving the SLP problem. A recent example of this was given by  Muhammad et al. \cite{Muhammad2020} that developed a VR simulation of a site with the intention of evaluating job site organization for Site Layout Planning, collision detection and assessment of construction site layout arrangement with the final purpose of evaluating the impact that adding VR technology has on solving the SLP problem and, in a similar fashion as the “VR smart” team, compare it to the traditional approach of 2D documentation review by conducting a survey pos-testing the simulation. The results showed that while VR accelerated the rate of comprehension of the participants regarding the Site Layout and increased the identification of collision points, it also showed that traditional 2D methods are easier to understand and less time-consuming when dealing with tasks that require a bigger overview of the whole site such as Site Layout Planning, however, they denoted that this preference could also be due to the inexperience of the participants when using VR headsets.

The authors of the previous examples \cite{Wolfartsberger2019} \cite{Muhammad2020} have confirmed the relevance that VR will have as a new tool to help the way that we interact and approach a project. Therefore, the possibility of harvesting the potential of VR technology in facilitating the comprehension of complex sets of data represented as responsive graphs with immediate feedback could prove to be the path toward embracing data-driven design optimization.