%!Asbtract guidelines
        %A concise and factual abstract (100-150 words ) is required. The abstract should state briefly the purpose of the research, the principal results and major conclusions. An abstract is often presented separately from the article, so it must be able to stand alone. For this reason, References should be avoided, but if essential, then cite the author(s) and year(s). Also, non-standard or uncommon abbreviations should be avoided, but if essential they must be defined at their first mention in the abstract itself.

This paper examines user perceptions and acceptance of complex facade designs in contemporary architecture, integrating digital fabrication, virtual reality (VR), and computer vision.
Our research aims to shed light on the evolving trends in construction design.
A literature review reveals a historical oscillation between simplicity and complexity in architectural styles.
We introduce the `Computational Image Complexity Analysis' (CICA) method to quantitatively evaluate the complexity of building facades.
This approach verifies a noticeable trend towards greater complexity since the late 20th century.
In our VR experiment, participants evaluated various facades, expressing a preference for designs with a CICA complexity score of 3.82 out of 10.
Subsequent surveys indicated a positive reception towards intricate designs, with an average rating of 4.9 on a 7-point scale.
These outcomes align with our architectural analysis, indicating a continuing trend towards increased complexity in modern architectural design.

% check