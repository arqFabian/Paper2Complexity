This paper presents an exploration of pioneering approaches in architectural design, encompassing the convergence of digital fabrication, data-driven optimization, and mixed reality assessment to advance the comprehension of complex facade design. The primary aim of this study is to assess the tolerance and acceptance of users towards intricate facade designs, thereby providing valuable insights into the potential trajectories of future construction practices, specifically with regard to the incorporation of complex facades. Through the utilization of mixed reality exploration, we embark on an endeavor to quantify user responses to complex facades, shedding light on their levels of tolerance and acceptance. By employing this innovative methodology, we seek to refine our understanding of the intricate interplay between architectural design and human perception.

The outcomes of this research, driven by the integration of digital fabrication, data-driven optimization, and mixed reality assessment, offer a holistic perspective on the evolution of complex facade design.
On average the results indicate that participants favor a complexity, view rate of \(XX.X\%\) and the aesthetics during a post survey reflect this composition to be  highly regarded by participants, receiving an average acceptance rate of XX on a 7-point Likert scale.
The results indicate that the utilization of the MR simulation significantly enhances the understanding and acceptance of the recommended data-driven optimized design solutions.

        %This research focuses on assessing the effectiveness of virtual reality (VR) simulations in promoting the acceptance of data-driven optimized design solutions for Site Layout Planning (SLP). To achieve this, a multi-objective optimization model was developed, taking into account important factors such as earthwork volume calculations, cost, and environmental impact, particularly for an educational building situated on the site. The model was then transformed into an interactive VR simulation, allowing participants to visually observe the real-time consequences of different building placements on the site. 
        %The findings of the research indicate that the utilization of the VR simulation significantly enhanced the understanding and acceptance of the recommended data-driven optimized design solutions. On average, participants experienced a notable \(48.3\%\) increase in accuracy when compared to traditional methods. The technology itself was highly regarded by participants, receiving an average acceptance rate of 5.4, across several levels, on a 7-point Likert scale.
        %It is important to note that while VR simulations show promise in expediting the adoption of data-driven optimized design solutions, they are not meant to replace existing design review methods. Instead, they serve as a means to streamline the decision-making process and provide stakeholders with a more immersive and comprehensive understanding of the design solutions.
        
