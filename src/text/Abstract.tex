This research aims to evaluate the efficacy of virtual reality (VR) simulations in promoting the acceptance of data-driven optimized design solutions for Site Layout Planning (SLP). The study involves the development of a multi-objective optimization model, considering essential factors such as earthwork volume calculations, cost, and environmental impact, particularly for an educational building located on the site. The model is subsequently transformed into an interactive VR simulation, enabling participants to visually observe the real-time impact of various building placements on the site. The results indicate that the utilization of the VR simulation significantly enhances the understanding and acceptance of the recommended data-driven optimized design solutions. On average, participants experienced a notable \(48.3\%\) increase in accuracy compared to traditional methods. The technology was highly regarded by participants, receiving an average acceptance rate of 5.4 on a 7-point Likert scale. It is essential to recognize that while VR simulations show promise in expediting the adoption of data-driven optimized design solutions, they are not intended to replace existing design review methods. Instead, they serve as a means to streamline the decision-making process and provide stakeholders with a more immersive and comprehensive understanding of the design solutions.

        %This research focuses on assessing the effectiveness of virtual reality (VR) simulations in promoting the acceptance of data-driven optimized design solutions for Site Layout Planning (SLP). To achieve this, a multi-objective optimization model was developed, taking into account important factors such as earthwork volume calculations, cost, and environmental impact, particularly for an educational building situated on the site. The model was then transformed into an interactive VR simulation, allowing participants to visually observe the real-time consequences of different building placements on the site. 
        %The findings of the research indicate that the utilization of the VR simulation significantly enhanced the understanding and acceptance of the recommended data-driven optimized design solutions. On average, participants experienced a notable \(48.3\%\) increase in accuracy when compared to traditional methods. The technology itself was highly regarded by participants, receiving an average acceptance rate of 5.4, across several levels, on a 7-point Likert scale.
        %It is important to note that while VR simulations show promise in expediting the adoption of data-driven optimized design solutions, they are not meant to replace existing design review methods. Instead, they serve as a means to streamline the decision-making process and provide stakeholders with a more immersive and comprehensive understanding of the design solutions.
        
