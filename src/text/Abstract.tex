%!Asbtract guidelines
        %A concise and factual abstract (100-150 words ) is required. The abstract should state briefly the purpose of the research, the principal results and major conclusions. An abstract is often presented separately from the article, so it must be able to stand alone. For this reason, References should be avoided, but if essential, then cite the author(s) and year(s). Also, non-standard or uncommon abbreviations should be avoided, but if essential they must be defined at their first mention in the abstract itself.

% check
% new version under reviewer guidelines: Abstracts should contain only 6 short sentences: 1) What is the problem being addressed? 2) What is the research question being asked? 3) What is the methodology being used to answer the stated research question? 4) What are the results obtained? 5) What is the meaning and importance of these results? 6) What are the directions for follow-up research?

Architectural practice is evolving through digital fabrication, enabling complex designs that challenge the uniformity of barren walls and fully glazed facades that often dominate contemporary streetscapes.
This paper addresses the challenge of quantifying complexity in architectural facade design.
It asks whether a Virtual Reality (VR) and Computer Vision (CV) approach can effectively measure facade complexity and align with user perceptions.
The study employs the Computational Image Complexity Analysis (CICA) system, integrating VR and CV algorithms, to assess reactions to various facade complexities.
Results reveal an average standard deviation of 9\% between the system's complexity measurements and participants' perceptions, with a preference for moderate complexity.
These findings highlight the importance of aligning architectural complexity with user preferences to enhance sustainability and satisfaction and the potential of this approach to quantify complexity and guide data-driven building design processes.
Future research should explore the long-term impact of complex facades on user well-being and environmental sustainability.



