%!Asbtract guidelines
        %A concise and factual abstract (100-150 words ) is required. The abstract should state briefly the purpose of the research, the principal results and major conclusions. An abstract is often presented separately from the article, so it must be able to stand alone. For this reason, References should be avoided, but if essential, then cite the author(s) and year(s). Also, non-standard or uncommon abbreviations should be avoided, but if essential they must be defined at their first mention in the abstract itself.



% check
This research uses virtual reality (VR) assessment, and computer vision to understand complexity in architectural facade design.
We aim to examine user tolerance and acceptance of complex facades, providing insights for future construction practices.
A literature review confirms a contemporary trend towards increased facade complexity, moving away from modernist minimalism.
We introduce the `Computational Image Complexity Analysis' (CICA) system to quantify this trend, revealing an upward complexity trajectory since the late 20th century.
A VR experiment indicates a user preference for moderate complexity, suggesting a balance between intricacy and simplicity in future architectural trends.
Discrepancies between participant perceptions and CICA rankings highlight the subjective nature of complexity perception.
Qualitative data suggests a shift towards customizable, user-responsive designs.
Overall, the study underscores a contemporary shift towards embracing complexity in facade design, emphasizing the need for a balanced approach that aligns with user preferences and cultural contexts.

%previous
%This paper examines user perceptions and acceptance of complex facade designs in contemporary architecture, integrating digital fabrication, virtual reality (VR), and computer vision.
%Our research aims to shed light on the evolving trends in construction design.
%A literature review reveals a historical oscillation between simplicity and complexity in architectural styles.
%We introduce the `Computational Image Complexity Analysis' (CICA) method to quantitatively evaluate the complexity of building facades.
%This approach verifies a noticeable trend towards greater complexity since the late 20th century.
%In our VR experiment, participants evaluated various facades, expressing a preference for designs with a CICA complexity score of 3.82 out of 10.
%Subsequent surveys indicated a positive reception towards intricate designs, with an average rating of 4.9 on a 7-point scale.
%These outcomes align with our architectural analysis, indicating a continuing trend towards increased complexity in modern architectural design.

%!Notes
Beauty is too difficult to tranlate into number so why bother?
Is beauty really subjective?
Why would we care? there is an

A 2011 survey in the United States found the strongest correlation between a place's physical beauty, and peoples satisfaction out of any other attributes!

Design disconnect. Architects and the public each seem to like different kinds of buildings. This effect was discovered by psychologist David Halpern. Study in 1987.

Beauty is nature way of

According to Denis Dutton, all things that we find beautiful have three things in common: Firstly: they have a shape or characteristics we inherently like. Secondly, they are fit for their purpose. And thirdly, they are well-made and display skill in their making.

Organised complexity. As humans, we seem to need some complexity or diversity of form but not too much. Only order is boring, but only complexity is chaos. We seem to like things that are somewhere in the middle. A plain facade too ordered, so we ignore it. A facade like this on the other hand is too

Evidence based design

%!Sustainability
%cause gimmicks, like fashion, get outdated at some point. Many buildings built in the last 50 years already need to be torn down, as they did not have the qualities that made people connect to them. All this renovating and rebuilding requires massive amounts of new concrete, glass and steel. All at a huge cost for society. And, of course, for the environment.


%!Thesis
%we link fractals and organized complexity to environmental optimization to combine them in a theory od data driven beauty environental design. We established that the complexity theory and the vr influence as representative of MR tehconologies serves to built on the aspect of creating a paneling system with organized complexity since it has proven that beuaty is perceived by it and we guarantee to add value and improtance to cities based on the papers and video "What makes a building beuatiful".
%!Video article

Architecture:\cite{Aesthetic2022}

Organise your facade in a clear, readable way, so people can make sense of how load bearing features connect to each other
Apply some form of ornament to connect separate parts of the building and to offer the fractal & symmetrical qualities people subconsciously connect to
Prevent the creation of large blank walls or glass facades at eye level. Glass is hard to ‘read’ – people can’t focus their eyes well on it as it is partly reflective, partly translucent.
Apply (local) symmetry in your design. The building can be asymmetrical if building volumes on both sides of a central axis are ‘balanced’
Richly detail the facade if possible, with details on various levels of scale, or utilise materials with some pattern to offer fractal qualities in the surface
Build according to local preferences, history, culture – study the ‘Genius Loci’.
Use curves in your design wherever possible
Urbanism:

Utilise street trees wherever possible
Apply ‘gentle density’: “In contrast to high density, which includes mid- and high-rise residential buildings, gentle density refers to development of duplexes, triplexes, accessory dwelling units (ADUs), stacked townhouses, semi-detached homes and small-scale apartment and condominium buildings within and among single-family zoned neighborhoods. Gentle density promotes mixed-use development, with single-family homes alongside small multi-family homes, businesses and commercial buildings. Gentle density aims to retain neighborhoods’ residential identity and feel while alleviating housing crises.”