%!Asbtract guidelines
        %A concise and factual abstract (100-150 words ) is required. The abstract should state briefly the purpose of the research, the principal results and major conclusions. An abstract is often presented separately from the article, so it must be able to stand alone. For this reason, References should be avoided, but if essential, then cite the author(s) and year(s). Also, non-standard or uncommon abbreviations should be avoided, but if essential they must be defined at their first mention in the abstract itself.



% check
% new version under reviewer guidelines: Abstracts should contain only 6 short sentences: 1) What is the problem being addressed? 2) What is the research question being asked? 3) What is the methodology being used to answer the stated research question? 4) What are the results obtained? 5) What is the meaning and importance of these results? 6) What are the directions for follow-up research?


Architectural practice is evolving through digital fabrication, enabling complex designs that challenge the uniformity of barren walls and fully glazed facades that often dominate contemporary streetscapes.
This study investigates user tolerance and acceptance of complex facades using Virtual Reality (VR) and Computer Vision (CV) through the Computational Image Complexity Analysis (CICA) system.
We applied VR simulations and CICA to quantitatively and qualitatively assess reactions to various facade complexities.
Results reveal a preference for moderate complexity, suggesting an ideal balance between simplicity and intricacy.
This highlights the importance of aligning architectural complexity with user preferences to enhance sustainability and satisfaction.
Future research should explore the long-term impact of complex facades on user well-being and environmental sustainability.


%version 2024/05
%This research uses virtual reality (VR) assessment and computer vision to explore complexity in architectural facade design.
%We aim to examine user tolerance and acceptance of complex facades, providing insights for future construction practices.
%A literature review confirms a trend towards increased complexity, preferring richly detailed facades with elements at various scales or materials with fractal qualities, diverging from modernist minimalism.
%We introduce the Computational Image Complexity Analysis (CICA) system to quantify this trend, revealing an upward complexity trajectory since the late 20th century.
%A VR experiment indicates a user preference for moderate complexity, suggesting a balance between intricacy and simplicity.
%Discrepancies between participant perceptions and CICA rankings highlight the subjective nature of complexity perception.
%Qualitative data suggests a shift towards customizable, user-responsive designs.
%Overall, the study underscores a shift towards embracing complexity in facade design, emphasizing the need for a balanced approach that aligns with user preferences and cultural contexts.

% version 2024/04
%This research uses virtual reality (VR) assessment, and computer vision to understand complexity in architectural facade design.
%We aim to examine user tolerance and acceptance of complex facades, providing insights for future construction practices.
%A literature review confirms a contemporary trend towards increased facade complexity, moving away from modernist minimalism.
%We introduce the `Computational Image Complexity Analysis' (CICA) system to quantify this trend, revealing an upward complexity trajectory since the late 20th century.
%A VR experiment indicates a user preference for moderate complexity, suggesting a balance between intricacy and simplicity in future architectural trends.
%Discrepancies between participant perceptions and CICA rankings highlight the subjective nature of complexity perception.
%Qualitative data suggests a shift towards customizable, user-responsive designs.
%Overall, the study underscores a contemporary shift towards embracing complexity in facade design, emphasizing the need for a balanced approach that aligns with user preferences and cultural contexts.



%previous
%This paper examines user perceptions and acceptance of complex facade designs in contemporary architecture, integrating digital fabrication, virtual reality (VR), and computer vision.
%Our research aims to shed light on the evolving trends in construction design.
%A literature review reveals a historical oscillation between simplicity and complexity in architectural styles.
%We introduce the `Computational Image Complexity Analysis' (CICA) method to quantitatively evaluate the complexity of building facades.
%This approach verifies a noticeable trend towards greater complexity since the late 20th century.
%In our VR experiment, participants evaluated various facades, expressing a preference for designs with a CICA complexity score of 3.82 out of 10.
%Subsequent surveys indicated a positive reception towards intricate designs, with an average rating of 4.9 on a 7-point scale.
%These outcomes align with our architectural analysis, indicating a continuing trend towards increased complexity in modern architectural design.


