%!Asbtract guidelines
        %A concise and factual abstract (100-150 words ) is required. The abstract should state briefly the purpose of the research, the principal results and major conclusions. An abstract is often presented separately from the article, so it must be able to stand alone. For this reason, References should be avoided, but if essential, then cite the author(s) and year(s). Also, non-standard or uncommon abbreviations should be avoided, but if essential they must be defined at their first mention in the abstract itself.

This paper explores complex facade design in contemporary architecture using digital fabrication, virtual reality, and computer vision.
We aim to understand how users perceive and accept intricate facades, offering insights into future construction trends.
A literature review confirms a growing complexity trend in architecture.
We introduce the `Computational Image Complexity Analysis' (CICA) system to quantitatively assess historical building complexity, revealing a consistent rise since the late 20th century.

In a virtual reality experiment, participants showed a preference for complex facades, with an average score of 3.82 out of 10.
Post-survey ratings averaged 4.9 on a 7-point scale, reflecting positive attitudes toward complex designs.
These results align with the historical analysis, reinforcing the shift towards complexity in contemporary architecture.

