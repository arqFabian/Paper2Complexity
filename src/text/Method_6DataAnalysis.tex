%\subsection{Data Analysis and Validation}
%    \label{subsec:Data_analysis}
%    The final phase of our methodology involves a detailed analysis of the data collected during the experiments, crucial for validating the effectiveness of the `Complexity Analysis' system in evaluating facades.

It is structured as follows:

\begin{enumerate}
    \item \textit{Data Processing and Analysis:} ...Advanced statistical tools
    \item \textit{Performance Evaluation:} We assess the `Complexity Analysis' system's impact on user by examining:
        \begin{itemize}
            \item \textit{Accuracy Analysis:} ...
            \item \textit{Participant Perception Survey:} ...
        \end{itemize}

        These metrics are critical

    \item \textit{Results Interpretation and Reporting:} Data synthesis ...
\end{enumerate}

%Closing statement

%

The final phase of our methodology focuses on analyzing the data collected during the experiments and the application of the CICA system to images of historical buildings, aiming to validate the effectiveness of the `Complexity Analysis' system in quantifying facade complexity and aligning it with user perceptions.

\textit{Data Processing and Analysis:} We assess CICA scores from historical buildings to identify patterns across different architectural styles and analyze experiment data using statistical tools to understand perceptions of complexity.

\textit{Performance Evaluation:} The efficacy of the `Complexity Analysis' system and the CICA score is assessed through:
\begin{itemize}
    \item \textit{Accuracy Analysis:} Evaluating the alignment between CICA scores and user perceptions.
    \item \textit{Participant Perception:} Analyzing user feedback to gain insights into the impact of complex facades.
\end{itemize}

\textit{Results Interpretation and Reporting:} Synthesizing data to confirm the validity of the CICA system and its applicability in architectural design.

This structured approach ensures a thorough evaluation, providing insights into the relationship between facade complexity and user perception.


